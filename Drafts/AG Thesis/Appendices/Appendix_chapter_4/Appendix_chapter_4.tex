
\chapter[Appendix D]{Appendix D}
\label{app:D}


\section{In-Depth Interviews}
\label{app:D1}

Table \ref{tab:art3-app-tab1} on the next page provides some additional information about the in-depth interviews.

\begin{landscape}
\begin{ThreePartTable}
\begin{TableNotes}
\footnotesize
\item [a] To safeguard the anonymity of our interviewees, we only make a distinction between the editorial board (including directors, CEO’s, and editors) which might have more impact on the selection of news items and journalists who might have more impact on the framing of selected news items.
\item [b] In a few interviews, religion was not asked.
\end{TableNotes}
\begin{longtable}{@{}clllp{3.3cm}p{8.5cm}@{}}
\caption{Information In-Depth Interviews}
\label{tab:art3-app-tab1}
\small \\
\toprule
\hline
No. & Date & Function\tnote{a} & Religion\tnote{b} & Outlet & Explanation Outlet \\* \midrule
\endfirsthead
%
\multicolumn{6}{c}%
{{Table \thetable\ continued \dots}} \\
\toprule
No. & Date & Function\tnote{a} & Religion\tnote{b} & Outlet & Explanation Outlet \\* \midrule
\endhead
\hline
\multicolumn{6}{r}{\textit{Continued on next page}} \\
\endfoot
\hline
\insertTableNotes  % tell LaTeX where to insert the contents of "TableNotes"
\endlastfoot
%
I1 & 9/11/2016 & Editorial board & Muslim & Radio Nigeria & Radio Nigeria was launched in the late seventies and is owned and operated by   Federal Radio Corporation of Nigeria, the public radio network. It broadcasts   across the country in 15 languages, aiming to serve a diverse variety of   listeners. \\
I2 & 9/11/2016 & Editorial board & Christian & Radio Nigeria & (see above) \\
I3 & 9/11/2016 & Journalist & Christian & Radio Nigeria & (see above) \\
I4 & 11/11/2016 & N/A & Muslim & Nigerian Union of Journalist   (NUJ) & The   NUJ is a professional media organization aiming to connect journalists   nationwide with the information and opportunities they need to advance   professionally and improve media in Nigeria. According to their website, NUJ   had currently 5,091 members (both Christian and Muslim journalists). \\
I5 & 17/11/2016 & Journalist & Christian & News Agency Nigeria (NAN) & NAN is a news reporting   agency owned and run by the Federal Government of Nigeria. NAN operates   across the 36 states of Nigeria and is based in the capital Abuja. \\
I6 & 17/11/2016 & Journalist & Muslim & News Agency Nigeria (NAN) & (see above) \\
I7 & 21/11/2016 & Journalist & Christian & Silverbird TV & Silverbird Television is part of the Silverbird Group, a leading media/entertainment   business in sub-Saharan Africa. Silverbird TV commenced transmission in Lagos   in 2003. \\
I8 & 21/11/2016 & Journalist & Christian & Silverbird TV & (see above) \\
I9 & 21/11/2016 & Editorial board & Muslim & Daily Trust & Daily Trust is a privately-owned Nigerian newspaper based in Abuja with stronger   ties to the Northern states. Daily Trust began publishing in 1998 and is now one   of the largest circulating newspapers in Nigeria. \\
I10 & 21/11/2016 & Journalist & Muslim & Daily Trust & (see above) \\
I11 & 22/11/2016 & Journalist & Christian & Nigerian Television   Authority   (NTA) & NTA  is a Nigerian government-owned and partly commercial broadcaster that was   inaugurated in 1977. At inauguration it had a monopoly on television   broadcasting in the country which was broken in the 1990s. \\
I12 & 30/11/2017 & Editor & Muslim & The Nation & The   Nation is the one of the most widely circulated newspapers in Nigeria. It is   affiliated to the All Progressives’ Congress political leader Bola Tinubu   (formerly Action Congress of Nigeria). \\
I13 & 1/12/2017 & Editor & Christian & Authority & Authority   is a Nigerian daily with a Southern base. \\
I14 & 1/12/2017 & Editor & Muslim & News Diary & News   Diary is an online news outlet in Nigeria. \\
I15 & 4/12/2017 & Editor & Christian & National Mirror & National   Mirror is a Nigerian daily with a Southern base. \\
I16 & 4/12/2017 & Editor & Christian & Tribune & Tribune   is a Nigerian daily with a Southern base. \\
I17 & 4/12/2017 & Editor & Christian & Nigerian Pilot & The   Nigerian Pilot is a Nigerian daily, which can be associated with the   South-South/South-East regions and the Peoples Democratic Party (PDP) \\
I18 & 7/12/2017 & Editor & Muslim & Voice of Nigeria & Voice   of Nigeria or (VON) is the official international broadcasting station of   Nigeria. \\
I19 & 7/12/2017 & Editor &  & Daily Trust & (see above) \\
I20 & 12//12/2017 & Editor & Muslim & News Agency Nigeria (NAN) & (see above) \\
I21 & 12//12/2017 & Journalist & Muslim & News Agency Nigeria (NAN) & (see above) \\
I22 & 18/12/2017 & Editor & Christian & The Guardian & The   Guardian is the one of the most widely circulated newspapers in Nigeria. It   is privately owned business, but is sometimes considered to be biased towards   the South and/or The People’s Democratic Oarty (PDP) \\
I23 & 18/12/2017 & Editor & Christian & The Guardian & (see above) \\
I24 & 18/12/2017 & Editor &  & The Guardian & (see above) \\
I25 & 19/12/2017 & Editor & Christian & News Agency Nigeria (NAN) & (see above) \\
I26 & 19/12/2017 & Editor &  & The Nation & (see above) \\* \hline
\bottomrule
\end{longtable}
\end{ThreePartTable}
\end{landscape}


\section{Sampling Protocol and Codebook}
\label{app:D2}

\subsection{Sampling Protocol}

Newspaper data for\textit{ the Daily Trust} and \textit{Guardian} newspapers are available in digital format from April 2014 to March 2015. An issue appears each day, even on Sundays (except for some holidays). To select news reports related to Boko Haram, we use the following procedure:
\begin{itemize}[noitemsep, nolistsep]
    \item[--] We upload the pdfs into Nvivo
    \item[--] We use the search query for “Boko Haram” per issue
    \item[--] We select a random number between 1 and the total number of references to “Boko Haram” in the issue (via random.org)
    \item[--] We select the report which contains the Xth reference to Boko Haram (the more references a report has, the higher the chance of selection)
    \item[--]We select one report per issue (initially). Some reports continue from last week or yesterday, yet these parts are excluded from the analysis. 
\end{itemize}


\subsection{Codebook}

\subsubsection{General Information}

\begin{enumerate}[noitemsep, nolistsep]
    \item Sampler ID: Sampler identification number
    \item Newspaper: Daily Trust or Guardian
    \item Newspaper issue date: Integer numbers indicating (a) day, (b) month, and (c) year of publication
    \item Total BH count: Integer number of the total number of Boko Haram references in the issue
    \item Random number: Integer number between 1 and total number of Boko Haram references in the issue via https://www.random.org/integers/ . \textit{Note}: if the selected report is an advertisement or singular photo caption repeat the process.
    \item Page: Page of selected report, put the page where the headline starts
    \item Total Pages: Total number of pages of the issue
    \item Headline: Main headline of report in text
    \item Type of report:
    \begin{enumerate}[noitemsep, nolistsep]
        \item[(0)] Standard news report
        \item[(1)] Opinion piece/column
        \item[(2)] Interview
        \item[(3)] Editorial
        \item[(4)] Foreign (i.e., taken from New York Times, Le Monde etc.)
        \item[(5)] Other
    \end{enumerate}
    \item Write name and position of writer/interviewee in case of opinion pieces/columns and interviews
    \item Word count report: Integer number of total number of words in the report (without headline, journalist names). The number of words can be an approximation if difficulties occur in the copy-pasting process.
    \item Coder ID: Coder identification number
\end{enumerate}

\vspace{3mm}

\subsubsection{A.	What CAUSES of the Boko Haram crisis are mentioned in the report? }

\textit{These questions concern the mentioning of different causes of the Boko Haram crisis in news reports. It is important that something is indeed \textbf{represented as a (possible) \underline{cause} of the insurgency} rather than as a consequence. For example, if poverty is mentioned, make sure that it is linked with the outbreak of the Boko Haram crisis and not simply represented (as a consequence). The variables concern the mentioning of causes, regardless of whether the author or interlocutor supports, criticizes or nuances the issue as a cause of the insurgency.
}

\begin{enumerate}[noitemsep,font=\bfseries]
   \item \textbf{Does the article mention \underline{socio-economic causes} of the insurgency?} Socio-economic causes include poverty, deprivation, inequality, unemployment, lack of education etc. either of the individuals participating in the insurgency or the North(-East) in general.
    \item \textbf{Does the article mention that \underline{political leaders/elites} have a role in the crisis?} This includes the arming of Boko Haram by political elites, the manipulation of Boko Haram to win elections, political corruption, malfunctioning of the state structure and governance, hierarchical/authoritarian systems, lack of leadership etc.
    \item \textbf{Does the article mention \underline{religious beliefs, religious radicalism or extremism} as the cause of the insurgency?} This includes explicit references to (the nature of) Islam as a cause/driver of the insurgency, mentioning of BH as driven by religious fundamentalism/extremism, or references to the goals of Boko Haram to Islamize society, establish a caliphate, destroy Western education\dots This also includes reference to Islamic education (the al majiri system) as a cause of the insurgency. This does not include the mere mentioning of the ‘Islamic terrorist group’
    \item \textbf{Does the article explicitly \underline{distance the Boko Haram insurgency from Islam?}} This can include arguments that Boko Haram does not represent true Islam, that they misinterpret the teachings, make false religious claims, hide under the cloak of religion, that they will be punished by their God etc.
\end{enumerate}

\noindent Answer categories: Yes (1)/No (0)

\subsubsection{B.	What CONSEQUENCES of the Boko Haram crisis are mentioned in the report?}

These questions concern the mentioning of different consequences of the Boko Haram crisis. As before, please make sure an issue is \textbf{represented as a \underline{consequence}} rather than a cause. The variables concern the mentioning of consequences, regardless of whether the author or interlocutor supports, criticizes or nuances the issue as a consequence of the insurgency.

\begin{enumerate}[noitemsep, font=\bfseries]
    \item \textbf{Does the article mention \underline{socio-economic consequences} of the Boko Haram crisis?} This includes infrastructure destruction, houses burnt, cattle looted, livelihoods lost as a result of the crisis, loss of educational opportunities, interruption of economic activities (e.g., cessation of farming) etc. The death of traders (including hawkers, peddlers etc.) are both socio-economic and human consequences.
    \item \textbf{Does the article mention \underline{human consequences} of the Boko Haram crisis?} This includes the mentioning of wounded and lives lost, as well as the consequences for internally displaced people having fled the crisis (families separated, dire conditions in refugee camps). This does NOT include extra-judicial killings/arbitrary arrests by security services, army, and police.
    \item \textbf{Does the article mention \underline{psychological consequences} of the Boko Haram crisis, including that it evokes fear, suspicion, distrust, despair, hopelessness etc. (‘negative emotions’) in society? }This includes descriptions of grieving/suffering people (e.g., Chibok parents), fearful people (e.g., not daring to go out) etc.
    \item \textbf{Does the article mention that \underline{Christians} are suffering from the Boko Haram crisis (both in terms of socio-economic and human consequence)?} This includes the death of Christians, abducted Christians, attacked churches, or the mere mentioning that Christians are affected. Labelling victims as Christian suffices. Also pay attention to Christian names mentioned as victims, usually English names (e.g., Samuel, Emmanuel, Innocent, Blessing, Gloria, Precious, Faith, Mercy etc.). This does NOT include extra-judicial killings/arbitrary arrests by security services, army, and police.
    \item \textbf{Does the article mention that \underline{Muslims} are suffering from the Boko Haram crisis (both in terms of socio-economic and human consequence)?} This includes the death of Muslims, abducted Muslims, attacked Mosques, or the mere mentioning that Muslims are affected. Labelling victims as Muslims suffices. Also pay attention to Muslim names mentioned as victims (e.g., Muhammed, ‘Alhadji’, Abubakar, Yusuf, Abdullahi, Aisha, etc. NOT Yakubu) or statements of emirs/Muslim leaders that their people are affected. This does NOT include extra-judicial killings/arbitrary arrests by security services, army, and police.
\end{enumerate}

\noindent \textit{Note:} When the article mentions that both Christians and Muslims were affected, please code both question B3 and B4 as “yes”.

\vspace{3mm}
\noindent Answer categories: Yes (1)/No (0)

\subsubsection{C.	What SOLUTIONS to the Boko Haram crisis are suggested in the report?}


\textit{These questions concern the solutions and measures offered in news reports as possible ways out of the crisis. The Chibok girls kidnapping is subsumed under the Boko Haram crisis (e.g., military struggle for release or negotiations). The variables concern the mentioning of solutions, regardless of whether the author or interlocutor supports, criticizes or nuances the issue as a solution to the insurgency (unless otherwise noted).}

\begin{enumerate}[noitemsep, font=\bfseries]
    \item \textbf{Does the article mention \underline{socio-economic development} as a solution to the conflict?} This includes better and more education opportunities, more investment and employment, more public spending, higher resource allocations to the North-East etc.
    \item \textbf{Does the article mention \underline{diplomacy/reconciliation/deradicalization} as a solution to the conflict?} This includes negotiations and mediation, dialogue, the use of amnesty measures, reconciliation efforts or DDR policies (Disarmament, Demobilization, Reintegration) and deradicalization programs. Mentioning of carrot and sticks approaches fits this category as well.
    \item \textbf{Does the article mention \underline{military/police action} as a solution to the conflict?} This solution can concern military, police, security services (DSS) etc. and includes mentioning a need for better equipment, more action on the terrain, more troops, the reconquering of territory, increased foreign support in the form of combat troops in the North-East and support in terms of military financing, equipment, and intelligence. Note that the military can be criticized while a military solution is still called for. Not implicit (e.g., critique of too little security does not imply mentioning of security-focused solution unless explicitly stated)
\end{enumerate}

\noindent Answer categories: Yes (1)/No (0)


\begin{enumerate}[noitemsep, font=\bfseries]
    \item[4.] \textbf{Does the article present a positive, negative, or neutral perspective on the way the crisis has been handled by the President Jonathan, his federal government, and/or the armed forces} (incl. army/soldiers, security agencies, and police forces)?
\end{enumerate}


\noindent Answer categories:
\begin{enumerate}[noitemsep, nolistsep, font=\bfseries]
    \item[(0)] Very negative: the article heavily criticizes the army/government and this critique forms the main part of the article’s argument
    \item[(1)] Negative: Some critical comments are made, but this is not a major focus of the article
    \item[(2)] Neutral: no statements on the army/government’s actions are being made or there are arguments for and against which balance each other out
    \item[(3)] Positive: the article praises/defends the army/government in some comments but this is not the major focus of the article. Defending can occur vis-à-vis opposition claims, critiques etc.
    \item[(4)] Very positive: the article praises/defends the army/government and this praise forms the main part of the article. Defending can occur vis-à-vis opposition claims, critiques etc.
\end{enumerate}


\subsubsection{D.	Which SEMANTICS are used in the report to refer to Boko Haram?}


\begin{enumerate}[noitemsep, font=\bfseries]
    \item \textbf{Does the article use language referring to the religious  nature of Boko Haram?} For example, the Islamic terror group, suspected Islamists etc. But also, they came in shouting Allahu Akbar, cited verses of the Koran, were praying, their religious fervor etc.
    \item \textbf{Does the article use language depicting the Boko Haram crisis as a criminal activity?} e.g., crimes, criminals, bandits, thugs, villains etc. NOT murderers, armed gunmen/attackers.
\end{enumerate}

Answer categories: Yes (1)/No (0)

\subsubsection{E.	Does the article quote any or all of the following SOURCES with regard to the Boko Haram crisis? }


\textit{\underline{Quotes} refer to direct citations, copying or paraphrasing of statements. Not as authors of opinion pieces or as interviewees. Yes if interviewee cites Muslim authority, for example.}.

\begin{enumerate}[noitemsep, font=\bfseries]
    \item \textbf{Nigerian politician?} A politician can be a government representative, representative of the All Progressives Congress (APC), the People’s Democratic Party (PDP), Labour Party, or APGA. Politicians also include (former) elected officials such as the president (e.g., Yakubu Gowon, Ibrahim ‘IBB’ Babangida, Olusegun Obasanjo), governors, assembly representatives, lawmakers, senators, local councilors and chairmen, commissioners etc. NOT civilian joint task force
    \item \textbf{Nigerian Generals and other army or police representatives?} If former army men have become politicians, politician takes precedence and should be the only category filled in.
    \item \textbf{Christian authority or the Bible?} A Christian authority can be the Pope, a Pastor, Bishop, Reverend, Minister (religiously interpreted), or a representative from a Christian association (e.g., CAN, the Christian Association of Nigeria). Clerics can be both Christian and Muslim, so do not derive meaning from the word cleric alone.
    \item Muslim authority or the Quran? A Muslim authority can be an Imam, Emir, Sultan, Sardauna, or Waziri. Clerics can be both Christian and Muslim, so do not derive meaning from the word cleric alone.
    \item\textbf{ Civil society actor? }A civil society actor can be a NGO or other private organization, both Nigerian and foreign (e.g., Amnesty, International Crisis Group, elders fora, community association, vigilante grouping etc.). Civil society also includes academics (e.g., professors), writers (Wole Soyinka), poets etc. Civil society actors include protesters (e.g., Chibok girls protest) ONLY if a specific organization is mentioned (e.g., BBOG or Bring Back our Girls)
    \item \textbf{The public?} This includes both victims and other ordinary citizens (e.g., witnesses of an attack, hospital sources). This also includes protesters if no organization is mentioned. This does not include statements such as ‘Nigerians think’, ‘the majority believes that’ etc.
    \item \textbf{Boko Haram itself?} This includes statements from the leadership, messages from their videos, channels, etc. This also includes witnesses/victims stating that HH told them this or that.
\end{enumerate}

Answer categories: Yes (1)/No (0)
