
\vspace*{1mm}
\begin{flushright}
\Large{How to Forgive Former Fighters? \\
A Short Summary}
\end{flushright} 

\vspace{1cm}
\normalsize

\begin{Box1}{Lessons Learned !}
\begin{itemize}[noitemsep]
\item  Information on \textit{why} someone joined or left the Boko Haram insurgency affects post-conflict reintegration attitudes, with ex-fighters who were forced to join and left out of remorse being most welcome.
\item  Information on \textit{what} actions someone undertook since leaving the Boko Haram insurgency equally affects post-conflict reintegration attitudes, with ex-fighters helping the police and military combating Boko Haram being most welcome.
\item These effects seem to operate through \textit{belief in the success of reintegration}, and hold for both Christians and Muslims, victims and non-victims, and angry and non-angry respondents. 
\end{itemize}
\end{Box1}

\vspace{1cm}

\begin{Box1}{Persistent Puzzles ?}
\begin{itemize}[noitemsep]
\item  Why do common political-psychological constructs, in particular terrorism exposure and feelings of anger, \textit{not} moderate these relations?
\item Why are \textit{no indications of in-group bias} found? In other words, why are our Nigerian respondents \textit{not} more likely to punish those ex-fighters who primarily targeted their own religious in-group and vice-versa?
\end{itemize}
\end{Box1}
