
\vspace*{1mm}
\begin{flushright}
\Large{Hey, They Aren't Muslims!? \\
A Short Summary}
\end{flushright} 

\vspace{1cm}
\normalsize

\begin{Box1}{Lessons Learned !}
\begin{itemize}[noitemsep]
\item In line with much Western-based conflict news coverage, Nigerian newspapers tend to apply a \textit{war narrative} when reporting on Boko Haram.
\item In sharp contrast to much Western-based terrorism news coverage, Nigerian newspapers generally do \textit{not} trace the root cause of Boko Haram back to Islam and Muslims as such. This is true for \textit{both} the Northern (hence, Muslim-affiliated) and Southern (hence, Christian-affiliated) newspaper. 
\item Overall, both newspapers report in very similar ways on the Boko Haram conflict, which---again---contradicts the often-assumed in-group bias inherent in `terrorism' news coverage.
\item A micro-, meso-, and macro-level mechanism, based on \textit{intergroup contact theory and power hierarchies}, is proposed to explain how a country's religious-political demography may influence conflict news coverage---regardless of the outlet's background and readership.
\end{itemize}
\end{Box1}

\vspace{1cm}

\begin{Box1}{Persistent Puzzles ?}
\begin{itemize}[noitemsep]
\item Why did we, \textit{a priori}, hypothesize an ethnocentric, one-sided, in-group bias in Boko Haram news coverage? 
\item Why are intergroup relations---a factor of great importance from a social-psychological perspective---not more central to the study of terrorism news coverage and terrorism in general?
\end{itemize}
\end{Box1}
