% GEOMETRY
\usepackage[twoside, margin=1.5in]{geometry}
%\usepackage[twoside,margin=1.45in,bindingoffset=0.1in]{geometry}
%\usepackage[twoside,bindingoffset=6mm,verbose]{geometry}%
%\usepackage[twoside, margin=1.42in,bindingoffset=0.4in]{geometry}

% APA BIBLIOGRAPHY/REFERENCES
\usepackage[natbibapa, nodoi]{apacite} 
\renewcommand*{\bibfont}{\small}
%Dieter: %https://ctan.math.illinois.edu/macros/latex/contrib/apacite/apacite.pdf
%\usepackage[round,authoryear]{natbib}   % ORIGINALLY NATBIB WAS USED - I CHANGED TO APACITE. IF YOU USE NATBIB AGAIN, ACTIVATE TWO LINES BELOW (USEPACKAGE NATBIB AND BIBPUNCT)
\usepackage{etoolbox}
\renewenvironment{APACrefURL}[1][]{}{}
\AtBeginEnvironment{APACrefURL}{\renewcommand{\url}[1]{}}
\renewcommand{\doiprefix}{doi:~\kern-1pt}
%Amélie: I removed the doi's and URL's from the references. For websites, I "hacked" the code a bit and put the url into the publisher's argument so that it was printed. It's ugly but it works ;-)

% TOC SETTINGS
\usepackage[nottoc]{tocbibind}  % adds index/table of contents automatically
% add: [notlot,notlof] if you want to exclude list of tables and figures from TOC as well.
\setcounter{secnumdepth}{2} % increases/decreases the amount of subsections that are numbered.
\setcounter{tocdepth}{1} % only includes chapter (0) & section (1) titles in toc.
\makeatletter % gets rid of spacing between T or Fs from different chapters
% \patchcmd{<cmd>}{<search>}{<replace>}{<succes>}{<failure>}
\patchcmd{\@chapter}{\addtocontents{lof}{\protect\addvspace{10\p@}}}{}{}{}% LoF
\patchcmd{\@chapter}{\addtocontents{lot}{\protect\addvspace{10\p@}}}{}{}{}% LoT
\makeatother


% GENERAL SETTINGS
\usepackage[T1]{fontenc}  % to add the accent on Amélie ;-) and other accents etc.
\usepackage{lettrine}
\usepackage{rotating}   % performs all the different sorts of rotation one might like
\usepackage{setspace}  % setting the spacing between lines in a document
\frenchspacing % no space behind full stop
\usepackage{url}
\usepackage{enumitem} % to delete spacing between itemize/enumerate
\usepackage{verbatim} % for block comments
\usepackage[all]{nowidow} % geen losse lijntjes op het einde of begin van een pagina (!) 
\newenvironment{myquote}{\begin{quote} \small}{\end{quote}} % sets long quotes in small(er) font
\usepackage{epigraph}
\setlength\epigraphwidth{.8\textwidth}
\setlength\epigraphrule{0pt}
% distance between paragraphs and indentation:
\setlength{\parindent}{2em}
\setlength{\parskip}{1pt}
% distance between (sub)section titles and text:
\usepackage{titlesec}
\titlespacing*{\section}
  {0pt}{1.5\baselineskip}{0.5\baselineskip}
\titlespacing*{\subsection}
  {0pt}{1.5\baselineskip}{0.5\baselineskip}
%Indents
%\parindent0em
%\parskip\baselineskip

% LESSONS LEARNED BOX
\usepackage{tikz}
\usepackage[most]{tcolorbox}
\newtcolorbox{Box1}[2][]{
                lower separated=false,
                colback=white!80!gray,
colframe=white, fonttitle=\bfseries,
colbacktitle=white!50!gray,
coltitle=black,
enhanced,
attach boxed title to top left={xshift=0.5cm,yshift=-2mm},
title=#2,#1}
\usepackage[linewidth=1pt]{mdframed} % to put boxes around things

% SETTINGS APPENDIX
\usepackage[]{appendix}

% MATHTOOLS
\usepackage{amsfonts}
\usepackage{amsmath}
\usepackage{amsthm}
\usepackage{mathtools}
\usepackage{algorithm2e}

% HYPERLINKS
\usepackage{xcolor,hyperref} 
\hypersetup{
    colorlinks=TRUE,
    linkcolor=black,
    filecolor=black,      
    urlcolor=black,
    citecolor=black,
    citebordercolor=black,     % color of links to bibliography
    }
\usepackage{fancyref}

% TABLES AND FIGURES:
\usepackage{longtable}  % long tables: continue on next page
\usepackage{lscape} % wide tables: landscape
\usepackage{pdflscape}
\usepackage[usestackEOL]{stackengine}   % to use "Centerstack" instead of "shortstack"
\usepackage{booktabs} % for well-spaced horizontal rules
\usepackage{supertabular} % for supertables (?)
\usepackage{multirow}
\usepackage{tabularx} % for tables
\usepackage{graphicx}
\usepackage[flushleft]{threeparttable} % for tables with footnotes
\usepackage{threeparttablex} % for "ThreePartTable" environment
\usepackage{float}
\usepackage{adjustbox}  % for centering
\usepackage{array}
\usepackage{dcolumn}
\newcolumntype{d}[1]{D..{#1}}
\newcommand\MC[1]{\multicolumn{1}{c}{#1}} 
\newcolumntype{L}[1]{>{\raggedright\arraybackslash}p{#1}}
\newcolumntype{C}[1]{>{\centering\arraybackslash}p{#1}}
\newcolumntype{R}[1]{>{\raggedleft\arraybackslash}p{#1}}
\usepackage{makecell}  % voor lange cellen
\renewcommand\theadalign{vl}
\renewcommand\theadfont{\bfseries}
\renewcommand\theadgape{\Gape[4pt]}
\renewcommand\cellgape{\Gape[4pt]}
\usepackage{subcaption}
\usepackage[labelfont=bf,textfont=normalfont,labelsep=period,singlelinecheck=off,justification=raggedright]{caption} % captions of tables and figures
% caption label is bold, the caption text normal.
% period after caption label.
% justification is raggedright (i.e. left aligned).
% singlelinecheck=off means that the justification setting is used even when the caption is only a single line long. if singlelinecheck=on, then caption is always centered when the caption is only one line.
\newcommand{\tabitem}{~~\llap{\textbullet}~~}
\newcommand\mulcol[2]
  {%
    \mulcolA{#1}#2\endmulcolsecarg
  }
\newcommand\mulcolA{}
\long\def\mulcolA#1#2#{\mulcolB{#1}{#2}}
\newcommand\mulcolB{}
\long\def\mulcolB#1#2#3#4\endmulcolsecarg
  {%
    \multicolumn
      {#1}
      {%
        #2%
        {%
          \dimexpr
            #3%
            +#1\tabcolsep+#1\tabcolsep-2\tabcolsep
            +#1\arrayrulewidth-\arrayrulewidth
          \relax
        }%
        #4%
      }%
  }

% FANCY HEADERS
\usepackage{fancyhdr}
\pagestyle{fancy}
% Make sure headers are uppercase
\renewcommand{\chaptermark}[1]{\markboth{\uppercase{#1}}{\uppercase{#1}}}
\renewcommand{\sectionmark}[1]{\markright{\uppercase{#1}}}
\fancyhf{}
\fancyhead[RO]{\scriptsize\sf\rightmark\ \hrulefill\ \thepage}
\fancyhead[RE]{\scriptsize\sf\thepage\ \hrulefill\ \leftmark}
\renewcommand{\headrulewidth}{0pt}
\fancypagestyle{plain}{
\fancyhf{}
\fancyfoot[C]{\scriptsize\sf\thepage}
\renewcommand{\headrulewidth}{0pt}
\renewcommand{\footrulewidth}{0pt}
}
%====to remove the header====
\fancypagestyle{nofooter}{%
\fancyfoot{}%
}
\fancypagestyle{noheader}{%
\fancyhead{}%
}


% FANCY FOOTNOTES
\usepackage[bottom,norule,hang]{footmisc}
\setlength{\footnotemargin}{0.9em}

% LAY-OUT START CHAPTERS
% 1. Remove headers and footers on pages between chapters
\usepackage{emptypage}
% 2. Add decorative quote at the beginning of each chapter
\usepackage{quotchap}   
% 3. Lay-out chapter number and title
\makeatletter
%\renewcommand*{\sectfont}{\bfseries} % title in bold
\renewcommand*{\chapnumfont}{%
  \usefont{T1}{\@defaultcnfont}{b}{n}\fontsize{80}{100}\selectfont % Spacing default: 100/130
  \color{chaptergrey}%
}
\makeatother
% 4. Add abstract at the beginning of each chapter
\usepackage{changepage}% http://ctan.org/pkg/changepage
\makeatletter
\newenvironment{chapabstract}{%
    \par\setlength{\leftskip}{0.7cm}\noindent\ignorespaces
    \par\setlength{\rightskip}{0.7cm}\noindent\ignorespaces
    \small  
    \begin{center}%
        \bfseries Abstract
    \end{center}}%
   {\par}
\makeatother

%ADD BLANK END PAGE
\usepackage{afterpage}
\newcommand\blankpage{%
    \null
    \thispagestyle{empty}%
    \addtocounter{page}{-1}%
    \newpage}
