
\chapter*{Summary/Samenvatting}
\chaptermark{Summary/Samenvatting}
\addcontentsline{toc}{chapter}{Summary/Samenvatting}

\section*{Summary}
 
How does terrorism affect citizens' social and political attitudes? And how can we explain differences in attitudinal reactions between individuals and across societies? Scholars, policymakers, and the public alike often assume that terrorism is effective to the extent it is able to influence the electorate. As a result, an impressive body of literature has accumulated—especially since the 9/11 attacks—on citizens' attitudinal responses to terrorism. Yet, most of these studies focus on individual-level explanations within Western or Israeli settings. How terrorism changes citizens across different, and especially within non-Western, societies remains unknown.

By combining comparative research designs with an in-depth case-study of the Boko Haram conflict in Nigeria, this dissertation explores (1) how citizens across the globe react to terrorism, (2) how both citizen and perpetrator characteristics regulate such terrorism-induced reactions, and (3) how journalists are shaping an image of `the terrorist enemy.' The findings suggests that it might not be terrorism as such that is changing citizens, but rather the \textit{idea} of terrorism---an idea determined to a large extent by specificities of the \textit{intergroup context} in which threats or acts of violence take place. 


In short, this dissertation is a multimethod and interdisciplinary story—straddling the fields of social and political psychology, political science, and media studies—about the formation of attitudes in times of terror. The findings are not only relevant for scholars studying how citizens respond to terrorism but also make substantial inroads in how the discipline studies terrorism effects and how various stakeholders can design effective strategies to counter potential democratic backlashes in times of terror.



\clearpage


\shipout\null   %Add blank page (so Dutch summary starts on left page again)
\stepcounter{page}  %Add one page to page numbers, so the order is still correct


\section*{Samenvatting}
Hoe beïnvloedt terrorisme de sociale en politieke attitudes van burgers? En hoe kunnen we verschillen in reacties tussen individuen en over samenlevingen heen verklaren? Wetenschappers, beleidsmakers en het publiek gaan er vaak van uit dat terreur effectief is in die mate dat het kiezers beïnvloedt. Daarom, en vooral sinds de aanslagen van 11 september, hebben enorm veel studies onderzoek gedaan naar hoe en waarom burgers reageren op een terroristische aanslag. Echter, de meeste van deze onderzoeken focussen op verklaringen op individueel niveau binnen westerse of Israëlische samenlevingen. Het blijft een groot vragenteken hoe terrorisme burgers over verschillende samenlevingen heen, en vooral binnen niet-westerse samenlevingen, verandert.

Door vergelijkende onderzoeksontwerpen te combineren met een diepgaande case-study van het Boko Haram-conflict in Nigeria, onderzoekt dit werk (1) hoe burgers over de hele wereld reageren op terrorisme, (2) hoe kenmerken van zowel burgers als daders dergelijke door terrorisme veroorzaakte reacties reguleren, en (3) hoe journalisten vormgeven aan een beeld van `de terroristische vijand'. De bevindingen die hieruit voorvloeien suggereren dat het misschien niet terrorisme als zodanig is dat de burger verandert, maar eerder het \textit{idee} van terrorisme---een idee dat in sterke mate wordt bepaald door specifieke kenmerken van de \textit{intergroepscontext} waarin geweld plaatsvindt. 


Kortom, dit proefschrift is een multimethodisch en interdisciplinair verhaal---dat inzichten uit de sociale en politieke psychologie, politieke wetenschappen en mediastudies combineert---over de vorming van attitudes in tijden van terreur. De bevindingen zijn niet alleen relevant voor wetenschappers die bestuderen hoe burgers reageren op terrorisme, maar leveren ook een aanzienlijke bijdrage aan kennis over hoe de discipline de effecten van terrorisme bestudeert alsook hoe verschillende belanghebbenden effectieve strategieën kunnen ontwerpen om een mogeljke democratische terugval in tijden van terreur tegen te gaan.
