
\documentclass[12pt]{article}
\usepackage{graphicx}
\usepackage{caption}
\usepackage{natbib}
\usepackage{authblk}
\usepackage[utf8]{inputenc}
\usepackage{setspace}
\usepackage{rotating}
\usepackage[british]{datetime2}


\renewcommand\Affilfont{\itshape\small}

\title{Geo-ISD vs Paine}
\author[1]{Marius Swane Wishman}
\affil[1]{Department of Sociology and Political Science, NTNU}

\date{\today}

\providecommand{\keywords}[1]
{
	\small	
	\textbf{\textit{Keywords---}} #1
}

\begin{document}

\maketitle

\begin{abstract}
\end{abstract}

\keywords{}

\pagebreak

%tableofcontents
%\pagebreak

\onehalfspacing

\section{Introduction}

The literature on historical state entities (hereafter HSE's), pre-colonial
states and so on, and conflict has drawn mixed conclusions.  The general
tendency has been toward that HSE's are often locally peace inducing
\citep{Wig2016, Wig2018} but nationally conflict inducing \citep{Paine2019}.
Employing new data allows this paper to test the existing theories in the
literature, and answer the question of where, relative to HSE borders, does
modern (post 1946) conflict occur.

\section{Literature/theory}

\subsection{Recap \citet{Wig2016} and counterpoints}

\citet{Wig2016} argues that ethnic groups with ties to pre-colonial statehood
are more likely to have inherited institutions that allow the ethnic group to
punish defections and hold their leaders accountable.  In this way, ethnic
groups with ties to pre-colonial statehood are better able to make credible
commitments, than 'non-state' ethnic groups.  Credible commitments help such
groups both prevent conflict from occurring in the first place, but also make
them better able to end conflicts when then they have broken out.  Empirically
\citet{Wig2016} finds that groups with histories of statehood do indeed
experience less dyadic conflict with their government.
\citet{Depetris-Chauvin2016} makes a similar argument and finds that regions
with exposure to pre-colonial statehood are more peaceful, \emph{ceteris
paribus}.

On the other hand, someone else found the opposite. Possibly \citet{Besley2014}.

\subsection{Recap \citet{Paine2019}}

\subsection{Conflict reducing}

\subsubsection{Internal monopoly of violence}

The \citet{Tilly1990} argument, TODO: infuse more Tilly into the mix, other
relevant authors? States as stationary bandits gradually remove internal
competitors. Over time this reduces the number of actors within the borders of a
state that are able to wield organized forms of violence, and the remaining
ones' ability to do so. In the case of pre-colonial states, they are now either
once again `the' state (for example Morocco or Ashanti/Ghana), have been
incorporated into a larger state as part of its apparatus, or had its
institutions destroyed by some larger state (colonial or indigenous)
consolidating its role as the sole stationary bandit within its borders. In
other words within the former borders of a pre-colonial state there should be a
reduced number of potential wielders of organized violence (ceteris paribus)
depending on the pre-colonial states centralization/consolidation, itself a
product of time, reforms/political organization/idiosyncrasies and the proximity
to its capital. If the pre-colonial state was incorporated only partially into
the modern state, it could still pose a threat to the central state through
desertion (more on this later). If the pre-colonial state was destroyed, for
example by colonizers, without new state (colonial or post-colonial) entering
the resulting power vacuum other actors would do so, and become new stationary
bandits rivalling the state. How does this compare to other areas not formally
part of a pre-colonial state? These areas could inhabit roving bandits
\citep{Scott2009} or other actors already having filled an equivalent vacuum of
power. In other words, in this scenario pre-colonial state areas should be no
worse than other areas in terms of violence. Any resulting conflict running
through this mechanism should occur shortly after decolonization.

\subsubsection{Better Angles}

Building on among others \citet{Tilly1990} 

\subsubsection{Alternatives to Violence}

Institutions of conflict resolution and bargaining

\subsubsection{Economic advantage?}

\subsection{Conflict inducing}

\subsubsection{Distance from capital}

\subsubsection{Central State Weakness/Collapse}

\subsubsection{Political Inequality}

\citet{Paine2019}'s argument. State groups either exclude other groups that
eventually revolt, or they are excluded themselves, and will reclaim their
dominating position.

\subsubsection{Multiple HSE's}

Bargaining problems and multiplies any other effect.

\section{Research design}

\section{Conclusion}

\pagebreak

\bibliographystyle{agsm}
\bibliography{../lib.bib}

\section{Appendix}

\end{document}

