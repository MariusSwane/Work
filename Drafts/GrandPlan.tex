
\documentclass[12pt]{article}
\usepackage{graphicx}
\usepackage{caption}
\usepackage{natbib}
\usepackage{authblk}
\usepackage[utf8]{inputenc}
\usepackage{setspace}
\usepackage{rotating}
\usepackage[british]{datetime2}
\usepackage{amssymb}

\renewcommand\Affilfont{\itshape\small}

\title{The Plan}
\author[1]{Marius Swane Wishman}
\affil[1]{Department of Sociology and Political Science, NTNU}

\date{\today}

\providecommand{\keywords}[1]
{
	\small	
	\textbf{\textit{Keywords---}} #1
}

\begin{document}

\maketitle

\begin{abstract}
	A plan for how to restart after parental leave and eventually finish the
	PhD.

\bigskip
\hangindent = 3.5em
\emph{`Plans are worthless, but planning is everything.' | 
		Dwight D. Eisenhower.
	}
\end{abstract}

\bigskip
\keywords{Productivity, goals, plan}

\pagebreak

%tableofcontents
%\pagebreak

\onehalfspacing

\section{Main Paper on HSE's and conflict}
\subsection{Literature}
\begin{itemize}
	\item[$\square$] The State and violence. General theories.
		Along the lines of \citet{Pinker2012} and \citet{Tilly1990}.
		With increasing state penetration comes an increasing monopoly
		on violence \citep{Tilly1990} and with this, certain
		cultural/psycho-social effects that translate into a generally
		more peaceful citizenry.
	\item[$\square$] Internal organized (as opposed to communal) violence.
	\item[$\square$] How old institutions affect current institutions
		(formal and non-formal or networks).
\end{itemize}

\subsection{Theory}
\begin{itemize}
	\item[$\square$] As predicted by the literature.
	\item[$\square$] Old institutions represent a measure of "governing
		capacity" that has hitherto been modestly explored. The more
		historical state presence the more people are used to and accept
		state intervention. 

		The state monopoly on violence should also be more consolidated.
		The degree of consolidation would probably vary with the degree
		of continuity between the historical state and the present
		state, but it should be a positive relationship in both cases. 

		The more historical state presence, the more time the cultural or
		psycho-social effects that \citet{Pinker2012} hypothesize have
		to affect the people in the area.
\end{itemize}

\subsection{Hypothesis}
	\bigskip
	\hangindent = 3.5em
	\textit{H\textsubscript{1}: Grid cells with higher levels of historical
		state presence experience less conflict.
		}
	\bigskip 

\subsection{Variables}
\begin{itemize}
	\item[$\square$] Theoretical and/or methodological grounding.
	\item[$\square$] Finding measures (data).
	\item[$\square$] Incorporate with the Geo-ISD data.

	\item[$\square$] Geography
	\begin{itemize}
		\item[$\square$] "Ruggedness", mountains or elevation.

			Provides shelter for rebels and prevents "exit-options"
			for subjects, to the benefit of stationary bandits
			(states).

			What is more, ruggedness could affect (negatively) the
			chances that cartographers and (to a lesser extent)
			historians have place a state there.

		%\item[$\square$] Desert.
		\item[$\square$] Rivers.
			Provide natural boundaries useful for state building.
			Navigable rivers bind cities together (again useful for
			state building). Rivers could often be navigated by
			explorers, thus increasing the likelihood that maps
			include a state there.
		\item[$\square$] Jungle.
	\end{itemize}
	\item[$\square$] Distance from X.
	\begin{itemize}
		\item[$\square$] Capital.
		\item[$\square$] Coast.
		\item[$\square$] Europe.
	\end{itemize}
	\item[$\square$] Exposure to slave trade.
	\item[$\square$] Population.
	\item[$\square$] Resources.
	\begin{itemize}
		\item[$\square$] Gold.
		\item[$\square$] Silver.
		\item[$\square$] Copper.
		\item[$\square$] Iron.
	\end{itemize}
	\item[$\square$] Spatial interdependence.
	\item[$\square$] Past Conflict.
	\item[$\square$] Regional and country-fixed effects.
\end{itemize}

\subsection{Data}
\begin{itemize}
	\item[$\square$] How the old maps were made.
	\item[$\square$] Compare old maps to new. Do they agree on the core
		areas?
	\item[$\square$] Benefits of the Geo-ISD.
	\item[$\square$] Drawbacks.
	\item[$\square$] Alternative measures.
\end{itemize}

\subsection{Analysis}
\begin{itemize}
	\item[$\square$] Cross section to avoid post-treatment bias.
	\item[$\square$] Alternative models.
\end{itemize}

\section{Communal violence}
\begin{itemize}
	\item[$\square$] Talk to Ole Magnus.
\end{itemize}
\section{}

\section{}

\pagebreak

\bibliographystyle{agsm}
\bibliography{../lib.bib}{}

\section{Appendix}

\end{document}

