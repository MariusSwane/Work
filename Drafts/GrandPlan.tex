
\documentclass[12pt]{article}
\usepackage{graphicx}
\usepackage{caption}
\usepackage{natbib}
\usepackage{authblk}
\usepackage[utf8]{inputenc}
\usepackage{setspace}
\usepackage{rotating}
\usepackage[british]{datetime2}
\usepackage{amssymb}

\renewcommand\Affilfont{\itshape\small}

\title{The Plan}
\author[1]{Marius Swane Wishman}
\affil[1]{Department of Sociology and Political Science, NTNU}

\date{\today}

\providecommand{\keywords}[1]
{
	\small	
	\textbf{\textit{Keywords---}} #1
}

\begin{document}

\maketitle

\begin{abstract}
	A plan for how to restart after parental leave and eventually finish the
	PhD.

\bigskip
\hangindent = 3.5em
\emph{`Plans are worthless, but planning is everything.' | 
		Dwight D. Eisenhower.
	}
\end{abstract}

\bigskip
\keywords{Productivity, goals, plan}

\pagebreak

%tableofcontents
%\pagebreak

\onehalfspacing

\section{Main Paper on HSE's and conflict | Send to journal before the summer}
\subsection{Literature | Deadline: End of March}
\begin{itemize}
	\item[$\square$] The State and violence. General theories.
		Along the lines of \citet{Pinker2012} and \citet{Tilly1990}.
		With increasing state penetration comes an increasing monopoly
		on violence \citep{Tilly1990} and with this, certain
		cultural/psycho-social effects that translate into a generally
		more peaceful citizenry.
	\item[$\square$] Internal organized (as opposed to communal) violence.

	\item[$\square$] How old institutions affect current institutions
		(formal and non-formal or networks). How old institutions were
		integrated into the new states. There is going to be endless
		variation on this, so I will need to find some illustrative
		examples. Emphasize that in the cases of formal integration and
		continued existence, these institutions will probably have
		\emph{some} effect on conflict because of their legacy and
		continued involvement in local (if not national) conditions. In
		addition, I should emphasize that even when no formal
		institutions survive, HSE's can still affect current politics
		and conflict. Use good illustrative examples to make these
		points. Could refer to some of the examples in the first paper
		to save some labor.
\end{itemize}

\subsection{Theory | Deadline: Early version by end of April}
\begin{itemize}
	\item[$\square$] As predicted by the literature.
	\item[$\square$] Old institutions represent a measure of "governing
		capacity" that has hitherto been modestly explored. The more
		historical state presence the more people are used to and accept
		state intervention. 

		The state monopoly on violence should also be more consolidated.
		The degree of consolidation would probably vary with the degree
		of continuity between the historical state and the present
		state, but it should be a positive relationship in both cases. 

		The more historical state presence, the more time the cultural or
		psycho-social effects that \citet{Pinker2012} hypothesize have
		to affect the people in the area.
\end{itemize}

\subsection{Hypothesis}
	\bigskip
	\hangindent = 3.5em
	\textit{H\textsubscript{1}: Grid cells with higher levels of historical
		state presence experience less conflict.
		}
	\bigskip 

\subsection{Variables | Deadline: Mid March}
\begin{itemize}
	\item[$\square$] Theoretical and/or methodological grounding.

	\item[$\square$] Finding measures (data).

	\item[$\square$] Incorporate with the Geo-ISD data.

	\item[$\square$] Geography
	\begin{itemize}
		\item[$\square$] "Ruggedness", mountains or elevation.

			Provides shelter for rebels and prevents "exit-options"
			for subjects, to the benefit of stationary bandits
			(states).

			What is more, ruggedness could affect (negatively) the
			chances that cartographers and (to a lesser extent)
			historians have place a state there.

		%\item[$\square$] Desert.

		\item[$\square$] Rivers.
			Provide natural boundaries useful for state building.
			Navigable rivers bind cities together (again useful for
			state building). Rivers could often be navigated by
			explorers, thus increasing the likelihood that maps
			include a state there.

		\item[$\square$] Jungle. Potentially some
			circumscription/limiting escape option effects, and
			certainly an impediment to Western explorers and
			surveyors.
	\end{itemize}

	\item[$\square$] Distance from X.
	\begin{itemize}
		\item[$\square$] Capital. In many cases the old capitals became
			the new capitals, around which I expect relatively high
			levels of HSP (historical state presence). Capitals
			probably (source) experience more of some forms of
			violence (coups, violent protests, etc., and less of
			other forms (combat deaths).

		\item[$\square$] Coast. Helpful for trade and communication and
			this state building. Dramatically increases the odds of
			being discovered by Western explorers. Also increases
			the odds of early colonization. Increases odds of being
			involved in slave trading. Connection to conflict is
			less clear, but perhaps through \citet{Nunn2008} slave
			trade leads to lasting trust issues leads to
			conflict-mechanism.

		\item[$\square$] Europe. Institutional "spillover" could lead to
			increased chance of HSP. Ease of discovery (when
			distance to coast and desert is taken into account) is
			certainly affected. Proximity to Europe could be
			correlated with conflict due to increased likelihood of
			being a transit-country for drugs and/or people being
			smuggled into Europe across the Mediterranean, and
			potentially through increased exposure to European
			neo-colonial interests. 
	\end{itemize}

	\item[$\square$] Exposure to slave trade. Partially explained in the
		distance to coast variable. Can be measured directly for some
		countries at least using data from \citet{Nunn2008}. Can be
		related to statebuilding in costal regions, and potentially
		state "de-building" inland. To capture slaves you must have an
		army. As stated by \citet{Tilly1990} among others, an increasing
		army creates the need for a state. Someone must surely have
		written something about this for East Africa. Inland I believe
		the slave trade (capturing of slaves) primarily led to
		devastation and not the "Tillian", $conflict \rightarrow state
		\rightarrow conflict \rightarrow etc.$, but I will need some sources
		on this.

	\item[$\square$] Population. Madison project data for historical
		population. Correlates with current population which affects the
		likelihood of conflict events. Related to state building (and
		thus HSP) because states need a certain population to coalesce
		and an inclution criteria in the ISD.

	\item[$\square$] Disease prevalence. Missionary mortality rate if no
		other measure is available. Could be better than extrapolating
		current levels of disease prevalence backwards. Disease levels 
		affect both state building and the penetration of
		European explorers. The connection to conflict is less clear.

	\item[$\square$] Resources. Potentially correlated to state building
		through trade, coinage and weapons (iron) and modern conflict
		through competition over resources. Only for extended controls
		models. Most likely not relevant. Good measures will be hard to
		come by.
	\begin{itemize}
		\item[$\square$] Gold.

		\item[$\square$] Silver.
			
		\item[$\square$] Copper.

		\item[$\square$] Iron.
	\end{itemize}

	\item[$\square$] Spatial interdependence. I need to think more about
		this. How big of a problem is it? How best to deal with it?

	\item[$\square$] Past Conflict. \citet{Brecke1999} data to control for
		the \citet{Tilly1990} mechanism of war made states and states
		made war.

	\item[$\square$] Regional and country-fixed effects. This should further
		alleviate some of the selection bias issues of the old maps as
		well as omitted variables bias.
\end{itemize}

\subsection{Data} \begin{itemize} \item[$\square$] How the old maps were made.
	They were very much part of the colonialist/imperialist project, despite
	professed scientific integrity (I get the sense that the two were not
	seen as mutually exclusive). Nevertheless they seem to have strived for
	accuracy. Indeed, that is also to the benefit of the colonialist
	project, as expeditions (military and exploratory) depended on accurate
	maps of the regions they entered. And these maps, or in most cases, the
	maps on which `our' maps were based, \emph{were} the maps used in the
	field. The cartographers thus had clear incentives not to exaggerate
	colonial domains or make light of local kingdoms. Their main purpose was
	functional aids to navigation first, propaganda pieces second, if at
	all.

		\item[$\square$] I could test the colonial bias by comparing
			maps published in Paris (most French maps) and see if
			they are more accurate in the areas of French colonial
			expansion than elsewhere, and likewise for maps
			published in London. The British maps would be the best
			test, as we have more of them, and they colonised more
			coast (where we have measures of accuracy).

	\item[$\square$] Compare old maps to new. Do they agree on the core
		areas? How often do old maps place a HSE entirely outside the
		area given by a historical atlas.

	\item[$\square$] Benefits of the Geo-ISD. More states. Not tied to
		currently politically relevant ethnic groups. Does not include
		colonial rule as part of state experience or presence (as does
		State Antiquities Index). Provides a fine grained measure (I am
		considering using a finer mesh than the PRIO-GRID) of historical
		state presence that is not bound by current administrative
		boundaries (I think the closest is region versions of State
		Antiquities Index). Not only measure of historical state
		presence, but also of overlaps, borderlands and data on every
		state that make up these measures (through the tie to the isd).

	\item[$\square$] Drawbacks. Only Africa. Does not include all the states
		of the ISD. Old maps are skewed toward finding states along the
		coasts and navigable rivers. Historical atlas maps tend to draw
		borders as overly static. Has not been tied to ethnic groups,
		past or present.

	\item[$\square$] Alternative measures. States Antiquity Index.
		\citet{Murdock1967} map.
\end{itemize}

\subsection{Analysis | Deadline: Start running models by late March}
\begin{itemize}
	\item[$\square$] Cross section to avoid post-treatment bias.
	\item[$\square$] Alternative models.
\end{itemize}

\section{Communal violence}
\begin{itemize}
	\item[$\boxtimes$] Talk to Ole Magnus.

	\item[$\square$] Try to piece together some preliminary data for him to work on while I
	am on parental leave.

	\item[$\square$] Afrobarometer data will not come in time. 

	\item[$\square$] Find and include as many relevant (control) variables as possible.

\end{itemize}

\section{Paper on civil resistance}

I am hoping that this idea will mature a bit while I am on leave. I will try to
keep it in the back of my mind.

\section{First paper}

\begin{itemize}
	\item[$\square$] Get back from review.
\end{itemize}

\pagebreak

\bibliographystyle{agsm}
\bibliography{../lib.bib}{}

\section{Appendix}

\end{document}

