% {{{ Header
\documentclass[12pt]{article}
\usepackage{graphicx}
\usepackage{caption}
\usepackage{natbib}
\usepackage{authblk}
\usepackage[utf8]{inputenc}
\usepackage{setspace}
\usepackage{rotating}
\usepackage[british]{datetime2}
\usepackage{hyperref}
\usepackage{tabularx}
\usepackage{booktabs}

\hypersetup{
colorlinks=true,
citecolor=black,
urlcolor=blue
}

\renewcommand{\harvardurl}{\textbf{URL:} \url}

\renewcommand\Affilfont{\itshape\small}

\title{Kappa (working title)}
\author[1]{Marius Swane Wishman}
\affil[1]{Department of Sociology and Political Science, NTNU}

\date{\today}

\providecommand{\keywords}[1]
{
	\small	
	\textbf{\textit{Keywords---}} #1
}

\begin{document}

\maketitle

% }}}

% {{{ Abstract

\begin{abstract}
\end{abstract}

\keywords{}

\pagebreak

%tableofcontents
%\pagebreak

\onehalfspacing

% }}}

\section{Introduction}

In North West Indonesia 1976, GAM (Gerkan Aceh Merdeka: Free Aceh Movement)
declared independence for the province of Aceh, under the leadership of Hasan di
Tiro, a descendent of the last Sultan of the Aceh region. Initially the movement
consisted of the remnants of an old religious network, with its roots in the old
Sultanate and armed struggle against the Dutch. The resulting
conflict lasted until 2005 and resulted in an estimated 3402 combat related
fatalities after 1989 \citep{Aspinall2009, Pettersson2018, Sundberg2013}.

In Ethiopia 1975, the Dirge regime tried to arrest the Sultan of Aussa. However,
anticipating the move, the Sultan's son had already sent men to neighboring
Somalia to train in guerilla warfare \citep{Shehim1985}. The Sultan evaded
arrest and launched the Afar Liberation Front (ALF) organized around the men
trained in Somalia. The heavy handed response of the Ethiopian military left
over a thousand civilian casualties (https://ucdp.uu.se/conflict/363).

In 1960, in the newly formed Republic of the Congo (Léopoldville) (current
Democratic Republic of the Congo) South Kasai declares unilaterally to have
seceded from the nascent Republic under the leadership of traditional chief
Albert Kalonji \citep{Nzongola2002}. He then preceded to have his father
declared the new Mulopwe, thus resurrecting the royal title of the Luba kingdom
(1585-1889). His father promptly abdicated handing the title to Kalonji (now
styling himself Albert Ditunga, ``homeland''). South Kasai fought for independence
for just over two years, provoking a campaign by the Congolese armed forces that
at the time was characterized by UN Secretary-General Dag Hammarskjöld as an act
of genocide \citep{Nzongola2002}.

There is no shortage of examples where previously independent states are
involved in outbreaks of organized violence. Yet, both in the media and in the
academic literature these examples are referred to as ethnic conflicts, and
surprisingly little attention has been given to their connection to past
statehood. On the other hand, there are also examples of old state institutions
working for peace, mediation and reconciliation. For example [that guy in
Burkina Faso]. [Another example of peace inducing]. The nascent academic
literature on organized violence and the legacies of past statehood reflects
these diverging sets of examples. While some, in line with the examples given
above, find a conflict inducing effect of past states \citep{Englebert2002,
Paine2019}, others argue that past experience of statehood provides experience
and institutions that are peace inducing \citep{Wig2016, Wig2018,
Depetris-Chauvin2016}. Yet, all but one of these articles conceptualize states
in terms of currently (politically relevant) ethnic groups and to what degree
these groups have connections to past states. This risk excluding states that
are not readily tied to a current politically relevant ethnic group. It further
risks discrediting experiences of statehood of groups who have lived as part of
states for hundreds of years, without being the dominant ethnic group.
Additionally, this literature has been almost exclusively limited to Africa. The
diverging conclusions in the literature could in part be a result of the paucity
of quantitative data on past statehood. The literature has been limited to using
either the Murdoch map, which codes ``jurisdictional hierarchy" of ethnic
groups, or the State Antiquities Index, which measures country level experience
of statehood (including from foreign rule). In summary, there is a need for more
and better data, in order to answer the puzzle of whether there is a positive or
negative association between state histories and organized violence.
Potentially, both statements are true, but vary according to circumstances. In
which case, what determines when and where past statehood is conflict inducing
or peace inducing?

%\textbf{What do I do to address the problem/puzzle?}

How is organized violence shaped by the underlying topography of historical
statehood? This thesis seeks to answering this overarching research question,
adding to our general understanding of organized violence. While increasing the
general understanding of key concepts is a goal in itself for any academic
discipline, this understanding will hopefully contribute to the vital goals of
conflict prevention and de-escalation, however small and indirect this
contribution may be.

The thesis addresses this research question across four individual articles and
contribute to the literature through substantial data collection[?], and novel
theory building, which breaks new ground on a so far ``under-researched" part of
the larger peace- and conflict research. The thesis has contributed to two data
projects. The Anatomy of Resistance Campaigns (ARC) and the Geo-International
Systems Data (Geo-ISD). The ARC project collected yearly data on 1,426 organizations
engaging in maximalist dissent (non-violent and violent) in Africa from 1990 to
2015. The Geo-ISD geocodes the borders of independent states in Africa from
1800 to 1900, which are used to generate a measure of their respective
historical presence per 0.5 X 0.5 degrees grid cell. 

When historical state legacies are located far from the capital, they provide
\textit{symbols of sovereignty} that can be use to mobilize for violence, and
\textit{local elite networks} are often left behind and can mobilize their
networks if their interests are threatened by the government. While on the other
hand, when located near the capital, historical state legacies provide a
\textit{foundation} on which modern states could be built. Both in terms of
institutions and legitimacy.

The number of historical state legacies within a country matters because the
more of these there are, the more likely that one or more of them will be
located in a remote part of the country (and thus be conflict inducing).
Furthermore, increasing the number of potential claims making actors
incentivises the government to punish (engage in conflict) groups it would
otherwise accommodate in order to prevent other groups to make similar demands.

Whereas local elite networks might have incentives to violently oppose the
government, they also have incentives to minimize the amount of violence
between local groups and communities. Just as the state does on a country
level, and just as historical states did in their time. Therefore, historical
state legacies provide local institutions and traditions of conflict
resolution, as well as build up trust between communities, which prevent
outbreaks and escalation of communal violence. Where little or no historical
state legacies exist, and the modern state is weak or absent, groups are limited
to less effective mechanisms (such as intra group policing) to keep the peace.
In other words, when it comes to \textit{communal violence} there is an inverse
relationship between historical state legacies and organized violence.

In summary, the thesis finds that the historical legacies of statehood's
relation to organized violence varies according to:\\

1) Where in the modern state the historical state legacies are located. More
historical state legacies being conflict inducing far from modern capitals, and
conflict reducing when close.\\

2) How many distinct historical state legacies there are within a country's 
boundaries. More distinct historical state legacies leading to more conflict.\\

3) Type of organized violence. While there is more state-based violence in areas
with historical more historical state legacies, there is less communal
violence.\\

\section{Concepts} \label{Concepts}

The following section will discuss some of the key concepts of the thesis, and in
so doing elaborate on the broader academic literature and tradition on which it
builds and of which it is a part.

\subsection{Statehood and historical legacies} \label{Statehood and historical legacies}

\subsubsection{Degrees of statehood/``stateness"} \label{Degrees of statehood} % Also statebuilding


[Three conceptualisations of the state from Clapham - admin, idea and system.]

At the core of the thesis lies the concept of ``the state". However, the term is
ambiguous. Three of the four individual articles in the thesis use a specific
operationalisation, derived from \citet{Butcher2019} and \citet{Butcher2017},
but the concept merits further discussion than what the article format allows.
The subject of the thesis requires a definition that is broad enough to include
African and Asian states in the nineteenth century and modern European states.
It needs to be flexible to the changes in how we think of the state across time
(even in Europe, states were very different from today), as well as across
space. While at the same time it needs to draw the line at some point to say
what is not a state. To this end the thesis employs \citet{Clapham1996}'s three
aspect of the state, which allows assessing degrees of statehood along three
axes.

The first aspect of the state is the \textit{administrative}. The ideal of
which is an organization (government) which exercises sovereign jurisdiction
(the final legal arbiter) over a given population and territory. To exercise
this sovereignty the government controls a coercive apparatus (military and
police forces), which is usually financed by taxing the population. In return,
modern states are usually expected to ensure the welfare of its
citizens/population (externalities, health, security, education etc.). A state
in this sense may be more or less able to control its population, and more or
less able/willing to provide welfare.

The second aspect of the state is the ``idea of the state'', as constructed in
the minds of at least those who run it, but usually also a portion of the
population living within a state. This construction provides legitimacy for its
institutions and its use of coercive force (governmental legitimacy), and for
who, or where it should rule (territorial legitimacy). Today most states draw
their governmental legitimacy, their right to rule, from claiming (more or less
truthfully) to rule on behalf its citizens through democratic
principles.\footnote{Even the most blatant autocracies make this claim. [cite
Fukuyama?]} Historically, various forms of religious justification has been the
norm (divine right of kings in Europe, the mandate of Heaven in China, or rulers
claiming to be gods or dependents of gods themselves [what's the legitimation
for caliphs and amirs?]). Claims to territorial legitimacy (or lack thereof)
usually rests on a mix of historical precedence and the principle of national
self determination. Past claims include, rights to inheritance, religiously
based rights to world conquest, or the infamous ``white mans burden''. The
``idea of the state" and legitimacy is key to ensuring compliance with minimal
use (or threat) of coercion. [Reference Buzan and others in this section]

The third aspect of the state is the system of international recognition. States
recognising each other, and respecting (or even protecting) each others
sovereign territories. In the current globalized world, international
recognition has become essential to participate in international transactions.
States that are lacking in the first and second aspects of statehood can lean
more on the international system, through aid (both from other states but also
non-state actors) and ideology. Prior to the twentieth century, multiple
international state systems existed. Even as late as the nineteenth century what
mattered to most Muslim rulers was recognition by the Caliph in Istanbul, not
what the kings or queens of Europe considered had to say on the matter.
Similarly in East Asia China (the Middle Kingdom) was at the center of its
tributary-based international system, while South East Asia was organized in the
Mandala state system [Northedge and others?].

States can conform to each of these three aspects to a greater or lesser extent.
In other words, states have an overall degree of statehood, but also a
qualitative variation in terms of the individual aspects. [Poor performance] in
one aspect can be compensated, but only in part by strong performance in other.
Taiwan for example, has a robust and well functioning state apparatus, and is de
facto in undisputed control of its territory, enjoys a high degree of legitimacy
and [compliance] from its citizens, but struggles with a lack of full
international recognition. Israel has a [strong] administrative statehood, and
enjoys recognition from the most relevant actors (the exception being several
Muslim majority countries), but is viewed as largely illegitimate among many of
its Palestinian population [fair assumption?] (roughly 20\% of its population).
Somalia (and other so-called ``failed states") [score] poorly across all three
aspects of statehood. The Somali government barely functions in and around the
capital, let alone the rest of its [purported] territory. Its government is
viewed as corrupt and illegitimate. Its borders do not reflect the settlement of
the Somali ethnic group, lacks any historic president, and are the product of
exogenous factors (external diplomatic negotiations). What little claim to
statehood Somalia has rests almost exclusively on the international system.
[Link to artificiality here?]

[Relevance of state creation/formation/building. Contrasting the ``natural"
process with the artificial? Or is that going too deep?] 

Because this thesis is concerned with differences between state composition
relative to the local histories of statehood, I need to discuss how these
compositions can come about. I separate between endogenous forms of state
formation and exogenous forms of state formation. Relative to the three aspects
of state endogenous state formation relies on administrative state capacity and
the idea of the state, while exogenous forms of state formation relies almost
exclusively on the international system of states.

\citet{scott2017against} Against the grain - state formation - a look at how
states first emerged

Vast majority were non-state until at least 1600. -Frontiers Most of the history
of the state is about its interactions (trading and raiding) with the
``barbarians" beyond its frontiers. Stable, lasting states are rare outside, and
prior to, the Westphalian state system (Egypt, China and Rome are notable
exceptions). The default is a perilous existence, where states rise, thrive and
fall in rapid succession. Ending at the hand of other states, barbarian
invasions, population exodus or any combination of the above. The arrival of
gunpowder is the game changer. Led to a period of state ``building" or stating in
Africa (and elsewhere?) prior to the arrival of European colonialists [Source on
this?]. For many states, or parts of the world, this was still the reality in
the nineteenth century. As described by	\citet{Scott2009}. States ruling an
agrarian core surrounded by a large, permeable frontier. State penetration into
the frontier was in the from of relations with groups, ranging from tributary,
through allied or hostile to extracting ``protection" payments from the state.
The point of this discussion [which should be made clear earlier, or perhaps
means that this belongs elsewhere] is that drawing a line on a map between what
is and is not part of a state, or what areas belong to what state, prior to the
globalisation of the Wesphalian model allowed for a great deal of variation in
the frontier zone, based on what ones conceptualisation of statehood. A more
accurate representation would be shade/gradient of statehood that fades into the
frontier, for most of Africa and Asia in the nineteenth century. Boundaries
between states, when they occurred, would usually be in the frontiers of each
state, where neither would have full control.

The extent of states then, would vary according to their ability to project
military power outside their alluvial/grain producing core(s). This, in turn
rested on their administrative and ``idea of state" aspects of statehood, as the
limited and often local systems of statehood had limited influence.
\citet{Clapham1996}'s argument is that post-independence Africa represents a new
model of statehood, where statehood rested almost exclusively on the
international system. They were in large part created by (large and important
wars of liberation/de-colonization notwithstanding) and eventually sustained by
the international system. At the very least their \textit{extent} was. This
amounted to a novel form of limited, or artificial statehood. I argue that this
process is not unique to Africa. In Asia too, Europeans created colonial states,
and left countries with limited degrees of administrative statehood and vague
``ideas of state".

[Empirical implications of artificial states -- ethnic groups]

This limited statehood has been linked to a number of empirical outcomes. For
example, \citet{Alesina2011} uses ``squigglyness" of international boundaries as
a measure/instrument for whether the boundaries were drawn with local knowledge
(endogenously), or not (exogenously). They find that straightness of
international boundaries, what they term artificial borders, are linked with
lower levels of economic development. The presumed mechanism (which is not
tested) is that artificial borders, borders drawn by an exogenous process, group
together multiple ethnic groups and split others. These ethnic constellations
then make it difficult for the state to create a sense of nationhood and get
people to work toward common goals. \citet{Englebert2002} tests this more
explicitly and find that states whose borders split ethnic groups more, are more
often involved in international disputes. They also find that countries whose
boundaries group together ethnic groups with more different forms pre-colonial
political organization\footnote{As measured by the standard deviation of
\citet{Murdock1967}'s jurisdictional hierarchy index.}, are more susceptible to
civil wars, political instability and secession attempts. Following
\citet{Englebert2002} the conflict-literature has also uncovered links between
ethnic partitioning and civil conflict \citep{Ito2020}, as well as links between
resulting trans-border ethnic kin relations and conflict \citep{Cederman2013,
Salehyan2009, Weidmann2015}. [Tie back to (limited) degrees of statehood]

\subsubsection{Legacies of statehood past} \label{Legacies} 

Legacies of states, institutions and conflict

How states are composed today clearly matter for economic development, levels of
violence and a host of other things, but a major/pivotal/? argument of this
thesis, and a emerging literature, is that \textit{past} states matter as well.
On every continent there are hundreds of states, than until relatively recently
were independent and sovereign entities, much like the those who survived until
today. Many, if not most, of them were amalgamated into larger states through
the process of colonisation\footnote{The two major exceptions are the
unification of Germany and Italy.}. There is a growing literature demonstrating
how these past states and institutions still have lasting legacies today. Most
of this literature has examined the effects of past statehood
\citep{Bockstette2002, Borcan2018} and institutions related to statehood
\citep{Michalopoulos2013, Michalopoulos2018, Englebert2000} and largely agree on
a positive effect of statehood and institutions \citep{Nunn_2020,
Michalopoulos2016}, although there are notable exceptions \citep{Acemoglu_2002}.

A smaller/different branch of research has examined political implications of
state legacies. \citet{Acemoglu_2002} argues that European colonialism lead to a
``reversal  of fortunes" for the areas it affected. Poorer areas with less
population density (areas less likely to be at the center of states), were less
likely to be colonised early, ``deeply", and with European settlers (in part
because of states more effectively resisting colonization). This lead to a
larger transfer of European institutional innovations, and consequently a
reversal of economic development \citep{Acemoglu_2002}. Building on this
research, \citet{Hariri2012} test parts of the proposed mechanism and finds that
more pre-colonial experience of statehood decreased the likelihood of being
colonised, while it increased the likelihood of only being \textit{indirectly}
colonised. Furthermore, not being colonised, or being so indirectly, depresses
post-colonial democracy \citep{Hariri2012}. On the other hand, through a number
of case studies Wilfahrt \citeyear{Wilfahrt2018, Wilfahrt_2021} has documented
how areas and groups with pre-colonial experiences of statehood are better at
distributing public goods equitably and efficiently, and are generally more
likely to adopt [non-partisan (non-group based)] legislation and politics. The
actors themselves ascribe this to their history of cooperation (under the
umbrella of pre-colonial states). This is even true in Senegal, the poster child
of French direct rule and dismantling of pre-existing institutions
\citep{Wilfahrt_2021}.

Living up to the second part of the quote attributed to Charles Tilly, ``War
made the state, and the state made war.", past states also left legacies of
conflict. Historical levels conflict in Africa has been found to positively
affect modern levels of conflict \citep{Besley2014}, although the direction of
the relationship is does not hold in a global sample, and depends on colonial
experiences and wars of liberation (extra systemic conflicts)
\citep{Fearon2014}. 

\subsubsection{Historical state entities and Pre-colonial states} 
\label{Historical state entities and Pre-colonial states}

At this point a further clarification of the terms ``past states",
``pre-colonial states", and ``historical state entities" (the term used in the
second article of the thesis) is needed, as these terms are used more or less
interchangeably in this [introductory chapter/narrative]. The reason they are
used interchangeably is that they do refer to the same \textit{concept}.
However, they [apply to/are used in] different \textit{contexts}. Specifically,
I use ``pre-colonial states" in the context of Africa, where the majority of
states were at some point colonies. The other two terms are used in the global
context and refers to states that now form parts of sovereign states. The latter
refers is further restricted to the time period of the second article of the
thesis (1816-1939).

\subsection{Collective dissent and organized violence} \label{Collective dissent and organized violence}

\subsubsection{Maximalist dissent} \label{Maximalist dissent}

Chenoweth and Stephan and ARC definitions. A few of the greats of the
non-violent dissent literature.

\subsubsection{Civil Conflict/violence OR Intrastate conflict/violence} \label{Civil Conflict}

Motivations - Gurr, why men rebel - Wood, insurgent collective violence

Definitions

Types:

State based \\

Non-state \\ 

Communal violence \\

\section{Theoretical/conceptual framework} \label{Theoretical/conceptual framework}

Looking for the connection(s) between the two overarching concepts. What links
has been found? What has not been done? What is this thesis adding to this
literature? 

Institutionalist/longe durée? Acemoglu and Robinson - persisting institutions
and stickiness.

Main "thrust" is that ``it depends''. Expand on the answer to the puzzle in the
introduction. It depends on the relationship between the modern state and old
state(s), and on the type of violence.

Some more prior literature: \citet{Griffiths2016} \citet{Ahram2019}

\section{Analytical approach} \label{Analytical approach}

Data narrative (inductive/deductive). Empirical tradition.

Discussion of maps to uncover the state?

Discussion of ``state presence"?

\section{Article summaries} \label{Article summaries}

\section{Concluding remarks} \label{Concluding remarks}

% {{{ Footer
\pagebreak

\bibliographystyle{agsm}
\bibliography{../lib.bib}

\end{document}
% }}}
