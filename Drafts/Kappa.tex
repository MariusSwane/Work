% {{{ Header
\documentclass[12pt]{article}
\usepackage{graphicx}
\usepackage{caption}
\usepackage{natbib}
\usepackage{authblk}
\usepackage[utf8]{inputenc}
\usepackage{setspace}
\usepackage{rotating}
\usepackage[british]{datetime2}
\usepackage{hyperref}
\usepackage{tabularx}
\usepackage{booktabs}

\hypersetup{
colorlinks=true,
citecolor=black,
urlcolor=blue
}

\renewcommand{\harvardurl}{\textbf{URL:} \url}

\renewcommand\Affilfont{\itshape\small}

\title{Kappa (working title)}
\author[1]{Marius Swane Wishman}
\affil[1]{Department of Sociology and Political Science, NTNU}

\date{\today}

\providecommand{\keywords}[1]
{
	\small	
	\textbf{\textit{Keywords---}} #1
}

\begin{document}

\maketitle

% }}}

% {{{ Abstract

\begin{abstract}
\end{abstract}

\keywords{}

\pagebreak

%tableofcontents
%\pagebreak

\onehalfspacing

% }}}

\section{Introduction}

In North West Indonesia 1976, GAM (Gerkan Aceh Merdeka: Free Aceh Movement)
declared independence for the province of Aceh, under the leadership of Hasan di
Tiro, a descendent of the last Sultan of the Aceh region. Initially the movement
consisted of the remnants of an old religious network, with its roots in the old
Sultanate and armed struggle against the Dutch. The resulting
conflict lasted until 2005 and resulted in an estimated 3402 combat related
fatalities after 1989 \citep{Aspinall2009, Pettersson2018, Sundberg2013}.

In Ethiopia 1975, the Dirge regime tried to arrest the Sultan of Aussa. However,
anticipating the move, the Sultan's son had already sent men to neighboring
Somalia to train in guerilla warfare \citep{Shehim1985}. The Sultan evaded
arrest and launched the Afar Liberation Front (ALF) organized around the men
trained in Somalia. The heavy handed response of the Ethiopian military left
over a thousand civilian casualties (https://ucdp.uu.se/conflict/363).

In 1960, in the newly formed Republic of the Congo (Léopoldville) (current
Democratic Republic of the Congo) South Kasai declares unilaterally to have
seceded from the nascent Republic under the leadership of traditional chief
Albert Kalonji \citep{Nzongola2002}. He then preceded to have his father
declared the new Mulopwe, thus resurrecting the royal title of the Luba kingdom
(1585-1889). His father promptly abdicated handing the title to Kalonji (now
styling himself Albert Ditunga, `homeland'). South Kasai fought for independence
for just over two years, provoking a campaign by the Congolese armed forces that
at the time was characterized by UN Secretary-General Dag Hammarskjöld as an act
of genocide \citep{Nzongola2002}.

What the three preceding narratives have in common is that in both the media
and the academic peace-and-conflict literature, they have been cast almost
exclusively as ethnic conflicts. While they all have important ethnic components
as well, they clearly demonstrate a link to past states.

\textbf{Why is this problem worth addressing?} \\
How does it relate to other work? Puzzles etc in the literature.

\textbf{What do I do to address the problem?}
Overarching research questions? Aims? How is contentious politics/political
violence/organized violence shaped by topography of statehood? 

\section{Concepts} \label{Concepts}

\subsection{Statehood and historical legacies} \label{Statehood and historical legacies}

\subsubsection{Historical state entities and Pre-colonial states} \label{Historical state entities and Pre-colonial states}

\subsubsection{Artificial states} \label{Artificial states}

\subsection{Collective dissent and organized violence} \label{Collective dissent and organized violence}

\subsubsection{Maximalist dissent} \label{Maximalist dissent}

\subsubsection{Civil Conflict} \label{Civil Conflict}

State based \\

Non-state \\ 

Communal violence \\

\section{Theoretical/conceptual framework} \label{Theoretical/conceptual framework}

Looking for the connection(s) between the two overarching concepts. What links
has been found? What has not been done? What is this thesis adding to this
literature?

\section{Analytical approach} \label{Analytical approach}

Data narrative (inductive/deductive). Empirical tradition.

Discussion of maps to uncover the state?

Discussion of ``state presence"?

% {{{ Footer
\pagebreak

\bibliographystyle{agsm}
\bibliography{../lib.bib}

\end{document}
% }}}
