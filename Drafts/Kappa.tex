% {{{ Header
\documentclass[12pt]{article}
\usepackage{graphicx}
\usepackage{caption}
\usepackage{natbib}
\usepackage{authblk}
\usepackage[utf8]{inputenc}
\usepackage{setspace}
\usepackage{rotating}
\usepackage[british]{datetime2}
\usepackage{hyperref}
\usepackage{tabularx}
\usepackage{booktabs}

\hypersetup{
colorlinks=true,
citecolor=black,
urlcolor=blue
}

\renewcommand{\harvardurl}{\textbf{URL:} \url}

\renewcommand\Affilfont{\itshape\small}

\title{Kappa (working title)}
\author[1]{Marius Swane Wishman}
\affil[1]{Department of Sociology and Political Science, NTNU}

\date{\today}

\providecommand{\keywords}[1]
{
	\small	
	\textbf{\textit{Keywords---}} #1
}

\begin{document}

\maketitle

% }}}

% {{{ Abstract

\begin{abstract}
\end{abstract}

\keywords{}

\pagebreak

%tableofcontents
%\pagebreak

\onehalfspacing

% }}}

\section{Introduction}

In North West Indonesia 1976, the ASNLF (Aceh-Sumatra National Liberation
Front), predecessor of GAM (Gerkan Aceh Merdeka: Free Aceh Movement) declared
independence for the province of Aceh, under the leadership of Hasan di Tiro,
a descendent of the last Sultan of the Aceh region. Initially the movement
consisted of the remnants of an old religious network with its roots in the
old Sultanate and armed struggle against the Dutch colonizers. The resulting
conflict lasted until 2005 and resulted in an estimated 3402 combat related
fatalities \citep{Aspinall2009, Pettersson2018, Sundberg2013}.

In Ethiopia 1975, the Dirge regime tries to arrest the Sultan of Aussa. However,
anticipating the move, the Sultan's son has sent men to neighboring Somalia in
advance to train in guerilla warfare \citep{Shehim1985}. The heavy handed
response of the Ethiopian military left over a thousand civilian casualties
(https://ucdp.uu.se/conflict/363).

In 1960, in the newly formed Republic of the Congo (Léopoldville) (current Democratic
Republic of the Congo) South Kasai declares unilaterally to have seceded from
the nascent Republic under the leadership of traditional chief Albert Kalonji.
He then preceded to have his father declared the new Mulopwe, thus resurrecting
the royal title of the Luba kingdom (1585-1889). His father promptly abdicated
handing the title to Kalonji (now styling himself XXXX, "homeland"). South Kasai
fought for independence for just over two years, provoking a campaign by the
Congolese armed forces that at the time was characterized by UN
Secretary-General Dag Hammarskjöld as an act of genocide.

What all three preceding narratives have in common is that in both the media
and the academic peace and conflict literature, they have been cast almost
exclusively as ethnic conflicts. While they all have important ethnic components
as well, they clearly demonstrate a link to past states.

% {{{ Footer
\pagebreak

\bibliographystyle{agsm}
\bibliography{../lib.bib}

\section{Appendix}

\end{document}
% }}}
