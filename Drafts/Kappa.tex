% {{{ Header
\documentclass[12pt]{article}
\usepackage{graphicx}
\usepackage{caption}
\usepackage{natbib}
\usepackage{authblk}
\usepackage[utf8]{inputenc}
\usepackage{setspace}
\usepackage{rotating}
\usepackage[british]{datetime2}
\usepackage{hyperref}
\usepackage{tabularx}
\usepackage{booktabs}

\hypersetup{
colorlinks=true,
citecolor=black,
urlcolor=blue
}

\renewcommand{\harvardurl}{\textbf{URL:} \url}

\renewcommand\Affilfont{\itshape\small}

\title{Kappa (working title)}
\author[1]{Marius Swane Wishman}
\affil[1]{Department of Sociology and Political Science, NTNU}

\date{\today}

\providecommand{\keywords}[1]
{
	\small	
	\textbf{\textit{Keywords---}} #1
}

\begin{document}

\maketitle

% }}}

% {{{ Abstract

\begin{abstract}
\end{abstract}

\keywords{}

\pagebreak

%tableofcontents
%\pagebreak

\onehalfspacing

% }}}

\section{Introduction}

In North West Indonesia 1976, GAM (Gerkan Aceh Merdeka: Free Aceh Movement)
declared independence for the province of Aceh, under the leadership of Hasan di
Tiro, a descendent of the last Sultan of the Aceh region. Initially the movement
consisted of the remnants of an old religious network, with its roots in the old
Sultanate and armed struggle against the Dutch. The resulting
conflict lasted until 2005 and resulted in an estimated 3402 combat related
fatalities after 1989 \citep{Aspinall2009, Pettersson2018, Sundberg2013}.

In Ethiopia 1975, the Dirge regime tried to arrest the Sultan of Aussa. However,
anticipating the move, the Sultan's son had already sent men to neighboring
Somalia to train in guerilla warfare \citep{Shehim1985}. The Sultan evaded
arrest and launched the Afar Liberation Front (ALF) organized around the men
trained in Somalia. The heavy handed response of the Ethiopian military left
over a thousand civilian casualties (https://ucdp.uu.se/conflict/363).

In 1960, in the newly formed Republic of the Congo (Léopoldville) (current
Democratic Republic of the Congo) South Kasai declares unilaterally to have
seceded from the nascent Republic under the leadership of traditional chief
Albert Kalonji \citep{Nzongola2002}. He then preceded to have his father
declared the new Mulopwe, thus resurrecting the royal title of the Luba kingdom
(1585-1889). His father promptly abdicated handing the title to Kalonji (now
styling himself Albert Ditunga, `homeland'). South Kasai fought for independence
for just over two years, provoking a campaign by the Congolese armed forces that
at the time was characterized by UN Secretary-General Dag Hammarskjöld as an act
of genocide \citep{Nzongola2002}.

There is no shortage of examples where previously independent states are
involved in outbreaks of organized violence. Yet, both in the media and in the
academic literature these examples are referred to as ethnic conflicts, and
surprisingly little attention has been given to their connection to past
statehood. On the other hand, there are also examples of old state institutions
working for peace, mediation and reconciliation. For example [that guy in
Burkina Faso]. [Another example of peace inducing]. The nascent academic
literature on organized violence and the legacies of past statehood reflects
these diverging sets of examples. While some, in line with the examples given
above, find a conflict inducing effect of past states \citep{Englebert2002,
Paine2019}, others argue that past experience of statehood provides experience
and institutions that are peace inducing \citep{Wig2016, Wig2018,
Depetris-Chauvin2016}. Yet, all but one of these articles conceptualize states
in terms of currently (politically relevant) ethnic groups and to what degree
these groups have connections to past states. This risk excluding states that
are not readily tied to a current politically relevant ethnic group. It further
risks discrediting experiences of statehood of groups who have lived as part of
states for hundreds of years, without being the dominant ethnic group.
Additionally, this literature has been almost exclusively limited to Africa. The
diverging conclusions in the literature could in part be a result of the paucity
of quantitative data on past statehood. The literature has been limited to using
either the Murdoch map, which codes ``jurisdictional hierarchy" of ethnic
groups, or the State Antiquities Index, which measures country level experience
of statehood (including from foreign rule). In summary, there is a need for more
and better data, in order to answer the puzzle of whether there is a positive or
negative association between state histories and organized violence.
Potentially, both statements are true, but vary according to circumstances. In
which case, what determines when and where past statehood is conflict inducing
or peace inducing?

%\textbf{What do I do to address the problem/puzzle?}

How is organized violence shaped by the underlying topography of historical
statehood? This thesis seeks to answering this overarching research question,
adding to our general understanding of organized violence. While increasing the
general understanding of key concepts is a goal in itself for any academic
discipline, this understanding will hopefully contribute to the vital goals of
conflict prevention and de-escalation, however small and indirect this
contribution may be.

The thesis addresses this research question across four individual articles and
contribute to the literature through substantial data collection[?], and novel
theory building, which breaks new ground on a so far ``under-researched" part of
the larger peace- and conflict research. The thesis has contributed to two data
projects. The Anatomy of Resistance Campaigns (ARC) and the Geo-International
Systems Data (Geo-ISD). The ARC project collected yearly data on 1,426 organizations
engaging in maximalist dissent (non-violent and violent) in Africa from 1990 to
2015. The Geo-ISD geocodes the borders of independent states in Africa from
1800 to 1900, which are used to generate a measure of their respective
historical presence per 0.5 X 0.5 degrees grid cell. 

When historical state legacies are located far from the capital, they provide
\textit{symbols of sovereignty} that can be use to mobilize for violence, and
\textit{local elite networks} are often left behind and can mobilize their
networks if their interests are threatened by the government. While on the other
hand, when located near the capital, historical state legacies provide a
\textit{foundation} on which modern states could be built. Both in terms of
institutions and legitimacy.

The number of historical state legacies within a country matters because the
more of these there are, the more likely that one or more of them will be
located in a remote part of the country (and thus be conflict inducing).
Furthermore, increasing the number of potential claims making actors
incentivises the government to punish (engage in conflict) groups it would
otherwise accommodate in order to prevent other groups to make similar demands.

Whereas local elite networks might have incentives to violently oppose the
government, they also have incentives to minimize the amount of violence
between local groups and communities. Just as the state does on a country
level, and just as historical states did in their time. Therefore, historical
state legacies provide local institutions and traditions of conflict
resolution, as well as build up trust between communities, which prevent
outbreaks and escalation of communal violence. Where little or no historical
state legacies exist, and the modern state is weak or absent, groups are limited
to less effective mechanisms (such as intra group policing) to keep the peace.
In other words, when it comes to \textit{communal violence} there is an inverse
relationship between historical state legacies and organized violence.

In summary, the thesis finds that the historical legacies of statehood's
relation to organized violence varies according to:\\

1) Where in the modern state the historical state legacies are located. More
historical state legacies being conflict inducing far from modern capitals, and
conflict reducing when close.\\

2) How many distinct historical state legacies there are within a country's 
boundaries. More distinct historical state legacies leading to more conflict.\\

3) Type of organized violence. While there is more state-based violence in areas
with historical more historical state legacies, there is less communal
violence.\\

\section{Concepts} \label{Concepts}

\subsection{Statehood and historical legacies} \label{Statehood and historical legacies}

\subsubsection{Historical state entities and Pre-colonial states} \label{Historical state entities and Pre-colonial states}

\subsubsection{Artificial states} \label{Artificial states}

\subsection{Collective dissent and organized violence} \label{Collective dissent and organized violence}

\subsubsection{Maximalist dissent} \label{Maximalist dissent}

\subsubsection{Civil Conflict} \label{Civil Conflict}

State based \\

Non-state \\ 

Communal violence \\

\section{Theoretical/conceptual framework} \label{Theoretical/conceptual framework}

Looking for the connection(s) between the two overarching concepts. What links
has been found? What has not been done? What is this thesis adding to this
literature?

\section{Analytical approach} \label{Analytical approach}

Data narrative (inductive/deductive). Empirical tradition.

Discussion of maps to uncover the state?

Discussion of ``state presence"?

\section{Article summaries} \label{Article summaries}

\section{Concluding remarks} \label{Concluding remarks}

% {{{ Footer
\pagebreak

\bibliographystyle{agsm}
\bibliography{../lib.bib}

\end{document}
% }}}
