%{{{ Header
\documentclass[12pt]{article}

\usepackage{graphicx}
\usepackage{caption}
\usepackage[round]{natbib}
\usepackage{authblk}
\usepackage[utf8]{inputenc}
\usepackage{setspace}
\usepackage{rotating}
\usepackage[british]{datetime2}
\usepackage{hyperref}
\usepackage{tabularx}
\usepackage{booktabs}
\usepackage{multicol}
\usepackage{dcolumn}
\usepackage[top=28truemm,bottom=26.5truemm,left=25.5truemm,right=25.5truemm]{geometry}

\hypersetup{
colorlinks=true,
citecolor=black,
urlcolor=blue
}

\renewcommand{\harvardurl}{\textbf{URL:} \url}

\renewcommand\Affilfont{\itshape\small}

\title{After forever: Pre-colonial states and civil conflict}
\author[1]{Marius Swane Wishman}
\affil[1]{Department of Sociology and Political Science, NTNU}

\date{\today}

\providecommand{\keywords}[1]
{
	\small	
	\textbf{\textit{Keywords---}} #1
}

\begin{document}

\maketitle
%}}}

%{{{ Abstract
\begin{abstract}

This paper examines the relationship between the degree of pre-colonial state
presence and post cold war civil conflict, and finds that the level of
pre-colonial state presence is positively correlated with civil conflict
fatalities. Histories of statehood leave behind powerful symbols of independence
useful for mobilization, and regional elite networks (formal or informal) with
the potential to violently resist state expansion into their sphere of
influence. However, higher levels of pre-colonial state presence are conflict
reducing in areas surrounding modern capital cities. I argue this is due to
greater continuity of traditions and institutions associated with statehood that
are inherently conflict reducing. Additionally, the paper introduces the Geo-ISD
data set, which dynamically maps the borders of 82 independent states in Africa
in the 1800-1914 period. 

\end{abstract}

\keywords{Conflict, civil war, pre-colonial states, GIS}

%tableofcontents
%\pagebreak

\onehalfspacing

%\begin{multicols}{2}

\newpage
%}}}

\section{Introduction}

Arguably, the political history of Africa in the nineteenth century has thus far
been largely overshadowed by the colonial empires of Europe, and the so called
Scramble for Africa. What tends to be overlooked is that for the most part what
the Europeans conquered were pre-existing states\footnote{An extended discussion
of what kind of statehood existed in pre-colonial Africa will follow below.},
with their own armies, dynasties and sometimes hundreds of years of history.
Likewise, numerous studies have examined how colonial experiences have shaped
post independence level of conflict \citep{achankeng2013conflict, Blanton_2001,
carton2000blood, Cohen_2014, Nunn2008, Wucherpfennig2016}, yet only a handful of
recent studies have done the same for independent pre-colonial African polities,
despite their simultaneous or immediately preceding existence. Within this
emerging literature, the effect of pre-colonial states on civil conflict appears
contradictory. While some find a conflict inducing effect due to differences
between ethnic groups with and without histories of statehood
\citep{Englebert2002, Paine2019}, others find a conflict reducing effect from
institutions inherited from these states \citep{Depetris-Chauvin2016, Wig2016}.

This paper addresses the puzzle by arguing that the effect of pre-colonial
states is determined by whether or not the given pre-colonial state forms the
basis of the post-independence state. In many countries in Africa there is at
least some continuity of rule and institutions running from a pre-colonial
state, through colonial administration to post-colonial state. In these cases
the state is able to draw on old institutions and governance capacity. On the
hand, there are areas with long histories of independent statehood in the
pre-colonial era who find themselves ruled from far away capitals, whose
populations prior to the colonial era had little to no contact. In these cases,
local elite networks with the capacity to mobilize for conflict often find their
interests at odds with the central state's.

The second contribution of this paper, is the introduction of new and innovative
data on pre-colonial statehood in Africa. The data improves on existing sources
on a number of points. It contains far more states than comparative data sets,
without compromising on the criteria for statehood. Crucially, it adds geocoding
of individual pre-colonial states, as opposed to aggregations to current
administrative levels or tied to settlement patterns of related ethnic groups.
Nor does the data take (currently politically relevant) ethnic groups as a
starting point for inclusion. % Why is this a good thing? Lastly, it leverages
variations in contemporaneous sources to provide a measure of fuzzy borders and
gradually dissipating state presence outside core areas.

\section{Literature review} 

Despite the emerging literature on the outcomes of traditional and pre-colonial
states and institutions, the nature of the relationship between such historical
roots and conflict remains disputed. From a game theoretical perspective like
that of \citet{Fearon1995}, one would expect pre-colonial institutions to be
conflict reducing. Groups who interact with the (modern) state through
(traditional or otherwise) institutions reduce the uncertainty of future
behaviour relative to groups who bargain through individuals, who are inherently
more unpredictable and more prone to spoilers \citep{Wig2016}. Additionally,
institutions are able to make credible commitments by putting restraints on
their leaders through imposing violation costs. If a leader reneges on a
commitment ratified by an institution, it reduces the legitimacy of the
institutions and thus other laws passed by it. Pre-colonial or traditional
institutions could also be conflict reducing by improving local state capacity,
which could have a direct effect on the states ability to impose and preserve
order as well as an indirect effect through economic development
\citep{Depetris-Chauvin2016}. In support of this latter argument
\citet{Depetris-Chauvin2016} finds that areas with more `state history' have
higher levels of trust in various political leaders.

On the other hand, a more empirical branch of the literature has argued for 
potential conflict inducing effects of pre-colonial states. Particularly when
considering how groups with ties to pre-colonial states interact not just
vis-a-vis the modern state, but also with other groups.

Colonial boundaries that bundled together multiple ethnic groups with different
historical experiences of political organization, lead to a `suffocation' effect
\citep{Englebert2002}. In such an environment post-independent states found it
difficult to create a sense of nationality, cohesion or solidarity among its
population, leading to higher levels of conflict \citep{Englebert2002}. This
argument ties in to the larger literature on `artificial states', which argues
that many states, in Africa in particular, are artificial in the sense that
their boundaries do not reflect the underlying topography of statehood
\citep{Alesina2011}. This artificiality has been linked to lower levels of
economic development, presumably working in part through increased ethnic
tensions and conflict \citep{Alesina2011}. In this view, both having no
pre-colonial states, as well as having more than one, could be considered
artificial.\footnote{At least when these states were incorporated into the
current boundaries by external force (such as colonizers), as opposed to
`indigenously' (as for example the 100+ states of Germany being unified by
Prussia).} 

Having multiple groups with similar claims to pre-colonial independence also
changes the game theoretical calculus for the state. Choosing to accommodate one
claims-making group in such an environment could lead to further claims by
similar groups, making the option of punishing any group who makes demands
relatively cheaper, and thus making conflict a more likely outcome
\citep{Wishman}.

In countries where an ethnic group has a history of statecraft, said group is
likely to be have an over sized share of power in government
\citep{Wucherpfennig2016}. This can come about through indirect colonial rule,
which preferred to leaving existing power structures intact, or by seizure from
less politically experienced groups following independence \citep{Paine2019}.
When faced with a trade off between including strong rivals\footnote{`Strong
	rivals' refer to rival (ethnic) groups who would be capable of punishing
	the ruling group for exclusion.} in government and risking coups, and
	excluding them and risking civil conflict, rulers generally avoid risk
	of coup \citep{Paine2019, Powell_2014, Roessler_2011}.\footnote{Note
		that in some cases the ruler is forced to include the rival
		group, for example in cases of split dominion, when the colonial
		power split the responsibility of the civil and military
		administration between different ethnic groups
		\citep{Paine2019}.} Given a pre-colonial history that often
		included unequal and violent relations between state and
		non-state groups\footnote{See for example \citet{Nunn2008} on
		the lasting impact of slave trade.}, and how colonial rule often
		accentuated these cleavages, \citet{Paine2019} argues that
		ruling pre-colonial state groups typically faced commitment
		problems, which lead them into the trade-off between risk of
		coup or civil conflict \citep{Paine2019}. This also ties in with
		the substantial literature on the conflict inducing impact of
		horizontal inequalities \citep{CEDERMAN_2011}.

Apart from institutions, pre-colonial states potentially leave behind symbols of
sovereignty and vertical elite networks \citep{Wishman}. Past independence has
become an important ingredient in most struggles for national independence, and
is used by conflict entrepreneurs to overcome collective action problems, as
well as to provide a basis for ethnic claims making by referring to past
violations of sovereignty \citep{Ahram2019, Shelef2016}. Vertical elite networks
matter for conflict because their vertical nature\footnote{The vertical
orientation of these networks stem from the vertical power structures typical of
states.} is purpose build for mobilization, and the fact that they are elite
networks means that they have expectations of being included in government, have
substantial regional autonomy, or both \citep{Wishman}. Recent work by
\citet{Ying_2020} indicates that civil conflict tends to occur when the state
increases its presence in areas that it has hitherto not been present, i.e. when
it challenges the autonomy of regional elites. For example, in Ethiopia, the
Afar Liberation Front was originally formed by the sultan of the former
pre-colonial state of Awsa, when the Dirge regime tried to depose the sultan
\citep{Shehim1985, Hanfare2011}.  In Libya the Cyraneica Liberation Army
demonstrates both the symbolic mechanism as well as the elite networks, as its
name refers to a short lived kingdom in Eastern Libya, and the group elected a
descendent of the former king as their leader \citep{Ahram2019}.

% [Include resistance to Western influences/institutions?]

\section{Pre-colonial states}

For the purpose of this paper I follow the definition of `state' used by the
International Systems Data (ISD) v2 \citep{Butcher2020} as a political entity
with a population of at least 10,000, which has autonomy over a specific
territory and sovereignty that is either uncontested or acknowledged by relevant
international actors \citep{Butcher2020}\footnote{For a more in depth discussion
of the definition of and criteria for statehood that the ISD is based on, see
\citet{Butcher2017}.}. By this definition the ISD v2 identifies 109 pre-colonial
states in Africa during the 1800-1914 period, of which 82 are included in the
data used for this paper. This is a heterogeneous group of political entities
along most metrics. In size they range from small city states like Harar (today
part of Ethiopia), to empires like the Sokoto Caliphate. In political
organization they range from loose federations (for example Oyo \citep{Law1977})
to relatively centralized kingdoms (Abyssinia/Ethiopia, Buganda, or Zulu).

While smaller states might have been relatively mono-ethnic, larger states were
often multi-ethnic, although often politically dominated by one group. While the
geographic scope of the paper is Africa, pre-colonial states do include settler
states, as long as they were independent. In other words Liberia, the Boer
Republics and eventually South Africa are included. Based on the relationship
with regional powers, some states `come and go' as sovereign entities. Examples
of this include the North African states in their relationship with the Ottoman
empire, or Zinder (Sultanate of Damagaram), a city state on the periphery of the
Bornu empire, at times nominally subject, de facto subject or fully independent
of Bornu.

While it is difficult to generalize about what this heterogeneous group of
states usually were, they were not modern states as we think of states today.
For one, there were (to my knowledge) no polices forces in the way we think of
them today, nor welfare state of any kind. Bureaucracies and state apparatus
while at times existing and relatively centralized, were rarely comprehensive in
size. Nor did any of them have international boundaries in the sense that
countries do today. 
% Describe the frontiers and borders they did have?

In other words, states were `shallow' relative to today. For most of the
population, for most of the time, the state was embodied by a local
representative (chief, bureaucrat, imam, lord etc.), often with some level of
judicial and tax responsibility and wide autonomy. Nominal subjugation and local
self rule was also wide spread. For example, the kingdom of Wadai/Bergoo (in
Chad) did not have a civil government, but had a royal council (fásher) and a
vertical network of political organization running down through regions, to
provinces to tribes and villages \citep{barth1857travels}. Tax (diván) rates
differed on the basis of prosperity of the area, individual political standing,
ethnic affiliation and religious holidays etc., but were generally more uniform
in the central provinces. In the surrounding provinces tribute was paid by the
province as a whole, reflecting the decreasing reach of the Wadai state.
Immediately outside that control, lay the neighboring kingdom of Baghirmi. Who
at least in some periods, paid tribute to Wadai while retaining its sovereignty
\citep{barth1857travels}.

Despite their relatively shallow state structures there is ample evidence that
many pre-colonial states left marks that were felt long after they were
colonized, also in Africa. For example, Pre-colonial states left behind
traditional political institutions \citep{Beall_2005, Holzinger_2020,
Neupert_Wentz_2021, Ubink_2008}. These institutions have at times acted as
mediators between ethnic groups and the central state \citep{boone2014property,
Englebert2002}, and has been an important sources of legitimacy for current
institutions \citep{Wig2016}\footnote{According to \citet{mamdani2018citizen}
	colonial authorities also felt the need to legitimize their rule through
	ties to pre-colonial institutions, at times going as far as to invent
pre-colonial roots.}. Nevertheless, some have argued that pre-colonial states
have represented competitors to the central state as well \citep{Herbst2014}.
There is also a growing literature on the role of pre-colonial states and
institutions in long term economic development \citep{Michalopoulos2018,
Acemoglu2014, Gennaioli2007, Bockstette2002}.

\section{The importance of pre-colonial roots}

I argue that how pre-colonial states affect conflict (which set of mechanisms
proposed by the literature are active), is determined by whether or not the
pre-colonial state in a given area is the one that inherited the state.
In other words, by the degree to which the current state has been build on top
existing state structures, as opposed to colonial structures in most other
instances. 

As discussed previously, pre-colonial state groups were in a prime position to
seize the state apparatus upon independence \citep{Paine2019,
Wucherpfennig2016}. Often, this happened through more or less direct hand-over
from colonial authorities, as in Rwanda or Libya, or by physical proximity to
the levers of power, as in Ghana where the Ashanti reached a power sharing
agreement with the state in Accra (less than 200km South of the Ashanti capital
of Kumasi). In these cases I expect the institutions and experience of
statecraft to work toward greater state capacity and political trust
\citep{Depetris-Chauvin2016}. This should be especially true in the areas
surrounding the capital (of the pre-colonial state), where the pre-colonial
state would have deeper historical and institutional roots than in formerly more
peripheral areas. In areas where the modern capital (more or less) overlaps with
that of a pre-colonial state, this better reflects the existing underlying
topgraphy of statehood , than cases where the post-colonial capital does not
have any pre-colonial roots. In other words, these areas can be said to be
relatively less `artificial' \citep{Alesina2011}. Examples of this include
Morocco and Burkina Faso who have enjoyed relative continuity in their power
structures, and relatively less state based violence (despite a number of
coups). 

I further argue that the likelihood that a post-colonial state has at least some
basis in a pre-colonial state, in terms of institutions, legitimacy or elite
networks, can be proxied by the distance to the current capital. In cases where
the pre-colonial state territory coincides with the post-independence capital, it
is a strong indication of continuity. Moving further away one can observe
`hybrid' cases, such as the aforementioned Ashanti in Ghana. 

% Cases were the  capitals are near, but not the same (Ashanti/Ghana)

% Even when the institutions and
% networks of the pre-colonial state was not handed the keys to the kingdom
% directly or physically situated close to the levers of power (EXAMPLE?), they
% often dominated the post-independence political landscape, presumably through
% superior organization. An example of this is the Majarteen sultanate in Somalia,
% where the corresponding clan quickly came to hold a commanding position in
% nascent republic, and were only dislodged from that position by a grand
% coalition of the other major Somali clans [WHERE DID I READ THIS?]. 

Conversely then, in cases where a pre-colonial state capital ends up far from
the post-independence capital, I argue it is an indication that there is less
continuity. This excludes the state capacity and political trust mechanism, and
the institutions, networks and symbols of the pre-colonial states act as
regional counter weights instead. The central state has two options: either
continue to allow some degree of indirect rule, or impose central authority.
Empirically central states tend to expand their influence over time, even if it
results in conflict \citep{Ying_2020}. Nigeria is an interesting example of this
as the post-independence capital Lagos is on the southern coast, whereas the
North contained the lions share of two pre-colonial empires (Sokoto and Bornu).
In the years following independence `Northerners' wielded an out sized share of
power, as indicated by the location of numerous development projects and
military installations \citep{Bates2008a}. This is another example of the effect
of the political organization of pre-colonial states relative to the rest of the
country. While `Northerners' dominated the political system, that system was not
in any way based on their own institutions, which were still relatively intact
in both Sokoto and Borno. However, with time, their dominant position faded, and
when in 1991 the government tried to split the federal state of Sokoto in two,
the sultan opposed the regime and was instrumental in its downfall
\citep{HiribarrenVincent2017AHoB}. While not leading to violent confrontation
between the state and the regional elite, the case nevertheless illustrates the
tensions created when pre-colonial states do not form the basis of the
post-colonial state, but exists rather as a competitor to the central state.
While the causal chain is harder to trace, the current federal state of Borno
has been a hot bed of state based violence, much of which has been centered
around Maiduguri, the last capital of independent Borno.
% Perhaps this is presicely because Borno was never fully incorporated which
% left the state woulnorabel to outside threats.

In summary, I expect pre-colonial state presence\footnote{This term is explained
in detail below.} to correlate negatively with conflict in areas close to the
(post-independence) capital, and positively in areas further away from the
capital.

\section{The Geo-ISD}

A persistent problem in the literature on pre-colonial states, has been the
availability of reliable data. The \citet{Murdock1967} map of ethnic groups and
their corresponding `jurisdictional hierarchy' index of political organization
is one of the most frequently used data sources for constructing
per-ethnic-group measures of statehood. However, this data has a number of
issues (as enumerated by \citet{Michalopoulos2018}), such as lack of potential
overlap between ethnic groups, static borders, covering a relatively short time
span and lack of within-group variation. A further disadvantage of this approach
is that by using ethnic groups (and not states) as a starting point, there is a
substantial potential for missingness (as not all states are easily tied to a
specific ethnic group). For example, \citet{Paine2019} despite using a `low bar'
for statehood and consulting numerous sources, only codes 28 groups in
Sub-Saharan Africa as having ties to a pre-colonial state.\footnote{This is
partially also a result of the criteria of `independence on the eve of
colonization'.} Using a similar approach \citet{Wig2016} identifies 45 state
groups in the same region. Using the State Antiquity Index
\citep{Bockstette2012} as a starting point \citet{Depetris-Chauvin2016} avoids
the limitation of only including state with clear ties to ethnic groups.
Nevertheless, his data only includes 54 states in the 1800-1850+ period, as
compared to 104 in the ISD version 2, despite using less strict criteria for
statehood.

%{{{ comments
% Depetris-chau lack some states (53 in SSA), despite an apparently less stringent definition of statehood (also less clear), includes Hausa states, Aro trading confederacy, Air sultanate?, Azande?. And lacks most of the variation across time and
% space. Starts with states antiquity compliments with Murdoch and other
% historical atlases
%}}}

For this paper, as part of the Geo-ISD project consisting of Charles Butcher,
research assistant Eirin Haugseth and myself, constructed a novel data set,
which builds on earlier work by the ISD project \citep{Butcher2020}. The ISD v2
picks up a large number of states that are missed by similar data sets, while
avoiding the use of arbitrary demands of statehood such as recognition by
European powers \citep{Butcher2020}. The Geo-ISD builds on the original data set
by geocoding the borders of the included states. It also introduces an
innovative method for capturing the historical presence of pre-colonial states
that addresses some of the weaknesses of the existing geocoded data, namely the
number of states identified, static borders, lack of overlap and implied
uniformity in state control across territory.

The original idea behind the Geo-ISD was to use historically contemporary maps,
as close to the primary sources as possible, containing the borders of states to
determine the extent of different states as close to yearly as possible. Maps
were georeferenced in QGIS by connecting recognizable features in the maps
(cities, distinctive capes, islands, etc) to their real locations when compared
to satellite imagery containing exact location data.  The result is a version of
the map that is slightly distorted to better fit reality, as can be seen in
Figure \ref{Arrowsmith}.\footnote{The exact specifications for the
georeferencing and subsequent transformation will be supplied in the code book,
included in the online supplemental material.} Shapes of the states depicted in
the map were traced for those shapes that were included in the ISD v2 for the
given year. For example Bornou would be included, but not the neighboring Howssa
in Figure \ref{Arrowsmith}, as the shape drawn could potentially refer to either
the Haussa ethnic group (a common occurrence in these maps) or the multiple
Houssa states, neither of which qualify as states in the ISD. This process was
repeated for all the maps at the time included in
\href{https://www.davidrumsey.com}{the David Rumsey project} data base of
historical maps, matching the region of Africa in the period 1800 to 1914. The
start date was chosen due to the limitations of the ISD v2 which only extends
back to 1814 (but includes the founding date of states going further back), and
the fact that the quality of contemporaneous maps becomes substantially  worse
beyond this point \citep{Bassett_1994}. To control for some of the potential
biases of relying on maps drawn a long time ago, and by non indigenous (mostly
Western) mapmakers,\footnote{Discussed further below.} the same process was
repeated using historical atlases compiled by later historians (several of which
were also consulted by \citet{Depetris-Chauvin2016} and \citet{Paine2019}). The
result was over 3400 polygons (state-shape-years) covering the period 1800 to
1914 for continental Africa and Madagascar. For some pre-colonial states in the
ISD there were no maps for any years, some are covered only for some of the
years they are in the ISD, but a number of them are covered by multiple maps for
many years. Of the 109 states included for Africa in the ISD v2, the Geo-ISD was
able to identify at least one set of borders for 82 states. This is a considerable
improvement on previous data sets, yet still not capturing the universe of
independent African states in the period.

%\end{multicols}

\begin{figure}[h!tpb]
	\centering
	\includegraphics[width=0.8\linewidth]{img/Arrowsmith.jpg}
	\caption{Example of georeferenced map}%
	\label{Arrowsmith}
\end{figure}

%\begin{multicols}{2}

\subsection{The mountains of Kong}

The accuracy of the historical maps used to create the polygons of the Geo-ISD
is a natural concern. Who typically drew these maps? Based on what sources? For
what purposes? And with what level of technical accuracy? 

From working with them, and by tracing the creation of some of the maps, I found
that most of the maps from the David Rumsey project were from atlases published
for commercial purposes by individuals or small publishing companies
specializing in this type of publications. Most are from English or American
atlases, but French, Italian, German and other sources are included as well. The
maps were based on a combination of existing maps updated with `the latest
sources' (a fact frequently boasted in the title of the atlas), which for the
majority of the period meant explorers or geographers on missions from their
respective geographical societies.\footnote{Later maps additionally draw on the
	work of military surveyors, but as far as I have been able to tell the
majority are still based primarily on the work of explorers and past maps.} In
the words of \citet[47-48]{Stone1995}: `Cartography in Africa [in the 19th
century] is still a mix of measurement, less accurate observations, word of
mouth, previous maps and sources, educated guesses and pure conjecture.
Nevertheless a distinct improvement on the maps of previous periods.'. Because
of this, I expect two types of errors: errors resulting from measurement, and
bias resulting from misconceptions (deliberate or not) of what constituted the
borders of a polity at the time. I also expect errors to be replicated by other
maps, before eventually being corrected. One example of this is the nonexistent
Mountains of Kong, which can be seen in Figure \ref{Arrowsmith} as the mountain
range stretching across most of the continent from East to West. These mountains
were replicated in maps for the better part of the nineteenth century before
being finally wiped from the map \citep{Bassett_1991}.

While the quote above might lead one to expect considerable measurement error,
by my estimate this error only amounts to 36.9km on average for the shapes
included in the GeoISD. This estimate is based on the estimated mean distance of
the coastline in the maps to the real coastline, along the borders of the states
that were traced. This captures measurement error explicitly, because regardless
of where states did or did not extend their control, the coastline in the map
should line up with the real coastline. This means that there are often multiple
estimates for each map, reflecting how the accuracy is better in some places
than in others. If this difference was above 100km the maps were deemed too
inaccurate and excluded from the sample. Error estimates for states that lay
inland were not included, because consistently matching features from the maps
to real geographical features was not feasible other than when using the
coastline, which provides a visible line of comparison. Additionally, this type
of error only adds noise, because it is equally likely to err in one direction
or the other. Thus it should not affect coefficient estimates, but could
potentially affect standard error estimates.

Although I have not been able to find sources discussing specifically how
cartographers determined the borders of different polities, it is probable that
they in large part relied on local verbal sources (word of mouth). An example of
this can be glimpsed when the explorer Mungo Park effectively dubbed the
mountains of Kong.\footnote{`I gained the summit of a hill, from whence I had an
	extensive view of the country. Towards the south-east, appeared some
	very distant mountains, which I had formerly seen from an eminence near
	Marraboo, where the people informed me, that these mountains were
	situated in a large and powerful kingdom called Kong; the sovereign of
	which could raise a much greater army than the King of Bambarra.'
\citep[CHAPTER XVIII]{ParkMungo2015Titi}.} Or from when expeditions such as
Park's were escorted by representatives of the rulers of the various polities
they passed through until reaching \textit{frontier towns}, where they would be
met by representative of the next ruler \citep{ParkMungo2015Titi}. In the maps
resulting from such encounters both their sources as well as the cartographers
themselves could have introduced bias to the resulting maps. Sources were likely
to be rulers or their representatives, with incentives for aggrandisement. The
explorers and cartographers on their part represented European rulers with an
eye toward colonial expansion. It less clear how this would affect the resulting
borders drawn. One potential bias would be to exaggerate the domains of your own
governments prospective colonies, and vice versa. However, I did not observe any
systematic differences between the maps based on their nationality. In fact,
their colonial ambitions could just as easily have promoted accuracy, as any
potential military expedition would benefit from accurate information
\citep{Bassett_1994}.

At the very least there should be heterogeneity in the conceptualization of
territoriality. What determines where a given source (or the cartographer in the
second instance) draws the borders of a polity, or how this would vary with
their respective conceptualization of states, polities and ethnic groups, is
impossible do determine. However, thanks to (usually) having multiple maps for
each state, the variation can be leveraged to create a measure of the
\textit{degree} to which a state had a presence in a given area over the time
period as a whole. When maps disagreed on where the various borders were, I
interpret this as either true variation across time, or as an indication of the
ambiguity of where a given state had nominal or real control. In the areas where
all the maps agree, one could be quite sure that the given polity had real
presence.  While in areas where only one map indicated that the state was
presence, this could either be wrong, an indication of nominal as opposed to
more real presence or some other form of limited presence. The coding process of
looking at hundreds of maps strengthened this initial intuition, and the
resulting figures of state presence drawn from the complete data lends it
further credence. Figure \ref{libya} demonstrates how the authority of the
Libyan state gradually faded into the desert, that the Benghazi/Cyraneica region
was not always or fully under the control of Tripoli, and its even more tenuous
hold on the Fezzan region to the south.\footnote{More such examples are included
in the online appendix.}

%\end{multicols}

\begin{figure}[htpb]
	\centering
	\includegraphics[width=0.8\textwidth,keepaspectratio]{img/libya.png}
	\caption{The historical state presence of Libyan state of Tripoli.
	Current borders of Libya and Egypt in red.}
	\label{libya}
\end{figure}

%\begin{multicols}{2}

To counter balance some of the potential bias in the historically
contemporaneous maps a number of maps from historical atlases were included.
These are works compiled by historians with both the accumulated knowledge of
history and modern instruments of cartography at their disposal. These sources
should be able to reduce problems of the historically contemporaneous maps, such
as word of mouth sources exaggerating the extent of polities.\footnote{Although
	they should also be more accurate in terms measurement error, I found
that this was not always the case. For example \citep{Kasule1998} completely
misplaces Wadai, putting the capital of Wara (and thus the rest of the polity
with it) at least 200km North West of its actual position.} The historical
atlases were scanned, then georeferenced and polities traced in the same manner
as the pre-colonial states in the historically contemporaneous maps. The
historical atlases frequently depicted the borders of states over a period of
years in a single map. In these cases the resulting state shapes were duplicated
for each year in the period. This has the benefit of placing a larger emphasis
on the historical atlases, at the cost of being more static.\footnote{Implicitly
assuming constant borders, and in the absence of other sources implying uniform
control across territory.} Figure \ref{atlasmaps} in the appendix lists the
historical atlases used in the Geo-ISD.

The resulting data can be compiled in different ways, to provide different
insights. For the analysis in this paper I rely on a measure of `state
presence', similar in concept to that of `state history' introduced by
\citet{Depetris-Chauvin2016}.\footnote{The main difference between these
	measures is that `state history' is measured in 2 by 2 decimal degree
	grid cells, only includes Sub Saharan Africa, includes fewer states
	despite going further back in time, usually only includes one shape per
state, which implies static borders and uniform presence throughout the
territory.} I measure `state presence' the as number of maps that indicate that
a state was present there, counting only those of the state most often present
in that cell. If a cell bordering Tunisia and Libya contains 40 shapes of
Tunisia and 7 shapes of Libya/Tripoli, only the Tunisian shapes are counted.
Including the Libyan shapes would risk over counting, as it most likely does not
represent additional state presence, but rather overlapping or contested state
presence. 

%\end{multicols}

\begin{figure}[htpb]
	\centering
	%\includegraphics[width=0.8\linewidth]{../Rplot_ln_sp_int.pdf}
	\includegraphics[width=\linewidth]{../R/Output/sqrtSpAll.pdf}
	\caption{State presence (sqrt transformed) with interpolated years based
	on historical atlases.}
	\label{Sp_i}
\end{figure}

%\begin{multicols}{2}

\section{Research design}

\subsection{Dependent variable}

The units of analysis are PRIO grid cells with a non-zero population density in
1600,\footnote{This primarily excludes the Sahara and Kalahari deserts.} which
are 0.5 by 0.5 decimal degree cells, which equals about 55 by 55km at the
equator \citep{Tollefsen2012}. Due to the explanatory variable being
time-invariant, the analysis is limited to a cross section. The dependent
variable is state based conflict related fatalities per grid cell over the
period 1989-2020, from the GED project \citep{Sundberg2013}. This is a measure
of the overall level of conflict in the post cold war period. The start date of
1989 was dictated by availability rather than chosen by design, as it likely
biases against finding results (positive or negative), because all the
mechanisms discussed in the theory section are more relevant closer to
independence and would fade in relative importance with the passing of time.

% While some of the theoretical mechanisms discussed above might be more closely
% related to a specific type of state-based conflict, such as wars of secession,
% these are difficult to separate entirely. That is, they are unlikely to manifest
% \textit{exclusively} as a specific type. This does not mean that the choice of
% dependent variable biases in favor or finding a relationship. In fact
% \citet{Wishman2021a} finds that pre-colonial state presence has a negative
% effect on communal violence events. 

\subsection{Independent variable}

The main explanatory variable is the per grid cell state presence, as defined in
the data section.

% However, only the state that has the most presence overall in that grid cell is
% included, so as to avoid over counting in cases of overlap (contested
% sovereignty) or when territory has changed hands. This measure has the benefit
% of including more data, which allows for the approximation of relative degrees
% of state presence by one state in any year, as described in the data section
% above. At the same time it avoids over counting state presence where there were
% overlaps in sovereignty or changes in who controlled the territory. 

Because I do not expect the relationship between pre-colonial state presence and
civil conflict to be linear, and because the data is heavily skewed, the
variable is square root transformed. While log transformation is more often
employed in the previous literature, and does produce a more evenly distributed
variable in this case as well, adding 1 to all values could potentially
introduce bias. I therefore present the more conservative approach of reporting
square root transformation as the main findings.

\subsection{Controls}

Because the treatment variable in this case far predates the outcome, there is a
substantial risk of introducing post treatment bias when including control
variables. In choosing which control variables to include, I therefore 
balanced a trade off between potential post treatment bias and omitted variable
bias. 

Mountains help in early state formation by providing protection and limiting the
exit options of sedentary farmers \citep{Carneiro1988}. Mountainous terrain has
also been linked with civil conflict by providing shelter for rebel groups
\citep{Hegre2006}, although this relationship is debated 
\citep{Buhaug2002}. The data is from the PRIO-grid data set, but originally 
from \citet{Blyth2002}. 

Water is essential for state formation. States typically formed either as
coastal cities, close to navigable rivers or by the shores of great lakes.
People still tend to live next to a source of water, thus this acts as a proxy
for population density, and fighting usually happens where there are (at least
some) people. The data on water as a percentage of the grid surface is from the
PRIO-grid data set, but originally from the European Space Agency
\citep{Bontemps2009}.

Distance to the coast could affect both state presence and conflict in a number
of ways. First, as stated above, states were more likely to be formed along the
coast as it connected cities and people. A special case for Africa is also the
existence of slave raiding/trading states that formed along the eastern and
western coasts of the continent. These state's raison d'être was raiding slaves
from tribes and peoples inland and selling them to coastal traders (European in
the West and Arab in the east). \citet{Nunn2008} argues that this state of
affairs left legacies of mistrust and antagonism, which has resulted in
increased levels of current day conflict. Distance to the coast could also be
related to the measure of state presence through the fact that our measure is
based on European observations (maps), which undoubtedly had better coverage
along the coast, especially for the earlier periods. Distance to the coast could
further be related to conflict through lower levels of development. The distance
to coast data is from \citet{Wessel1996}. The variable was log transformed to
account for a non-linear relationship.

As with water, barren terrain could be a (negative) pre-condition for state
building as well as proxy for later population densities, and thus could
correlate with both state presence and levels of conflict. The data is from
\citet{Bontemps2009}.

The states of North Africa are overrepresented in the Geo-ISD data, due to the
geographical proximity, and the accompanying historical familiarity to European
map makers. This affects Morocco most particularly, as can be seen in Figure
\ref{Sp_i}. The reason this affects Morocco in particular is that the remaining
North African states were under Ottoman suzerainty for much of the period,
meaning more time for more maps to accumulate. If North Africa is also more or less
conflict prone than the rest of Africa on average, the inflated values of state
presence would bias the estimated coefficients. Accordingly, a dummy variable
for the region of North Africa was included.

Population density is only added as an extended control despite the stronger
theoretical expectation that it could be a confounding variable, because there
are no accurate measures of population densities that predate most of the
states in the Geo-ISD. The best available estimates come from the HYDE project
\citep{Goldewijk2016}. In order to predate most of the states in the Geo-ISD I
use the estimates from 1600.

Distance to international boundaries could be related to state presence because,
despite their reputation, African borders were not drawn completely at random
(or along meridian lines). For example, the boundary between northern Nigeria and
Niger were based on the extent of the Sokoto Caliphate and the neighboring
Kanem-Bornu (or just Bornu) empire \citep{HiribarrenVincent2017AHoB}. Proximity
to international boundary has also been found to predict conflict
\citep{Buhaug2002}. I use the measure included in the PRIO grid data, which is
originally from \citet{Weidmann2010a}.

\subsection{Alternative measures}

As an alternative to combat related fatalities, I also ran models using the
count of state based conflict events. This captures much of the same general
level of conflict during the period as the fatalities measure does, but the
focus is slightly different. Fatalities captures severity of conflict as part of
the general conflict level, whereas the number of conflict events captures the
frequency of conflict. This would be the difference between few, or short, but
highly lethal conflicts, versus lengthy conflict of relatively low
intensity, or recurring conflicts. I do not expect there to be a substantial
difference in the results from these measures based on the theory presented
above.

In addition to the square root transformed version of the main independent
variable, I also ran models using the more common log transformation. The
results remained substantially the same (see online appendix?). 

The results of the main model are also robust to using all PRIO grid cells
regardless of population density (when they converge).

As a further robustness check I also ran models using a measure of state
presence that counts if there was any maps that included a state in a
grid-cell-year (as a sum of yearly dummies). In other words, this could be at
most 214, and is more so a measure of the maximum \textit{extent} of state
presence, and is less accurate in terms of variations in depth. The benefit of
this measure is that it avoids some of the potential over representation of
countries frequently mapped by Europeans, such as the North African states (due
to proximity). As with the main measure of state presence, only the shapes of
the state that was most present in that grid cell throughout the sample period
was traced. Results remain substantially the same for all specifications.

\subsection{Modelling}

To account for the potential post treatment bias (vis-a-vis potential omitted
variable bias) controls were added step wise with increasing potential for post
treatment bias.

To account for the dependent variable being count data (count of deaths
(fatalities) and count of conflict events (state based)) all model
specifications reported below are negative binomial regressions. A fitness test
for whether negative binomial or Poisson regression is most appropriate, was
conducted and confirmed that negative binomial produced a better fit than
Poisson.

The theoretical expectation is that pre-colonial state presence is conflict
reducing in areas close to the post independence capital, and conflict inducing
in more remote areas. In other words that the effect of pre-colonial state
presence is moderated by distance to capital. I model this relationship using an
interaction models using the log transformed distance to capital, in order to
account for skewed data and because I do not expect the relationship to be
linear the distance to capital is log transformed using the natural logarithm.

The dependent variable contains excess zeros (8937 zeros relative to 100 counts
of 1 fatality, the second most frequent outcome). Additionally, I can not be
sure that the main independent variable affects affects the likelihood of there
being any fatalities in a cell (or if it remains a zero), the same way it might
influence the severity of conflict once a cell has seen at least one fatality. I
therefore employ a zero inflated negative binomial model. The first step of this
two step approach is a logit model that models whether cells experience conflict
or not. I used the same set of controls as I do not expect any of them to
exclusively affect conflict severity, nor do I expect their relation to the main
independent variable to be substantially different for an onset model. The
second step is a negative binomial estimation of conflict severity, or the
number of fatalities in a cell that has seen at least one.

Given what is known about spatial diffusion of conflict, there is reason to
suspect some spatial autocorrelation. However, controlling for this would
introduce a source of post treatment bias. Nevertheless, as an additional
robustness check I ran some models with queen pattern spatial lags. Most of
these models did not converge, but those that did remained substantially similar
to the main models.

\subsection{Testing the mechanism}

Testing the proposed mechanisms explicitly would require data on where elite
networks and institutions have survived from pre-colonial states, or data on
rebel groups use of symbols invoking past statehood. Fortunately, the
differences in the approach to colonial governance between Britain and France
could provide a proxy for the elite networks and institutions mechanisms. France
generally sought (more successfully in some cases than in others) to fully
incorporate and rule their colonies directly, dismantling as much of the
existing institutions as possible and avoiding a reliance on native
administrators \citep{Blanton_2001}. Britain on the other hand pursued a
strategy of indirect rule, preferring to leave local rulers to administer their
own territories, and relying on their own institutions to do so
\citep{Blanton_2001}. Former British colonies should therefore be more likely to
have preserved the elite networks and institutions that could be used to
mobilize against the state. I therefor ran all the models including controls for
former French and British colonies. In these models, the North Africa control
had to be dropped because most of North Africa were French colonies at some
point and models would not converge with both included. 

\section{Results}

Despite the data on conflict starting nearly three decades after most of Africa
achieved independence (at which point the effects of pre-colonial state presence
on conflict should be most pronounced), there is a significant
and positive direct effect of state presence on conflict (.11, SE = .01 and .10,
SE = .01 in the main models). This effect is robust and stable across all models
(see Table \ref{statebaseddeaths} and Table \ref{state_based}). All controls
have the expected signs except distance to coast in Table \ref{state_based},
using state based conflict events as the dependent variable. However, the
coefficient indicates a substantially negligible effect, and the statistical
significance might be more a reflection of the large number of observations
(9492) rather than any causal effect. The second model in Table
\ref{statebaseddeaths} indicates that North Africa has experienced more conflict
fatalities than the rest of Africa comparatively. However, this effect becomes
smaller and is no longer significant at the $ p < .01 $ level after controlling
for population density in 1600, suggesting at least part of the effect could be
due to higher levels of population density.

However, the results of the interaction models reveal a more nuanced picture. As
seen in Table \ref{interaction_statebaseddeaths} and Table
\ref{interaction_state_based}, when controlling for the effect of the
interaction term between state presence and distance to capital, state presence
has a conflict reducing effect, albeit a non-significant effect in the
fatalities models. All control variables behave as expected, and similarly as in
the non interaction models. Unsurprisingly, distance to capital (log) is
negative, as there is generally less conflict in and around capital cities. The
interaction term is significant and positive across all model specifications, in
line with the theoretical expectations. 

In terms of substantial effects Figure \ref{deaths_zinb}, which models the
predictions of the second stage ZINB model included in Table \ref{zinb},
provides a more intuitive interpretation of the interaction between state
presence and how it affects the severity state based conflict. State presence is
negatively correlated with both conflict measure close to the capital, but
becomes positive and significant further away from the capital. This is in line
with an interpretation that state presence can be conflict reducing in those
cases where it makes a territory less artificial, by providing institutions and
elite networks on which to build a state. In cases where there is no state
presence in capital areas, the model predicts an additional 406 fatalities from
state based conflict. The effect drops rapidly as state presence increases. On
the other hand, in areas with no experience of statehood that are far from the
capital, the model predicts almost no additional fatalities (similar to high
levels of state presence in/around the capital). However, as state presence
increases, so does predicted fatalities. For a moderately high level of
pre-colonial state presence of 200, the main model predicts an additional 380
fatalities. While this is not a test of the specific mechanisms outlined in the
theory section, the results do indicate that institutions, elite networks or
some other legacy of pre-colonial statehood raises the scope of conflict
when far from the capital area.

%\end{multicols}

\begin{figure}[htpb]
	\centering
	\includegraphics[width=\linewidth]{"../R/Output/zinbplot.pdf"}
	\caption{}
	\label{deaths_zinb}
\end{figure}

%\begin{multicols}{2}

The results of the models including controls for French and British colonies
(see Table \ref{zinb}) are interesting in that the predicted increase in conflict
severity in capital areas without prior experience of statehood is at 991
additional fatalities. The signs of the colonial controls are as
expected. However, the conflict inducing effect of being a former British
colony is only significant on conflict onset (the first stage logit
model). Similarly, the conflict reducing effect of being a former French colony
is only significant in terms of reducing conflict severity (second stage
negative binomial count model). These results can not be interpreted with any
certainty, but do perhaps suggest that French style governance left stronger
central government that were better equipped to limit the scope of conflicts,
while the British tradition of indirect rule left a larger number of potential
rivals to the central government.

\subsection{Alternative explanations}

An alternative interpretation of the results could be that this is a
story of more coherent ethnic groups being more likely to be associated with
states, and more likely to (perhaps better able to) challenge the government
when situated far from the capital. However, getting closer to the real
causality requires untangling, in each case, if a given conflict is related to
an ethnic group's ties to a pre-colonial state or not. This lies outside the
scope of this paper, if it is even possible.

Another interpretation that is not controlled for in this paper is the
possibility that past conflict drives both state creation and current conflict.
However, data on past conflict is meager, especially so for Africa. To my
knowledge the \citet{Brecke1999} data set is the most complete. Even so, it
relies on written histories of which there is little for pre-colonial
sub-Saharan Africa. What is worse, the missing will be considerably biased
because kings and states are far more likely to chronicle their warfare in the
form of written records.

\section{Conclusion}

Drawing on the emerging literature on pre-colonial states \citep{Paine2019,
Depetris-Chauvin2016}, institutions \citep{Wig2016, Englebert2002,
Michalopoulos2018} and civil conflict, and on newly compiled data this paper
reexamined the relationship between pre-colonial states and civil conflict. I
find support for a conflict reducing effect of pre-colonial state presence, but
this is conditioned on proximity to the post-colonial capital. On the other
hand, I find strong evidence for a conflict inducing effect, particularly in
areas far from the post-independence capital. These results are robust to
alternative measurements and model specifications.

These findings have a few important implications. First, they demonstrate that
pre-colonial states can be a blessing or a curse depending on whether they form
the basis of the modern state or a point of opposition to it. Second, that local
political histories matter, and should be taken into consideration by
policymakers and scholars alike. Finally, these results suggests that wherever
colonizers left novel constellations of pre-existing polities, they potentially
sowed the seeds of future conflict. More research using global data is needed to
test whether general trends for Africa hold outside that continent as well.

%\end{multicols}

\pagebreak

\bibliographystyle{apalike}
\bibliography{../lib.bib}

\pagebreak
\section*{Appendix}


% Table created by stargazer v.5.2.2 by Marek Hlavac, Harvard University. E-mail: hlavac at fas.harvard.edu
% Date and time: Tue, Dec 14, 2021 - 12:56:19 PM
\begin{table}[!htbp] \centering 
  \caption{Summary Statistics} 
  \label{summarystats} 
\begin{tabular}{@{\extracolsep{1pt}}lccccccc} 
\\[-1.8ex]\hline 
\hline \\[-1.8ex] 
Statistic & \multicolumn{1}{c}{N} & \multicolumn{1}{c}{Mean} & \multicolumn{1}{c}{St. Dev.} & \multicolumn{1}{c}{Min} & \multicolumn{1}{c}{Pctl(25)} & \multicolumn{1}{c}{Pctl(75)} & \multicolumn{1}{c}{Max} \\ 
\hline \\[-1.8ex] 
Fatalities & 10,652 & 45.8 & 883.1 & 0 & 0 & 0 & 79,920 \\ 
Conflict events & 10,652 & 2.4 & 27.0 & 0 & 0 & 0 & 1,940 \\ 
State presence & 10,652 & 50.2 & 88.1 & 0 & 2 & 62 & 629 \\ 
Distance
		    to boundary & 10,652 & 138.1 & 122.6 & 0.003 & 39.6 & 200.7 & 668.0 \\ 
Distance to capital & 10,652 & 671.3 & 411.0 & 3.7 & 338.0 & 956.2 & 2,482.5 \\ 
Barren & 10,652 & 32.7 & 44.4 & 0 & 0 & 97.9 & 100 \\ 
Mountainous & 10,492 & 0.1 & 0.3 & 0.0 & 0.0 & 0.1 & 1.0 \\ 
Water & 10,652 & 4.8 & 17.7 & 0.0 & 0.0 & 0.1 & 100.0 \\ 
Distance to coast & 10,652 & 599,064.4 & 460,732.6 & 0 & 185,784.8 & 956,404.4 & 1,761,700 \\ 
popd & 10,559 & 2.5 & 9.9 & 0.0 & 0.1 & 2.1 & 447.9 \\ 
\hline \\[-1.8ex] 
\end{tabular} 
\end{table} 


\begin{sidewaystable}
\begin{center}
\scalebox{1}{
\begin{tabular}{l c c c c}
\hline
 & Geography & North Africa & Population densisty & Distance
		  to border \\
\hline
(Intercept)     & $2.95^{***}$  & $2.93^{***}$  & $2.04^{***}$  & $2.15^{***}$  \\
                & $(0.14)$      & $(0.14)$      & $(0.16)$      & $(0.16)$      \\
sqrtSpAll       & $0.12^{***}$  & $0.11^{***}$  & $0.11^{***}$  & $0.11^{***}$  \\
                & $(0.01)$      & $(0.01)$      & $(0.01)$      & $(0.01)$      \\
mountains\_mean & $1.27^{***}$  & $1.44^{***}$  & $0.97^{***}$  & $0.89^{***}$  \\
                & $(0.24)$      & $(0.24)$      & $(0.23)$      & $(0.23)$      \\
water\_gc       & $0.02^{***}$  & $0.02^{***}$  & $0.01^{**}$   & $0.01^{*}$    \\
                & $(0.00)$      & $(0.00)$      & $(0.00)$      & $(0.00)$      \\
barren\_gc      & $-0.02^{***}$ & $-0.03^{***}$ & $-0.02^{***}$ & $-0.02^{***}$ \\
                & $(0.00)$      & $(0.00)$      & $(0.00)$      & $(0.00)$      \\
distcoast       & $0.00$        & $0.00$        & $0.00$        & $0.00$        \\
                & $(0.00)$      & $(0.00)$      & $(0.00)$      & $(0.00)$      \\
region3         &               & $0.55^{**}$   & $0.39^{*}$    & $0.47^{*}$    \\
                &               & $(0.18)$      & $(0.18)$      & $(0.18)$      \\
logPopd         &               &               & $0.78^{***}$  & $0.78^{***}$  \\
                &               &               & $(0.09)$      & $(0.09)$      \\
bdist3          &               &               &               & $-0.00^{*}$   \\
                &               &               &               & $(0.00)$      \\
\hline
AIC             & $27602.81$    & $27592.23$    & $27501.07$    & $27497.05$    \\
BIC             & $27652.92$    & $27649.50$    & $27565.49$    & $27568.64$    \\
Log Likelihood  & $-13794.40$   & $-13788.12$   & $-13741.53$   & $-13738.53$   \\
Deviance        & $3517.71$     & $3518.78$     & $3526.90$     & $3527.38$     \\
Num. obs.       & $9492$        & $9492$        & $9492$        & $9492$        \\
\hline
\multicolumn{5}{l}{\scriptsize{$^{***}p<0.001$; $^{**}p<0.01$; $^{*}p<0.05$; $^{\cdot}p<0.1$}}
\end{tabular}
}
\caption{Fatalities}
\label{statebaseddeaths}
\end{center}
\end{sidewaystable}


\begin{sidewaystable}
\begin{center}
\scalebox{1}{
\begin{tabular}{l c c c c}
\hline
 & Geography & North Africa & Population densisty & Distance
		  to border \\
\hline
(Intercept)          & $5.15^{***}$  & $8.21^{***}$  & $3.65^{***}$  & $3.58^{***}$    \\
                     & $(0.75)$      & $(0.75)$      & $(0.75)$      & $(0.76)$        \\
sqrtSpAll            & $-0.20^{*}$   & $-0.25^{*}$   & $-0.26^{**}$  & $-0.23^{*}$     \\
                     & $(0.10)$      & $(0.10)$      & $(0.10)$      & $(0.10)$        \\
logCapdist           & $-0.40^{**}$  & $-0.88^{***}$ & $-0.26^{*}$   & $-0.23^{\cdot}$ \\
                     & $(0.12)$      & $(0.12)$      & $(0.12)$      & $(0.12)$        \\
mountains\_mean      & $1.70^{***}$  & $1.63^{***}$  & $1.12^{***}$  & $1.05^{***}$    \\
                     & $(0.24)$      & $(0.24)$      & $(0.23)$      & $(0.23)$        \\
water\_gc            & $0.02^{***}$  & $0.02^{***}$  & $0.01^{**}$   & $0.01^{*}$      \\
                     & $(0.00)$      & $(0.00)$      & $(0.00)$      & $(0.00)$        \\
barren\_gc           & $-0.02^{***}$ & $-0.02^{***}$ & $-0.02^{***}$ & $-0.02^{***}$   \\
                     & $(0.00)$      & $(0.00)$      & $(0.00)$      & $(0.00)$        \\
distcoast            & $0.00^{**}$   & $0.00^{**}$   & $0.00$        & $0.00$          \\
                     & $(0.00)$      & $(0.00)$      & $(0.00)$      & $(0.00)$        \\
sqrtSpAll:logCapdist & $0.05^{**}$   & $0.05^{**}$   & $0.06^{***}$  & $0.05^{***}$    \\
                     & $(0.02)$      & $(0.02)$      & $(0.02)$      & $(0.02)$        \\
region3              &               & $0.95^{***}$  & $0.40^{*}$    & $0.46^{*}$      \\
                     &               & $(0.18)$      & $(0.18)$      & $(0.18)$        \\
logPopd              &               &               & $0.80^{***}$  & $0.80^{***}$    \\
                     &               &               & $(0.09)$      & $(0.09)$        \\
bdist3               &               &               &               & $-0.00^{*}$     \\
                     &               &               &               & $(0.00)$        \\
\hline
AIC                  & $27586.19$    & $27594.05$    & $27494.61$    & $27491.37$      \\
BIC                  & $27650.62$    & $27665.63$    & $27573.35$    & $27577.27$      \\
Log Likelihood       & $-13784.10$   & $-13787.02$   & $-13736.31$   & $-13733.69$     \\
Deviance             & $3519.49$     & $3519.06$     & $3527.88$     & $3528.29$       \\
Num. obs.            & $9492$        & $9492$        & $9492$        & $9492$          \\
\hline
\multicolumn{5}{l}{\scriptsize{$^{***}p<0.001$; $^{**}p<0.01$; $^{*}p<0.05$; $^{\cdot}p<0.1$}}
\end{tabular}
}
\caption{Fatalities * Distance to capital}
\label{interaction_statebaseddeaths}
\end{center}
\end{sidewaystable}


\begin{sidewaystable}
\begin{center}
\scalebox{1}{
\begin{tabular}{l c c c c}
\hline
 & Geography & North Africa & Population densisty & Distance
		  to border \\
\hline
(Intercept)     & $0.79^{***}$  & $0.83^{***}$  & $-0.34^{**}$    & $-0.20^{\cdot}$ \\
                & $(0.10)$      & $(0.10)$      & $(0.11)$        & $(0.11)$        \\
sqrtSpAll       & $0.06^{***}$  & $0.04^{***}$  & $0.03^{***}$    & $0.04^{***}$    \\
                & $(0.01)$      & $(0.01)$      & $(0.01)$        & $(0.01)$        \\
mountains\_mean & $0.57^{***}$  & $0.47^{**}$   & $-0.30^{\cdot}$ & $-0.36^{*}$     \\
                & $(0.16)$      & $(0.16)$      & $(0.16)$        & $(0.15)$        \\
water\_gc       & $0.03^{***}$  & $0.02^{***}$  & $0.01^{***}$    & $0.01^{*}$      \\
                & $(0.00)$      & $(0.00)$      & $(0.00)$        & $(0.00)$        \\
barren\_gc      & $-0.02^{***}$ & $-0.02^{***}$ & $-0.01^{***}$   & $-0.01^{***}$   \\
                & $(0.00)$      & $(0.00)$      & $(0.00)$        & $(0.00)$        \\
distcoast       & $-0.00^{**}$  & $-0.00^{*}$   & $-0.00^{*}$     & $-0.00$         \\
                & $(0.00)$      & $(0.00)$      & $(0.00)$        & $(0.00)$        \\
region3         &               & $0.76^{***}$  & $0.47^{***}$    & $0.62^{***}$    \\
                &               & $(0.13)$      & $(0.12)$        & $(0.12)$        \\
logPopd         &               &               & $1.05^{***}$    & $1.02^{***}$    \\
                &               &               & $(0.06)$        & $(0.06)$        \\
bdist3          &               &               &                 & $-0.00^{***}$   \\
                &               &               &                 & $(0.00)$        \\
\hline
AIC             & $20459.50$    & $20423.98$    & $20064.82$      & $20038.51$      \\
BIC             & $20509.61$    & $20481.25$    & $20129.25$      & $20110.09$      \\
Log Likelihood  & $-10222.75$   & $-10203.99$   & $-10023.41$     & $-10009.25$     \\
Deviance        & $4081.51$     & $4090.69$     & $4149.68$       & $4154.02$       \\
Num. obs.       & $9492$        & $9492$        & $9492$          & $9492$          \\
\hline
\multicolumn{5}{l}{\scriptsize{$^{***}p<0.001$; $^{**}p<0.01$; $^{*}p<0.05$; $^{\cdot}p<0.1$}}
\end{tabular}
}
\caption{State based conflict events}
\label{state_based}
\end{center}
\end{sidewaystable}


\begin{sidewaystable}
\begin{center}
\scalebox{1}{
\begin{tabular}{l c c c c}
\toprule
 & Geography & North Africa & Population densisty & Distance
		  to border \\
\midrule
Precolonial state presence (sqrt)        & $-0.25^{***}$ & $-0.30^{***}$ & $-0.36^{***}$ & $-0.33^{***}$ \\
                                         & $(0.07)$      & $(0.07)$      & $(0.06)$      & $(0.06)$      \\
Precolonial state presence (sqrt)        & $-0.25^{***}$ & $-0.30^{***}$ & $-0.36^{***}$ & $-0.33^{***}$ \\
                                         & $(0.07)$      & $(0.07)$      & $(0.06)$      & $(0.06)$      \\
Mountainous terrain                      & $0.88^{***}$  & $0.87^{***}$  & $0.01$        & $-0.05$       \\
                                         & $(0.15)$      & $(0.15)$      & $(0.15)$      & $(0.15)$      \\
Water (\%)                               & $-0.01$       & $-0.00$       & $-0.01^{**}$  & $-0.01^{***}$ \\
                                         & $(0.00)$      & $(0.00)$      & $(0.00)$      & $(0.00)$      \\
Barren (\%)                              & $-0.02^{***}$ & $-0.02^{***}$ & $-0.01^{***}$ & $-0.01^{***}$ \\
                                         & $(0.00)$      & $(0.00)$      & $(0.00)$      & $(0.00)$      \\
Distance to coast (log)                  & $-0.13^{***}$ & $-0.09^{***}$ & $-0.12^{***}$ & $-0.09^{***}$ \\
                                         & $(0.02)$      & $(0.02)$      & $(0.02)$      & $(0.02)$      \\
Population density (log)                 &               &               & $1.00^{***}$  & $0.95^{***}$  \\
                                         &               &               & $(0.06)$      & $(0.06)$      \\
Distance to international boundary (log) &               &               &               & $-0.19^{***}$ \\
                                         &               &               &               & $(0.04)$      \\
North Africa                             &               & $0.39^{**}$   & $0.36^{**}$   & $0.44^{***}$  \\
                                         &               & $(0.13)$      & $(0.12)$      & $(0.12)$      \\
\midrule
AIC                                      & $20284.02$    & $20282.75$    & $19982.19$    & $19957.16$    \\
BIC                                      & $20348.44$    & $20354.33$    & $20060.93$    & $20043.06$    \\
Log Likelihood                           & $-10133.01$   & $-10131.37$   & $-9980.09$    & $-9966.58$    \\
Deviance                                 & $4117.11$     & $4118.67$     & $4168.50$     & $4172.13$     \\
Num. obs.                                & $9492$        & $9492$        & $9492$        & $9492$        \\
\bottomrule
\multicolumn{5}{l}{\scriptsize{$^{***}p<0.001$; $^{**}p<0.01$; $^{*}p<0.05$; $^{\cdot}p<0.1$}}
\end{tabular}
}
\caption{Conflict events *
		  Distance to capital}
\label{interaction_state_based}
\end{center}
\end{sidewaystable}

% latex table generated in R 4.1.2 by xtable 1.8-4 package
% Wed Jan 12 15:15:54 2022
\begin{table}[ht]
\centering
\begin{tabularx}{\textwidth}{l}
  \toprule
Atlas maps \\ 
  \midrule
\citet{Ajayi1985} \\ 
  \citet{Flint1976} \\ 
  \citet{Gailey1967} \\ 
  \citet{Kasule1998} \\ 
  \citet{mcevedy1996penguin} \\ 
  \citet{Oliver1985} \\ 
  \citet{Reid2012} \\ 
  The Times atlas of world history
			  (\citeyear{1978TTao}) \\ 
   \bottomrule
\end{tabularx}
\caption{List of maps from historical atlases
used in the Geo-ISD} 
\label{atlasmaps}
\end{table}


\begin{table}
\begin{center}
\scalebox{0.7}{
\begin{tabular}{l c c c}
\toprule
 & Model 1 & Model 2 & Model 3 \\
\midrule
Count model: (Intercept)          & $5.33^{***}$    & $3.43^{***}$  & $7.10^{***}$  \\
                                  & $(0.53)$        & $(0.55)$      & $(0.60)$      \\
Count model: sqrtSpAll            & $-0.33^{***}$   & $-0.39^{***}$ & $-0.36^{***}$ \\
                                  & $(0.07)$        & $(0.07)$      & $(0.07)$      \\
Count model: logCapdist           & $-0.28^{**}$    & $-0.38^{***}$ & $-0.46^{***}$ \\
                                  & $(0.09)$        & $(0.09)$      & $(0.09)$      \\
Count model: mountains\_mean      & $1.50^{***}$    & $0.13$        & $1.15^{***}$  \\
                                  & $(0.18)$        & $(0.17)$      & $(0.19)$      \\
Count model: region3              & $0.59^{***}$    &               & $0.18$        \\
                                  & $(0.13)$        &               & $(0.14)$      \\
Count model: water\_gc            & $0.01^{*}$      & $-0.01^{**}$  & $0.01^{*}$    \\
                                  & $(0.00)$        & $(0.00)$      & $(0.00)$      \\
Count model: logCDist             & $0.02$          & $-0.09^{***}$ & $0.07^{***}$  \\
                                  & $(0.02)$        & $(0.02)$      & $(0.01)$      \\
Count model: logPopd              & $0.83^{***}$    & $1.16^{***}$  & $0.56^{***}$  \\
                                  & $(0.07)$        & $(0.07)$      & $(0.07)$      \\
Count model: logBDist             & $-0.06$         & $-0.18^{***}$ & $-0.14^{**}$  \\
                                  & $(0.05)$        & $(0.04)$      & $(0.05)$      \\
Count model: sqrtSpAll:logCapdist & $0.06^{***}$    & $0.07^{***}$  & $0.06^{***}$  \\
                                  & $(0.01)$        & $(0.01)$      & $(0.01)$      \\
Count model: Log(theta)           & $-2.51^{***}$   & $-2.58^{***}$ & $-2.43^{***}$ \\
                                  & $(0.05)$        & $(0.03)$      & $(0.06)$      \\
Zero model: (Intercept)           & $0.39$          & $-11.90$      & $-0.54$       \\
                                  & $(0.96)$        & $(3235.61)$   & $(0.91)$      \\
Zero model: sqrtSpAll             & $-0.38^{\cdot}$ & $1.40$        & $-0.16$       \\
                                  & $(0.21)$        & $(136.83)$    & $(0.18)$      \\
Zero model: logCapdist            & $-0.07$         & $-73.25$      & $0.05$        \\
                                  & $(0.14)$        & $(701.24)$    & $(0.13)$      \\
Zero model: mountains\_mean       & $0.24$          & $27.63$       & $-0.01$       \\
                                  & $(0.23)$        & $(59.19)$     & $(0.23)$      \\
Zero model: region3               & $0.45^{**}$     &               & $0.35^{*}$    \\
                                  & $(0.15)$        &               & $(0.15)$      \\
Zero model: water\_gc             & $0.01^{*}$      & $-5.83$       & $0.01^{*}$    \\
                                  & $(0.01)$        & $(20.18)$     & $(0.01)$      \\
Zero model: logCDist              & $0.02$          & $-6.07$       & $0.02$        \\
                                  & $(0.02)$        & $(22.85)$     & $(0.02)$      \\
Zero model: logPopd               & $-2.81^{***}$   & $-20.02$      & $-2.50^{***}$ \\
                                  & $(0.28)$        & $(46.62)$     & $(0.28)$      \\
Zero model: logBDist              & $0.30^{***}$    & $1.12$        & $0.29^{***}$  \\
                                  & $(0.07)$        & $(10.80)$     & $(0.06)$      \\
Zero model: sqrtSpAll:logCapdist  & $0.04$          & $3.83$        & $0.01$        \\
                                  & $(0.03)$        & $(32.32)$     & $(0.03)$      \\
Count model: gbr                  &                 &               & $0.10$        \\
                                  &                 &               & $(0.15)$      \\
Count model: fra                  &                 &               & $-1.46^{***}$ \\
                                  &                 &               & $(0.14)$      \\
Zero model: gbr                   &                 &               & $0.62^{***}$  \\
                                  &                 &               & $(0.14)$      \\
Zero model: fra                   &                 &               & $0.20$        \\
                                  &                 &               & $(0.18)$      \\
\midrule
AIC                               & $42529.07$      & $19988.38$    & $42367.09$    \\
Log Likelihood                    & $-21243.53$     & $-9975.19$    & $-21158.55$   \\
Num. obs.                         & $9492$          & $9492$        & $9492$        \\
\bottomrule
\multicolumn{4}{l}{\scriptsize{$^{***}p<0.001$; $^{**}p<0.01$; $^{*}p<0.05$; $^{\cdot}p<0.1$}}
\end{tabular}
}
\caption{Statistical models}
\label{zinb}
\end{center}
\end{table}


\begin{figure}[htpb]
	\centering
	\includegraphics[width=\linewidth]{"../R/Output/SBzinbplot.pdf"}
	\caption{Predicted conflict events per state presence, grouped by
	distance to capital.}
	\label{state_int}
\end{figure}

\end{document}


