%{{{ Header
\documentclass[12pt]{article}

\usepackage{graphicx}
\usepackage{caption}
\usepackage[round]{natbib}
\usepackage{authblk}
\usepackage[utf8]{inputenc}
\usepackage{setspace}
\usepackage{rotating}
\usepackage[british]{datetime2}
\usepackage{hyperref}
\usepackage{tabularx}
\usepackage{xcolor}
\usepackage{booktabs}
\usepackage{multicol}
\usepackage{dcolumn}
\usepackage[top=28truemm,bottom=26.5truemm,left=25.5truemm,right=25.5truemm]{geometry}

\hypersetup{
colorlinks=true,
citecolor=black,
urlcolor=blue
}

\renewcommand{\harvardurl}{\textbf{URL:} \url}

\renewcommand\Affilfont{\itshape\small}

\title{After forever: Pre-colonial states and civil conflict}
\author[1]{Marius Swane Wishman}
\affil[1]{Department of Sociology and Political Science, NTNU}

\date{\today}

\providecommand{\keywords}[1]
{
	\small	
	\textbf{\textit{Keywords---}} #1
}

\begin{document}

\maketitle
%}}}

%{{{ Abstract
\begin{abstract}

This paper examines the relationship between the presence of pre-colonial states
and post cold war civil conflict. I argue that pre-colonial state presence can
be conflict inducing or reducing depending on the relationship between the
pre-colonial and post-independence states. To test this argument the paper
introduces the Geo-ISD data set, which maps the borders of 82 independent states
in Africa in the 1800-1914 period. I use these data to create a topographic
measure of state presence. Proxying the relationship between the pre-colonial
and post-independence state using the distance from the post-independence
capital, the article finds that higher levels of pre-colonial state presence are
conflict reducing in areas surrounding modern capital cities, which is
consistent with greater continuity of traditions and institutions associated
with statehood that are inherently conflict reducing. In areas further away from
the post-independence capital, higher levels of pre-colonial statehood are found
to be conflict inducing, consistent with the view that state legacies can
represent powerful symbols of past independence useful for mobilization, and
leave behind regional elite networks with the potential to violently resist
centralisation efforts of national governments. 

\end{abstract}

\keywords{Conflict, civil war, pre-colonial states, GIS, historical state
entities}

%tableofcontents
%\pagebreak

\onehalfspacing

%\begin{multicols}{2}

\newpage
%}}}

\section{Introduction} \label{Introduction}

Within the emerging literature on how independent pre-colonial African polities
have shaped post independence conflict, the findings appear contradictory. Some
find a conflict inducing effect attributed to differences between ethnic groups with
and without histories of statehood \citep{Englebert2002, Paine2019}. Others find
a conflict reducing effect from institutions generated by these states
\citep{Depetris-Chauvin2016, Wig2016}.

This paper addresses the apparent puzzle by arguing that the effect of
pre-colonial states on civil conflict, at the local level, is conditioned by the
relationship between the pre-colonial and post-independence states. On the one
hand, pre-colonial states can form the basis of the post-independence state. On
the other hand, pre-colonial states can form the basis of regional competitors
to the post-independence state. I argue that whether a pre-colonial state is
likely to look more like to the former or the latter, can be proxied by the
distance to the post-independence capital. Close to capitals, high levels of
pre-colonial state presence reflects a degree of continuation of rule, providing
legitimacy and institutions that are peace inducing. Conversely, low levels of
state presence near the capital reflect a relative lack of foundation for the
modern state. Outside of capital areas the story is reversed. In these areas
pre-colonial states leave behind powerful symbols of independence that can be
used for mobilization, as well as regional elite networks. As African states
gradually expand their reach into their own territories, they threaten the
autonomy of regional elites. The threat to autonomy triggers commitment
problems, historic roots and vertical networks provide capabilities, and when in
remote areas information is scares and bargaining breakdowns can occur. [Too
bargaining focused?]

Nigeria provides an illustrative example. Northern Nigeria is dominated by the
two former empires of the Sokoto Caliphate (and its twin state Adamawa) and
Borno (also referred to as Kanem-Borno), which form the basis of the
Muslim-North--Christian-South division of the country (see figure
\ref{nigeria}). The North has seen much state based violence in the post-cold
war period. Boko Haram, who has been responsible for the majority of this
violence, draw their inspiration and legitimacy from the jihad which led to the
establishment of the Sokoto Caliphate, and seek to implement this style of
caliphate in the current Borno state \citep{Pieri2016}. To the South, the
post-independence capital of Lagos\footnote{Lagos was the capital of Nigeria
until 1991, when the capital was moved to Abuja, specifically to be closer to
the North and serve as neutral ground in the deeply divided country
\citep{Moore_1984}.} lies within the borders of multiple pre-colonial Yoruba
states such as Oyo, Ibadan, Ife and others. In contrast to the North, the areas
surrounding Lagos have seen no state-based violence in the post-cold war period.
While it falls outside sample period, tensions between the North and the South
were also at the heart of the 1967-70 civil war.

To test the overall effect of pre-colonial state presence on local levels of
state-based violence, as well as how this relationship is moderated by distance
to the post-independence capital, I introduce new and innovative data on
pre-colonial statehood in Africa. The data leverages variations in how both
primary and secondary sources conceptualized and mapped statehood, to go beyond
the Cartesian model of mapping states, and provide a topographical measure of
historical state presence. Instead of a uniform measure of states' territories
with ``hard" boundaries, I propose one of gradually dissipating state presence
outside core areas and fuzzy borders. Additionally, the data covers more states
than comparative data sets, without compromising on the inclusion criteria for
statehood. Finally, it geocodes individual pre-colonial states, as opposed to
aggregating to current administrative levels or ties to settlement patterns of
related ethnic groups, both of which introduce issues of post treatment
bias.\footnote{For example, the Sokoto caliphate was the origin of the federal
state with the same name. The state was forcibly broken up by the government to
reduce its influence and the Sokoto sultans' resistance to the move influential
in toppling the current government \citep{HiribarrenVincent2017AHoB}. In Burkina
Faso the Mossi ethnic group is closely associated with the pre-colonial state of
Ouagadougou, but large areas of that state are in non-Mossi areas, and large
communities of Mossi live outside the areas controlled by that state
\citep{Wishman2021a}.}

The paper finds that there is a significant conflict inducing effect of
pre-colonial states. However, this effect is conditioned on the distance to
current capitals. In line with theoretical expectations, I find a substantial
conflict reducing effect at moderate levels of pre-colonial statehood near
post-independence capitals, from an initially high level of conflict in cases of
no, or very low levels of statehood, but only after an initial conflict onset. I
find that high levels of pre-colonial state presence are conflict inducing in
areas remote to post-independence capitals, across model specifications.

\section{The legacies of pre-colonial states}
\label{The legacies of pre-colonial states}

Despite the emerging literature on the impacts of traditional and pre-colonial
states and institutions, the nature of the relationship between such legacies
and conflict remains disputed. According to a game theoretical perspective like
that of \citet{Fearon1995}, pre-colonial institutions should be conflict
reducing. Groups who interact with the (modern) state through (traditional or
otherwise) institutions reduce the uncertainty of future behaviour relative to
groups who bargain through individuals, who are inherently more unpredictable
and prone to spoilers \citep{Wig2016}. Additionally, institutions are able to
make credible commitments by putting restraints on their leaders through
imposing violation costs. If a leader reneges on a commitment ratified by an
institution, it reduces the legitimacy of that institution and thus other laws
or decisions passed by it. There is also evidence that pre-colonial or
traditional institutions could be conflict reducing by improving local state
capacity, which could have a direct effect on the states ability to impose and
preserve order as well as an indirect effect through economic development
\citep{Depetris-Chauvin2016}, and better public goods provision
\citep{Wilfahrt_2021}. 

On the other hand, authors like \citet{Englebert2002} and \citet{Alesina2011}
have emphasized how countries can be negatively affected by multiple
pre-colonial states. \citet{Englebert2002} argues that colonial boundaries that
bundled together multiple ethnic groups with different historical experiences of
political organization, led to a `suffocation' effect. In such an environment,
post-independent states found it difficult to create a sense of nationality,
cohesion or solidarity among their populations, leading to higher levels of
conflict \citep{Englebert2002}. This argument ties in to the larger literature
on `artificial states', which argues that many states, in Africa in particular,
are artificial in the sense that their boundaries do not reflect the underlying
topography of statehood \citep{Alesina2011, Clapham1996, Jackson1991}. This
artificiality has been linked to lower levels of economic development,
presumably working in part through increased ethnic tensions and conflict
\citep{Alesina2011}. In this view, both having no pre-colonial states within a
country's boundaries, as well as having more than one, could be considered
artificial.\footnote{At least when these states were incorporated into the
current boundaries by external force (such as colonizers), as opposed to
`indigenously' (as for example the 100+ states of Germany being unified by
Prussia).} In cases where there are multiple groups with similar claims to
pre-colonial independence, this can make conflict a rational option for the
state.\footnote{Conflict is otherwise assumed to be the outcome of
miscalculations due to information problems in most game theory models.}
Choosing to accommodate one claims-making group in such an environment could
also lead to further claims by similar groups \citep{Walter2009}. This makes the
option of punishing any group who makes demands relatively cheaper, and thus
makes conflict a more likely outcome \citep{Wishman}.

Other studies have found a conflict-inducing effect related to ethnic relations.
Ethnic groups with a history of statecraft are likely to be have an over sized
share of power in government \citep{Wucherpfennig2016}. This can come about
through indirect colonial rule, which preferred to leaving existing power
structures intact, or by seizure from less politically experienced groups
following independence \citep{Paine2019}. This frequently puts
pre-colonial-state-groups in a trade off between including strong
rivals\footnote{`Strong rivals' refer to rival (ethnic) groups who would be
capable of punishing the ruling group for exclusion.} in government which means
a grater risk of coups, and excluding them from government and risking civil
conflict. Faced with this trade off rulers generally avoid the risk of coup
\citep{Paine2019, Powell_2014, Roessler_2011}.\footnote{Note that in some cases
	the ruler is forced to include the rival group, for example in cases of
split dominion, when the colonial power split the responsibility of the civil
and military administration between different ethnic groups \citep{Paine2019}.}
\citet{Paine2019} finds that pre-colonial state groups are more likely to
exclude other groups when in power (and thus increase the likelihood of civil
conflict), and coup their way into power if they themselves are excluded. This
also ties in with the literature on the conflict inducing impact of horizontal
inequalities \citep{CEDERMAN_2011}.

In addition to institutions, pre-colonial states can leave behind symbols of
sovereignty and elite networks \citep{Wishman}. Past independence has become an
important ingredient in most separatist struggles, and is used by conflict
entrepreneurs to overcome collective action problems\footnote{As exemplified by
Boko Haram's use of the Sokoto caliphate and Kanem Bornu referred to in section
\ref{Introduction}.}, as well as to provide a basis for ethnic claims making by
referring to past violations of sovereignty \citep{Ahram2019, Shelef2016}.
Vertical elite networks \footnote{The vertical orientation of these networks
stem from the vertical power structures typical of states.} are useful for
mobilization, and elites tend to have expectations of being included in
government, have substantial regional autonomy, or both \citep{Wishman}. Recent
work by \citet{Ying_2020} indicates that civil conflict tends to occur when the
state increases its presence in areas that it has hitherto not been present,
i.e. when it challenges the autonomy of regional elites. For example, in
Ethiopia, the Afar Liberation Front was originally formed by the sultan of the
former pre-colonial state of Awsa, when the Dirge regime tried to depose the
sultan \citep{Shehim1985, Hanfare2011}.  In Libya the Cyraneica Liberation Army
demonstrates both the symbolic mechanism as well as the elite networks, as its
name refers to a short lived kingdom in Eastern Libya, and the group elected a
descendent of the former king as their leader \citep{Ahram2019}. 

In summary, recent literature on pre-colonial statehood and conflict presents a
puzzle. Some find that more pre-colonial states, and more within-state diversity
of ethnic group experience of statehood lead to more conflict. Furthermore,
others that pre-colonial state groups are more likely to exclude other groups
even when doing so increases the risk of civil conflict. Yet, there is also
evidence of a conflict reducing effect. Traditional institutions can improve
local state capacity and public goods provision, as well as cause pre-colonial
state groups to be more successful in bargaining with the state. I address this
apparent puzzle by examining the conditionality of the different mechanisms
proposed in the literature.

\section{Theory}
\label{Theory} 

\subsection{Pre-colonial states} \label{Pre-colonial states}

Before going any further, I need to clarify the key concept of `pre-colonial
states'. For the purpose of this paper I follow the definition of `state' used
by the International Systems Data (ISD) v2 \citep{Butcher2020} as a political
entity with a population of at least 10,000, which has autonomy over a specific
territory and sovereignty that is either uncontested or acknowledged by relevant
international actors \citep{Butcher2020}\footnote{For a more in depth discussion
	of the definition of and criteria for statehood that the ISD is based
on, see \citet{Butcher2017}.}. By this definition the ISD v2 identifies 109
pre-colonial states in Africa during the 1800-1914 period, of which 82 are
included in the data used for this paper. This is a heterogeneous group of
political entities along most metrics. In size they range from small city states
like Harar (today part of Ethiopia), to empires like the Sokoto Caliphate. In
political organization they range from loose federations (for example Oyo
\citep{Law1977}) to relatively centralized kingdoms (Abyssinia/Ethiopia,
Buganda, or Zulu).

While smaller states might have been relatively mono-ethnic, larger states were
often multi-ethnic, although often politically dominated by one group. While the
geographic scope of the paper is Africa, pre-colonial states do include settler
states, as long as they were independent. In other words Liberia, the Boer
Republics and eventually South Africa are included. Based on the relationship
with regional powers, some states `come and go' as sovereign entities. Examples
of this include the North African states in their relationship with the Ottoman
empire, or Zinder (Sultanate of Damagaram), a city state on the periphery of the
Bornu empire, at times nominally subject, de facto subject, or fully independent
of Bornu.

While it is difficult to generalize about what this heterogeneous group of
states usually were, they were not modern states as we think of states today.
For one, there were (to my knowledge) no polices forces nor welfare states in
the way we think of them today.\footnote{Although some Islamic welfare systems
may have existed in Sokoto \citep{Buba_2018} and other Muslim states
\citep{WeissHolger2002SwiM}.} Bureaucracies and state apparatus, while at times
existing and relatively centralized, were rarely impersonal or large in size
\citep{Herbst2014}. Nor did any of them have international boundaries in the
sense that countries do today.\footnote{With the exception of some European settler
states toward the end of the 19th century, the Ethiopian empire and the partial
exception of the boundary between Sokoto and Borno.}

% Describe the frontiers and borders they did have?

In other words, states were relatively `shallow'. For most of the
population, for most of the time, the state was embodied by a local
representative (chief, bureaucrat, imam, lord etc.), often with some level of
judicial and tax responsibility and wide autonomy. Nominal subjugation and local
self rule was also wide spread. For example, the kingdom of Wadai/Bergoo (in
Chad) did not have a civil government, but had a royal council (fásher) and a
vertical network of political organization running down through regions, to
provinces to tribes and villages \citep{barth1857travels}. Tax (diván) rates
differed on the basis of prosperity of the area, individual political standing,
ethnic affiliation and religious holidays etc., but were generally more uniform
in the central provinces. In the surrounding provinces tribute was paid by the
province as a whole, reflecting the decreasing reach of the Wadai state.
Immediately outside that control, lay the neighboring kingdom of Baghirmi. Who
at least in some periods, paid tribute to Wadai while retaining its sovereignty
\citep{barth1857travels}.

Despite their relatively shallow state structures, there is ample evidence that
many pre-colonial states left marks that were felt long after they were
colonized, also in Africa. For example, pre-colonial states left behind
traditional political institutions \citep{Beall_2005, Holzinger_2020,
Neupert_Wentz_2021, Ubink_2008}. These institutions have at times acted as
mediators between ethnic groups and the central state \citep{boone2014property,
Englebert2002}, and have been an important sources of legitimacy for current
institutions \citep{Wig2016}\footnote{According to \citet{mamdani2018citizen}
	colonial authorities also felt the need to legitimize their rule through
	ties to pre-colonial institutions, at times went as far as to invent
pre-colonial roots.}. Nevertheless, some have argued that pre-colonial states
have represented competitors to the central state as well \citep{Herbst2014}.
There is also a growing literature on the role of pre-colonial states and
institutions in long term economic development \citep{Michalopoulos2018,
Acemoglu2014, Gennaioli2007, Bockstette2002, Wilfahrt_2021}.

\subsection{How state building and integration of pre-colonial states differ} 
\label{How state building and integration of pre-colonial states differ}

The post-independence state can most benefit from the conflict reducing effects
of pre-colonial states, when the pre-colonial state, at least in part, forms
basis of that state.\footnote{While
	this could could improve stability (as a product of legitimacy and state
capacity), it could prevent positive institutional influence from colonisers, in
the from of democracy \citep{Hariri2012, Woodberry2012}.} As it was
never colonised, Ethiopia would be the extreme case of this. There is a
continuation of institutions from the pre-colonial period into
modernity.\footnote{Although there is at least a minor break with the Shoa/Shewa
assent to power and moving the capital to Addis Ababa.} The North African states
and Madagascar,\footnote{In the ISD Madagascar contains two pre-colonial states,
only one of which is included in the Geo-ISD.} unlike Ethiopia, only contain one
pre-colonial state, which meant post-independence politicians were relatively
free to borrow from colonial or pre-colonial institutions as they saw fit. Most
countries in Africa though, contain more than one pre-colonial state.
Integrating multiple state, and ethnic, traditions and institutions in the post
independence state building process was not a simple task. Indeed, more
pre-colonial states, is associated with more post independence conflict.
Integration of pre-colonial states required political horse
trading, compromises and bargaining. Even where the post-independence capital is
a continuation of a pre-colonial capital, states were not free to directly
transfer its pre-colonial institutions to the country level, as it would provoke
other groups. This process created a system of mixed institutions with both
(potentially multiple) pre-colonial and colonial influences
\citep{englebert2013inside}. To preserve national unity, post-independence states
typically gave large concessions of regional autonomy et cetera initially, and
then gradually curtailed these over time.

Distance from the capital matters with regards to conflict for two reasons, both
functions of decreasing state capacity over distance. First, distance impedes
both the government's ability to project force \citep{Boulding1963, Buhaug_2010,
Buhaug2009, Herbst2014} and its ability to provide public goods. Even at medium
distance to the capital, most governments were (by 1989) able to project enough
force to deter all but the most substantial internal challengers. Relative
parity between the government and potential challengers matters because it
increases uncertainty of the outcome of an eventual conflict, and thus the
likelihood of miscalculations and bargaining breakdowns \citep{Boulding1963,
Buhaug_2010}. Furthermore the relative power parity increases the duration of
conflicts \citep{Buhaug2009}. Second, distance impedes the governments ability
to gather accurate information. Lack of accurate information further increases
the likelihood of miscalculations and bargaining breakdowns. 

\subsection{Close to the capital} \label{Close}

In line with the literature on artificial states, I argue that where there is
some pre-colonial state presence around the post independence capital, states
can build on pre-colonial state institutions and legitimacy.\footnote{At least
in the eyes of the group(s) with ties to that pre-colonial state.} Given that
most countries in Africa contain multi-pre-colonial states, I expect that
increased legitimacy is the primary pathway, because using the local
pre-colonial institutions is likely to antagonise other groups. Increased
legitimacy among group(s) tied to a local pre-colonial state in the capital area
matters because it is likely to represent a majority of the population there.
This means that people in and around the capital are generally more willing to
protect and preserve the state and its institutions, making it more robust to
violent challengers. Ties to a pre-colonial state matter further because of the
deep rooted elite networks they leave behind. Of course, there are elites in
areas without state presence as well, but they might not be equally deep rooted,
and as [uniformly] interested in preserving the state and its institutions[, or
play by the (institutional) rules]. Local pre-colonial experience with statehood
also provide relative improvements to state capacity, at least locally, further
shoring up both legitimacy and capacity to handle conflict
\citep{Depetris-Chauvin2016}.

On the other hand, where there is little to no pre-colonial state presence, the
state is necessarily built on colonial structures, which inherently lack
legitimacy. No groups feel a sense of ownership to the state, and the state
becomes instead an object of competition for individual or group based
[profit/gains/?][too speculative?]. Such competition exists in many countries,
but the key difference is the [universal] lack of legitimacy of the state itself
as an institution, which lowers the bar for
[extra-institutional/non-institutional] means. With the local population and
elites equally [apathetic (too strong, whats a better word?)], the capital, and
its immediate surroundings, risk becoming an arena for non-institutional,
potentially violent, competitions over the state. However, given that the state
usually has firm control over its capital, this would be rare events. Combat
death in the capital occur in one or two events, either as a result of military
defections by local garrisons, as in Bissau, Brazzaville, and
Juba,\footnote{Their respective pre-colonial state presence is 0, 59 and 5. For
comparison see summary statistics in Table \ref{summarystats}.} or from rebels
fighting their way to the capital, as in Bangui, Mogadishu, N'Djamena and
Kinshasa, often aided by government defections.\footnote{Their respective
	pre-colonial state presence is 4, 11, 117 and 60. For comparison see
summary statistics in Table \ref{summarystats}.} Both cases only occur in
countries where state capacity is very low, usually due to decades of
mismanagement and lack of national unity, potentially facilitated by the
artificiality of their foundation [too speculative?].

\subsection{Medium distance to capital} \label{Medium distance to capital}

At a medium distance to the post-independence capital states struggles to build
on pre-colonial state institutions and legitimacy. Because most likely other
groups, pre-colonial state or not, are the majority in the capital. These groups
would not view the medium-distance-pre-colonial state institutions as more
legitimate than others (colonial or more local pre-colonial state institutions),
unless they were previously part of that pre-colonial state. However, because of
the strategic advantages of modern states (in military terms), the relative
proximity means that the state can usually integrate pre-colonial state
institutions and groups, without violence. Conflict is only a plausible outcome
where elite networks and symbols of pre-colonial states are sufficiently strong
to mount a sustained challenge to the central state still.

\subsection{Far from the capital} \label{Far from the capital}

As stated above, civil conflict is more likely far from capitals in general.
However, I argue that high levels of pre-colonial state presence exacerbate this
risk for a number of reasons. First, local elite networks and symbols of
sovereignty means that \textit{ceteris paribus}, areas with high levels of
pre-colonial state presence are more likely to have the means to challenge the
state (closer to parity). As pointed out previously, there are elites in low/no
state presence areas as well. However, I argue they are less likely to have
dense social networks. Specifically, they are less likely to have the vertically
structured networks that are characteristic of states, and which are
particularly useful for mobilisation. Second, symbols of sovereignty makes it
more likely to turn any attempts of the government into a conflict over
territory and land. Several studies have found that conflict over such
indivisible spoils make bargaining more difficult and more likely to fail
[SOURCES]. Lastly, reducing the power of local elites invokes commitment
problems. Once the their power have been reduced, they are no longer able to
punish the state if it reneges on its end of the bargain. Knowing this makes
reaching an initial bargain more difficult. 

% {{{ table
%\begin{table}
%\begin{tabularx}{\textwidth} { c | X | X }
%	& Low distance to capital & High distance to capital \\ \hline
%	High state presence & {\color{blue} State capacity, artificial states,
%		credible commitments} & {\color{red} Artificial states, elite
%			networks, symbols}, {\color{blue} credible
%				commitments} \\ \hline
%	Low state presence & \color{red} Artificial states, state capacity &
%		\color{red} Artificial states\footnote{[I need to discuss why
%		this probably does not belong here.]}
%\end{tabularx}
%\caption{Conflict inducing predictions in {\color{red}red}, conflict reducing
%predictions in {\color{blue}blue}.}
%\label{twoXtwo}
%\end{table}
% }}}
	
Based on the theoretical discussion I propose the following hypotheses:

\bigskip
\hangindent = 3.5em \textit{H\textsubscript{1}: Low levels of pre-colonial
	state presence increase local levels of state-based violence in areas
	close to the national capital.}

\bigskip
\hangindent = 3.5em \textit{H\textsubscript{2}: Higher levels of pre-colonial
	state presence increase local levels of state-based violence in areas
	far from the post-independence capital.}

\section{Research design} \label{Research design}

\subsection{Dependent variable} \label{Dependent variable}

The units of analysis are PRIO 0.5 by 0.5 decimal degree grid cells with a
non-zero population density in 1600,\footnote{The year 1600 was chosen due to
	concerns of post-treatment bias, which will be explained further in
	section \ref{Controls}. This choice primarily excludes the Sahara and
Kalahari deserts.} which equals about 55 by 55km at the equator
\citep{Tollefsen2012}. Due to the explanatory variable being time-invariant, the
analysis is a cross section. The main dependent variable is intra-state and
internationalised internal conflict, conflict-related fatalities per grid cell
over the period 1989-2020, from the GED project \citep{Sundberg2013}. This is a
measure of the overall level of conflict in the post cold war period. The start
date of 1989 was dictated by availability rather than chosen by design, as it
likely biases against finding results (positive or negative), because the
mechanisms discussed in the theory section should have a more powerful effect
closer to independence and would fade in relative importance with the passing of
time.

As an alternative to combat related fatalities, I also ran models using the
count of intra-state and internationalised internal conflict events. This
captures much of the same general level of conflict during the period as the
fatalities measure does, but the focus is slightly different. Fatalities
captures the severity of conflict, whereas the number of conflict events
captures the frequency of conflict. This would be the difference between few, or
short, but highly lethal conflicts, versus lengthy conflict of relatively low
intensity, or recurring conflicts. I do not expect there to be a substantial
difference in the results from these measures based on the theory presented
above.

\subsection{Independent variables} \label{Independent variable}

\subsubsection{The Geo-ISD} \label{The Geo-ISD}

The availability of reliable data has been a persistent problem in the
literature on pre-colonial states. The \citet{Murdock1967} map of ethnic groups
and their corresponding `jurisdictional hierarchy' index of political
organization is one of the most frequently used data sources for constructing
per-ethnic-group measures of statehood. However, this data has a number of
issues (as enumerated by \citet{Michalopoulos2018}), such as lack of potential
overlap between ethnic groups, static borders, relatively short time span and a
lack of within-group variation, and misrepresentation in cases where state
control and ethnic settlement patterns do not align. An example of this is the
Mossi and the Mossi states in Burkina Faso, where about half of the area with
majority Mossi population lies outside the historical presence of the Mossi
states (primarily Ouagodogou), and about half the area of the state is outside
majority Mossi settlements. A further disadvantage of this approach is that by
using ethnic groups (and not states) as a starting point, there is a substantial
potential for missingness (as not all states are easily tied to a specific
ethnic group). For example, \citet{Paine2019}, despite using a `low bar' for
statehood and consulting numerous sources, only codes 28 groups in Sub-Saharan
Africa as having ties to a pre-colonial state.\footnote{This is partially also a
result of the criteria of `independence on the eve of colonization'.} Using a
similar approach \citet{Wig2016} identifies 45 state groups in the same region.
Using the State Antiquity Index \citep{Bockstette2002} as a starting point,
\citet{Depetris-Chauvin2016} avoids the limitation of only including states with
clear ties to ethnic groups. Nevertheless, his data only includes 54 states in
the 1800-1850+ period, as compared to 104 in the ISD version 2, despite using
less strict criteria for statehood. Additionally, aggregating experiences of
statehood to the country level is potentially problematic (as done by the State
Antiquity Index). The problem is twofold. First, experiences are unlikely to
accumulate uniformly to the country level. While some pre-colonial states do
contribute to national level institutions and statecraft, others do not, and
some have a potentially negative effect by putting pressure on national level
institutions. State Antiquities tries to address this with a weighting scheme,
but it only solves differences in how centralised states were, how long they
ruled and whether they were indigenous to the country. Second, aggregating to
current boundaries is inaccurate in cases where pre-colonial state presence
crosses current international boundaries. For example, it could result in giving
values of 0 to areas that potentially had high levels of state presence, but
whose presence was primarily in another current country.

This paper introduces the Geo-ISD, which geocodes the borders of independent
African states in the ISD v2 \citep{Butcher2020}. The ISD v2 picks up a large
number of states that are missed by similar data sets, while avoiding the use of
arbitrary criteria for statehood such as recognition by European powers
\citep{Butcher2020}. The Geo-ISD introduces an innovative method for going
beyond the Cartesian model of representing states as a flat, uniform sovereign
territory, by capturing the topographical historical presence of pre-colonial
states. In this way the Geo-ISD addresses some of the weaknesses of existing
geocoded data, namely the number of states identified, static borders across
time, lack of overlapping or contested rule, and implied uniformity in state
control across territory.

As its primary source, the Geo-ISD uses historically contemporary maps, sourced
from \href{https://www.davidrumsey.com}{the David Rumsey project} data base of
historical maps, matching the region of Africa in the period 1800 to 1914, to
code the geographical extent of pre-colonial states. The borders of pre-colonial
states were coded as follows.
% Maps were georeferenced in QGIS by connecting recognizable features in the
% maps (cities, distinctive capes, islands, etc) to their real locations when
% compared to satellite imagery containing exact location data. The result is a
% version of the map that is slightly distorted to better fit reality, as can be
% seen in Figure \ref{Arrowsmith}.\footnote{The exact specifications for the
% georeferencing and subsequent transformation will be supplied in the code
% book, included in the online supplemental material.} 
Shapes of the states included in the ISD v2 for the given year, that were
depicted in each map were traced in QGIS. For example, Bornou would be included,
but not the neighboring Howssa in Figure \ref{Arrowsmith}, as the shape drawn
could potentially refer to either the Haussa ethnic group (a common occurrence
in these maps) or the multiple Houssa states, neither of which qualify as states
in the ISD in that year. 

The start date was chosen due to the limitations of the ISD v2 which only extends
back to 1816 (but includes the founding date of states going further back), and
the fact that the quality of contemporaneous maps becomes substantially worse
prior to the nineteenth century \citep{Bassett_1994}. To control for some of the
potential biases of relying on maps drawn a long time ago, and by non indigenous
(mostly Western) mapmakers,\footnote{This issue is discussed further below.} the
same process was repeated using historical atlases compiled by later historians
(several of which were also consulted by \citet{Depetris-Chauvin2016} and
\citet{Paine2019}). The result was over 3400 polygons (state-shape-years)
covering the period 1800 to 1914 for continental Africa and Madagascar. 
% For some pre-colonial states in the ISD there were no maps for any years, some
% are covered only for some of the years they are in the ISD, but a number of
% them are covered by multiple maps for many years. Of the 109 states included
% for Africa in the ISD v2, the Geo-ISD includes at least one set of borders for
% 82 states. This is a considerable improvement on previous data sets, yet still
% not capturing the universe of independent African states in the period.

%\end{multicols}

\begin{figure}[h!tpb]
	\centering
	\includegraphics[width=0.8\linewidth]{img/Arrowsmith.jpg}
	\caption{Example of georeferenced map}%
	\label{Arrowsmith}
\end{figure}

%\begin{multicols}{2}

The accuracy of the historical maps used to create the polygons of the Geo-ISD
is a natural concern. Who typically drew these maps? Based on what sources? For
what purposes? And with what level of technical accuracy? 

Most of the maps from the David Rumsey project were from atlases published
for commercial purposes by individuals or small publishing companies
specializing in this type of publications. Most are from English or American
atlases, but French, Italian, German and other sources are included as well. The
maps were based on a combination of existing maps updated with `the latest
sources' (a fact frequently boasted in the title of the atlas), which for the
majority of the period meant explorers or geographers on missions from their
respective geographical societies.\footnote{Later maps additionally draw on the
	work of military surveyors, but as far as I have been able to tell the
majority are still based primarily on the work of explorers and past maps.} In
the words of \citet[47-48]{Stone1995}: `Cartography in Africa [in the 19th
century] is still a mix of measurement, less accurate observations, word of
mouth, previous maps and sources, educated guesses and pure conjecture.
Nevertheless a distinct improvement on the maps of previous periods.'. Because
of this, I expect two types of errors: errors resulting from measurement, and
bias resulting from misconceptions (deliberate or not) of what constituted the
borders of a polity at the time. I also expect errors to be replicated by other
maps, before eventually being corrected. One example of this is the nonexistent
Mountains of Kong, which can be seen in Figure \ref{Arrowsmith} as the mountain
range stretching across most of the continent from East to West. These mountains
were replicated for the better part of the nineteenth century before
eventually being wiped from the map \citep{Bassett_1991}.

While the quote above might lead one to expect considerable measurement error,
by my estimate this error only amounts to 36.9km on average for the shapes
included in the GeoISD. This estimate is based on the estimated mean distance of
the coastline in the maps to the real coastline, along the borders of the states
that were traced. This captures measurement error explicitly, because regardless
of where states did or did not extend their control, the coastline in the map
should line up with the real coastline. This means that there are often multiple
estimates for each map, reflecting how the accuracy is better in some places
than in others. If this difference was above 100km, the maps were deemed too
inaccurate and excluded from the sample. Error estimates for states that lay
inland were not included, because consistently matching features from the maps
to real geographical features was not feasible other than when using the
coastline, which provides a visible line of comparison. Additionally, this type
of error only adds noise, because there is no reason to suspect that the error
should be biased in any direction\footnote{It is just as likely to err toward
the South as it is to the North, or West etc.}. Thus it should not affect
coefficient estimates, but could potentially affect standard error estimates.

Although I have not been able to find sources discussing specifically how
cartographers determined the borders of different polities, it is probable that
they in large part relied on local verbal sources (word of mouth). An example of
this can be glimpsed in the exchange where the explorer Mungo Park effectively
dubbed the mountains of Kong.\footnote{`I gained the summit of a hill, from
	whence I had an extensive view of the country. Towards the south-east,
	appeared some very distant mountains, which I had formerly seen from an
	eminence near Marraboo, where the people informed me, that these
	mountains were situated in a large and powerful kingdom called Kong; the
	sovereign of which could raise a much greater army than the King of
Bambarra.' \citep[CHAPTER XVIII]{ParkMungo2015Titi}.} Alternatively, boundaries
were established when expeditions such as Park's were escorted by
representatives of the rulers of the various polities they passed through, until
reaching \textit{frontier towns}, where they would be met by representative of
the next ruler \citep{ParkMungo2015Titi}. In the maps resulting from such
encounters, both their sources as well as the cartographers themselves could
have introduced bias to the resulting maps. Sources were likely to be rulers or
their representatives, with incentives for aggrandisement. The explorers and
cartographers on their part represented European rulers with an eye toward
colonial expansion. It less clear how this would affect the resulting borders
drawn. One potential bias would be to exaggerate the domains of your own
governments prospective colonies, and vice versa. Another possibility is that
colonial cartographers under reported state sizes to declare `terra nullius'.
However, I did not observe any systematic differences between the maps based on
their nationality.\footnote{If drawing borders was driven by colonial ambitions,
there should be observable differences between the different colonial powers in
line with their differing colonial ambitions.} In fact, their colonial ambitions
could just as easily have promoted accuracy, as any potential military
expedition would benefit from accurate information \citep{Bassett_1994}.

\subsection{A topographical representation of the state} 
\label{A topographical representation of the state}

At the very least there should be heterogeneity in the conceptualization of
territoriality. What determines where a given source (or the cartographer in the
second instance) draws the borders of a polity, or how this would vary with
their respective conceptualization of states, polities and ethnic groups, is
impossible do determine. However, thanks to (usually) having multiple maps for
each state, the variation can be leveraged to create a measure of the
\textit{degree} to which a state had a presence in a given area over the time
period as a whole. When maps disagreed on where the various borders were, I
interpret this as either true variation across time, or as an indication of the
ambiguity of where a given state had nominal or real control. In the areas where
all the maps agree, one could be quite sure that the given polity had real
presence.  While in areas where only one map indicated that the state was
present, this could either be wrong, an indication of nominal as opposed to more
real presence or some other form of limited presence. The coding process of
looking at hundreds of maps strengthened this initial intuition, and the
resulting figures of state presence drawn from the complete data lends it
further credence. Figure \ref{nigeria} demonstrates how the estimate of
pre-colonial state presence captures a topographical measure. The Northern
states of Sokoto (North-West), Borno (North-East) and Adamawa (East) are clearly
distinguishable, while the Southern states (primarily the Benin kingdom,
neighbouring Dahomey and the multiple Yoruba states) have a weaker presence and
blend more together. The reason for the lower state presence in the South is in
part that these kingdoms were younger, or lasted shorter \footnote{some due to
being colonised, others due to conflict with the Sokoto Caliphate}, than the
Northern states, and that they conformed to fewer mapmakers' and historians'
conceptualisation of `state'. Furthermore, the figure demonstrates how the
presence of the Northern states gradually fades away from the center. In some
places it fades gradually, as the Sokoto Caliphate to the South. This presumably
reflects the temporary high water mark of its southward expansion, before being
repulsed by the Yoruba kingdom of Ibadan, as well as its less direct rule
further from the core. Other places state presence fades rapidly, as Borno to
the North, where its borders meet Lake Chad.

%\end{multicols}

\begin{figure}[htpb]
	\centering
	\includegraphics[width=\textwidth,keepaspectratio]{../R/Output/nigeria.pdf}
	\caption{Pre-colonial state presence in Nigeria (1800-1914).}
	\label{nigeria}
\end{figure}

%\begin{multicols}{2}

% To counter balance some of the potential bias in the historically
% contemporaneous maps a number of maps from historical atlases were included.
% These are works compiled by historians with both the accumulated knowledge of
% history and modern instruments of cartography at their disposal. These sources
% should be able to reduce problems of the historically contemporaneous maps, such
% as word of mouth sources exaggerating the extent of polities.\footnote{Although
% 	they should also be more accurate in terms measurement error, I found
% that this was not always the case. For example \citep{Kasule1998} completely
% misplaces Wadai, putting the capital of Wara (and thus the rest of the polity
% with it) at least 200km North West of its actual position.} The historical
% atlases were scanned, then georeferenced and polities traced in the same manner
% as the pre-colonial states in the historically contemporaneous maps. The
% historical atlases frequently depicted the borders of states over a period of
% years in a single map. In these cases the resulting state shapes were duplicated
% for each year in the period. This has the benefit of placing a larger emphasis
% on the historical atlases, at the cost of being more static.\footnote{Implicitly
% assuming constant borders, and in the absence of other sources implying uniform
% control across territory.} Figure \ref{atlasmaps} in the appendix lists the
% historical atlases used in the Geo-ISD.

The resulting data can be compiled in different ways, to provide different
insights. For the analysis in this paper I rely on a measure of `state
presence', similar in concept to that of `state history' introduced by
\citet{Depetris-Chauvin2016}.\footnote{The main difference between these
	measures is that `state history' is not a topographic measure, and
	usually only includes one shape per state, which implies static borders
	and uniform presence throughout the territory. It is also measured in 2
by 2 decimal degree grid cells, only includes Sub Saharan Africa, includes fewer
states despite going further back in time.} I measure `state presence' the as
number of state-shape-years that indicate that a state was present in a cell,
counting only those of the state most often present in that cell. If a cell
contains 40 shapes of the Sokoto Caliphate and 7 shapes of Borno, only the
Sokoto shapes are counted. Including the Borno shapes would risk over counting,
as it most likely does not represent additional state presence, but rather
overlapping or contested sovereignty.

%\end{multicols}

\begin{figure}[htpb]
	\centering
	%\includegraphics[width=0.8\linewidth]{../Rplot_ln_sp_int.pdf}
	\includegraphics[width=\linewidth]{../R/Output/sqrtSpAll.pdf}
	\caption{State presence (sqrt transformed) with interpolated years based
	on historical atlases.}
	\label{Sp_i}
\end{figure}

%\begin{multicols}{2}

Because I do not expect the relationship between pre-colonial state presence and
civil conflict to be linear, and because the data is heavily skewed, the
variable is square root transformed. While log transformation is more often
employed in the previous literature, the variable contains zeros. This is
usually solved by adding a constant to all values. However, this could
potentially introduce bias, and complicates the interpretation of the results
\citep{Ekwaru_2018}. I therefore present the more conservative approach of using
a square root transformation for the main models.\footnote{In addition to the
	square root transformed version of the main independent variable, I also
	ran models using the more common log transformation. The results
remained substantially the same, but with larger effect sizes.} This has the
benefit of not having to add a constant, but produces a less evenly distributed
variable. 

In summary, the main explanatory variable is an interaction between pre-colonial
state presence (square root transformed) and distance to capital (log
transformed). The theoretical expectation is that pre-colonial state presence is
conflict reducing in areas close to the post independence capital, and conflict
inducing in more remote areas. In other words the effect of pre-colonial state
presence is moderated by distance to the capital. Given how limited most
pre-colonial states were in size\footnote{Due to limited technologies of state,
the late introduction of reliable guns and epizootics rendering travel on foot
the only option for large parts of the continent.}, I expect positive effects of
pre-colonial state presence close to the capital to drop sharply. Consequently,
distance to the capital is log transformed anticipating a non-linear effect.
Because the variable is measured by distance from the center of each grid cell
to a point indicating the capital, there are no zeros and log transformation can
be done without adding a constant. Distance to capital is sourced from the
PRIO-grid data set, but originally from \citet{Weidmann2010a}.

As a further robustness check I also ran models using a measure of state
presence that sums if there were any maps that included a state in a
grid-cell-year (as a sum of yearly dummies). In other words, this could be at
most 214 (one for each year in the sample period), and is more a measure of the
maximum \textit{extent} of state presence, and is less accurate in terms of
variations in depth. The benefit of this measure is that it avoids some of the
potential for over representation of countries frequently mapped by Europeans,
such as the North African states (due to proximity). As with the main measure of
state presence, only the shapes of the state that was most present in that grid
cell throughout the sample period were traced. Results remain substantially the
same for all specifications.

\subsection{Controls} \label{Controls}

The treatment variable predates the outcome by a long
period of time and there is a substantial risk of introducing post treatment bias
when including control variables. In choosing which control variables to
include, I balanced a trade off between potential post treatment bias
and omitted variable bias. 

Mountains facilitate early state formation by providing protection and limiting the
exit options of sedentary farmers \citep{Carneiro1988}. Mountainous terrain has
also been linked with civil conflict by providing shelter for rebel groups
\citep{Hegre2006}, although this relationship is debated \citep{Buhaug2002}. The
data is from the PRIO-grid data set, but originally from \citet{Blyth2002}, and
measures the percentage of the cell with mountainous terrain based on elevation,
slope and local elevation range.

Water is essential for state formation. States typically formed either around
coastal cities, close to navigable rivers or by the shores of great lakes.
People still tend to live next to a source of water, and proximity to water acts
as a proxy for population density, and fighting usually happens where there are
(at least some) people. The data on water as a percentage of the grid surface is
from the PRIO-grid data set, but originally from the European Space Agency
\citep{Bontemps2009}. Because there are very few only-water tiles (Lake Victoria
contains some exceptions), high levels primarily represent cells on the coast,
on the shores of great lakes or in a few cases the banks of great rivers.

Distance to the coast could affect both state presence and conflict in a number
of ways. First, as stated above, states were more likely form along the coast as
it connected cities and people. A special case for Africa is also the existence
of slave raiding/trading states that formed along the eastern and western coasts
of the continent. These state's raison d'être was raiding slaves from tribes and
peoples inland and selling them to coastal traders (European in the West and
Arab in the east). \citet{Nunn2008} argues that this left legacies of mistrust
and antagonism, which has resulted in increased levels of current day conflict.
Distance to the coast could also be related to the measure of state presence
through the fact that our measure is based on European observations (maps),
which had better coverage along the coast, especially for the earlier periods.
Distance to the coast could further be related to conflict through lower levels
of economic development inland. The distance to coast data is from
\citet{Wessel1996}. The variable was log transformed to account for a non-linear
relationship.

As with water, barren terrain could be a (negative) pre-condition for state
building as well as proxy for later population densities, and thus could
correlate with both state presence and levels of conflict. The data is from
\citet{Bontemps2009}.

The states of North Africa are overrepresented in the Geo-ISD data, due to the
geographical proximity, and the accompanying historical familiarity to European
map makers. This affects Morocco most particularly, as can be seen in Figure
\ref{Sp_i}. The reason this affects Morocco in particular is that the remaining
North African states were under Ottoman suzerainty for much of the period. If
North Africa is also more or less conflict prone than the rest of Africa on
average, the inflated values of state presence would bias the estimated
coefficients. Accordingly, I included a dummy variable for the region of North
Africa.

Population density is added due to the theoretical expectation that it could be
a confounding variable. As discussed above, population density predicts both
state presence (states need a certain level of population density to form, and
survive \citep{scott2017against}), and conflict (conflict happens where there
are people). There are few accurate measures of population densities that
predate most of the states in the Geo-ISD, to avoid post treatment bias. The
best available estimates come from the HYDE project \citep{Goldewijk2016}, and I
use the estimates from 1600.

Distance to international boundaries could be related to state presence because,
despite their reputation, African borders were not drawn completely at random
(or along meridian lines). For example, the boundary between northern Nigeria and
Niger were based on the extent of the Sokoto Caliphate and the neighboring
Kanem-Bornu (or just Bornu) empire \citep{HiribarrenVincent2017AHoB}. Proximity
to an international boundary has also been found to predict conflict
\citep{Buhaug2002}. I use the measure included in the PRIO grid data, which is
originally from \citet{Weidmann2010a}.


\subsection{Modelling} \label{Modelling}

To account for the potential post treatment bias (vis-a-vis potential omitted
variable bias) controls were added step wise with increasing potential for post
treatment bias. The baseline model only includes geographic variables, which
should safely pre-date any states and thus be free of post-treatment bias.
Subsequent models add controls according to likelihood of introducing post
treatment bias.

To account for the dependent variable being count data (count of deaths
(fatalities) and count of conflict events (state based)) all model
specifications reported below are negative binomial regressions or zero inflated
negative binomial regressions. A fitness test for whether negative binomial or
Poisson regression is most appropriate, was conducted and confirmed that
negative binomial produced a better fit than Poisson.

The dependent variable contains excess zeros (8937 zeros relative to 100 counts
of 1 fatality, the second most frequent outcome). Additionally, the main
independent variable might affect the likelihood of any fatalities in a cell (or
if it remains a zero), differently to how it might influence the severity of
conflict once a cell has seen at least one fatality. I therefore include a zero
inflated negative binomial model. The first step of this two step approach is a
logit that models the likelihood that a cell is zero, in other words, that it
experiences no conflict. I used the same set of controls as I do not expect any
of them to exclusively affect conflict severity, nor do I expect their relation
to the main independent variable to be substantially different for an onset
model. The second step is a negative binomial estimation of conflict severity,
or the number of fatalities/events in a cell that has seen at least one
fatality/event.

Given what is known about spatial diffusion of conflict, there is reason to
suspect some spatial autocorrelation. However, controlling for this would
introduce a source of post treatment bias. Nevertheless, as an additional
robustness check I ran models with queen pattern spatial lags. While the main
model using fatalities as the depending variable does not converge, the main
model for conflict events remains largely unchanged.

[Territorial disputes far from the capital and government disputes close to
capital?]

\section{Results} \label{Results}

Despite the data on conflict starting nearly three decades after most of Africa
achieved independence (at which point the effects of pre-colonial state presence
on conflict should be most pronounced), there is a considerable conflict
inducing effect of high state presence far from the capital (see Table
\ref{interaction_interdeaths}. While the effect is only significant at the 10\%
level, the effect size is considerable (see Figure \ref{interdeaths}, supporting
H\textsubscript{2}. Far from the capital even at the mean value of pre-colonial
state presence the model predicts over 1000 fatalities. While there is a
significant negative effect of pre-colonial state presence in the first two
models, supporting H\textsubscript{1}, this effect becomes insignificant after
controlling for population density. Distance to capital (log-transformed) is not
significant in any of the models. Surprisingly the distance to coast
(log-transformed) is significant and negative, even after controlling for
population density. The rest of the controls are either insignificant or as
expected.

As discussed in Section \ref{Research design}, the data on fatalities contain
excess zeros, and a zero inflation model is probably a more correct
specification. Figure \ref{deaths_zinb}, models the predictions of the second
stage ZINB model included in Table \ref{zinb}, which predicts the number of
fatalities in grid cells with at least one fatality. Taken together with the
negative binomial interaction (in Figure \ref{interdeaths}), it appears that
little-to-no pre-colonial state presence in the capital does not, by itself lead
to more fatalities in that area. However, if conflict does erupt in the capital
states with some level of pre-colonial state presence appear far better at
quelling the violence. For capitals without pre-colonial state presence the
model predicts close to 20 000 fatalities, and an upper bound of over 58 000, as
compared to just under 2300 with a mean level of pre-colonial state presence.
[should I hint at state failure here? Currently saving it for the conclusion]
Although the effect size is less dramatic, the results for H\textsubscript{2}
still hold in this model as well. For a mean level of pre-colonial state
presence, the model predicts over 3600 fatalities in grid cells far form the
capital. Control variables are as expected for both the count and zero model. 

%\end{multicols}

\begin{figure}[htpb] \centering
	\includegraphics[width=\linewidth]{"../R/Output/deathsInterPlot.pdf"}
	\caption{}
	\label{interdeaths}
\end{figure}

\begin{figure}[htpb]
	\centering
	\includegraphics[width=\linewidth]{"../R/Output/interdeathszinbplot.pdf"}
	\caption{}
	\label{deaths_zinb}
\end{figure}

%\begin{multicols}{2}

\subsection{Alternative explanations} \label{Alternative explanations}

An alternative interpretation of the results could be that this is a
story of more coherent ethnic groups being more likely to be associated with
states, and more likely to (perhaps better able to) challenge the government
when situated far from the capital. However, getting closer to the real
causality requires untangling, for each case, if a given conflict is related to
an ethnic group's ties to a pre-colonial state or not. This lies outside the
scope of this paper, if it is even possible. [Icing: control for EPR excluded
groups]

Another interpretation that is not controlled for in this paper is the
possibility that past conflict drives both state creation and current conflict.
However, data on past conflict is meager, especially so for Africa. To my
knowledge the \citet{Brecke1999} data set is the most complete. Even so, it
relies on written histories of which there is little for pre-colonial
sub-Saharan Africa. What is worse, the missing will be considerably biased
because kings and states are far more likely to chronicle their warfare in the
form of written records. [Icing: use Besley and Reynal-Querol data]

\section{Conclusion} \label{Conclusion}

Drawing on the emerging literature on pre-colonial states \citep{Paine2019,
Depetris-Chauvin2016}, institutions \citep{Wig2016, Englebert2002,
Michalopoulos2018} and civil conflict, and on newly compiled data, this paper
re-examined the relationship between pre-colonial states and civil conflict. I
find support for both hypotheses. Overall, effect sizes are substantial,
indicating that levels of pre-colonial state presence has had a considerable
impact on post cold war levels of civil conflict in Africa. While armed conflict
in capital areas is always a serious threat to the state, the results in Table
\ref{zinb} and Figure \ref{deaths_zinb}, painting a far more dire picture in
areas with little or no pre-colonial state presence, perhaps reflecting state
collapse. Pre-colonial state presence can be conflict reducing, as suggested by
some of the existing literature, but only by providing states with a local
foundation for state building, which I argue made capital areas more robust to
civil conflict. On the other hand, in areas far from the capital that have high
levels of pre-colonial state presence, the central state is not able to build on
these existing institutions and source of legitimacy. Instead, states struggle to
integrate these areas without provoking a violent response. The results are
robust to alternative measurements and model specifications.

An interesting case for the next decade is the relationship between Puntland and
Somalia. Puntland corresponds to the borders of the Majarteen sultanate and thus
is the only part of Somalia with Somali pre-colonial state presence (other parts
have presence from Ethiopia and the Zanzibar sultanate). Following the break
down of the centraal Somali state, Majarteen (clan) leaders declared Puntland a
autonomous federal state, over fears of the status of the Majarteen clan within
post conflict Somalia \citep{Johnson_2014}. The region has fared relative well
compared to the rest of the country. While its leaders have cultivated continued
ties to the central government and aspirations of national leadership (as in the
post-independence period), they have also defended its autonomy. For example, the
local government has successfully resisted central state control over oil
prospecting in the northeast \citep{Johnson_2014}. As the central state in
Mogadishu grows more stable, so will call for regional reintegration. The results
presented in this paper suggest that such efforts might lead to yet more
violence in Somalia. Given that Puntland reportedly spend around 90\% of its
budget on security, police and recurrent administration \citep{Johnson_2014},
its leaders are signalling their intention of maintaining autonomy.

These findings have a few general implications as well. First, they demonstrate that
pre-colonial states can be a blessing or a curse depending on whether they form
the basis of the modern state or a point of opposition to it. Second, that local
political histories matter, and should be taken into consideration by
policymakers and scholars alike. Finally, these results suggests that wherever
colonizers left novel constellations of pre-existing polities, they unknowingly
sowed the seeds of future conflict. More research using global data is needed to
test whether general trends for Africa hold outside that continent as well.
[Somewhere in this section fit in the prediction that there is potential for
violence in places where the central government has not yet extended its reach
into areas with high degree of pre-colonial state presence, most notably in
Puntland, Somalia.]

%\end{multicols}

\pagebreak

\bibliographystyle{apalike}
\bibliography{../lib.bib}

\pagebreak
\section*{Appendix}


% Table created by stargazer v.5.2.2 by Marek Hlavac, Harvard University. E-mail: hlavac at fas.harvard.edu
% Date and time: Tue, Dec 14, 2021 - 12:56:19 PM
\begin{table}[!htbp] \centering 
  \caption{Summary Statistics} 
  \label{summarystats} 
\begin{tabular}{@{\extracolsep{1pt}}lccccccc} 
\\[-1.8ex]\hline 
\hline \\[-1.8ex] 
Statistic & \multicolumn{1}{c}{N} & \multicolumn{1}{c}{Mean} & \multicolumn{1}{c}{St. Dev.} & \multicolumn{1}{c}{Min} & \multicolumn{1}{c}{Pctl(25)} & \multicolumn{1}{c}{Pctl(75)} & \multicolumn{1}{c}{Max} \\ 
\hline \\[-1.8ex] 
Fatalities & 10,652 & 45.8 & 883.1 & 0 & 0 & 0 & 79,920 \\ 
Conflict events & 10,652 & 2.4 & 27.0 & 0 & 0 & 0 & 1,940 \\ 
State presence & 10,652 & 50.2 & 88.1 & 0 & 2 & 62 & 629 \\ 
Distance
		    to boundary & 10,652 & 138.1 & 122.6 & 0.003 & 39.6 & 200.7 & 668.0 \\ 
Distance to capital & 10,652 & 671.3 & 411.0 & 3.7 & 338.0 & 956.2 & 2,482.5 \\ 
Barren & 10,652 & 32.7 & 44.4 & 0 & 0 & 97.9 & 100 \\ 
Mountainous & 10,492 & 0.1 & 0.3 & 0.0 & 0.0 & 0.1 & 1.0 \\ 
Water & 10,652 & 4.8 & 17.7 & 0.0 & 0.0 & 0.1 & 100.0 \\ 
Distance to coast & 10,652 & 599,064.4 & 460,732.6 & 0 & 185,784.8 & 956,404.4 & 1,761,700 \\ 
popd & 10,559 & 2.5 & 9.9 & 0.0 & 0.1 & 2.1 & 447.9 \\ 
\hline \\[-1.8ex] 
\end{tabular} 
\end{table} 


\begin{sidewaystable}
\begin{center}
\scalebox{1}{
\begin{tabular}{l c c c c}
\toprule
 & Geography & North Africa & Population densisty & Distance
		  to border \\
\midrule
(Intercept)                              & $7.09^{***}$  & $7.70^{***}$  & $5.97^{***}$   & $6.00^{***}$   \\
                                         & $(0.96)$      & $(0.96)$      & $(0.97)$       & $(0.99)$       \\
Mountainous terrain                      & $1.58^{***}$  & $1.81^{***}$  & $1.43^{***}$   & $1.44^{***}$   \\
                                         & $(0.29)$      & $(0.28)$      & $(0.28)$       & $(0.28)$       \\
Water (\%)                               & $-0.00$       & $-0.00$       & $0.01$         & $0.01$         \\
                                         & $(0.01)$      & $(0.01)$      & $(0.01)$       & $(0.01)$       \\
Barren (\%)                              & $-0.03^{***}$ & $-0.03^{***}$ & $-0.01^{***}$  & $-0.01^{***}$  \\
                                         & $(0.00)$      & $(0.00)$      & $(0.00)$       & $(0.00)$       \\
Distance to coast (log)                  & $-0.07^{*}$   & $-0.06^{*}$   & $-0.09^{**}$   & $-0.09^{**}$   \\
                                         & $(0.03)$      & $(0.03)$      & $(0.03)$       & $(0.03)$       \\
North Africa                             &               & $1.70^{***}$  & $-0.07$        & $-0.07$        \\
                                         &               & $(0.22)$      & $(0.22)$       & $(0.22)$       \\
Population density (log)                 &               &               & $0.87^{***}$   & $0.86^{***}$   \\
                                         &               &               & $(0.12)$       & $(0.12)$       \\
Distance to international boundary (log) &               &               &                & $-0.05$        \\
                                         &               &               &                & $(0.07)$       \\
Distance to capital (log)                & $0.05$        & $-0.10$       & $0.02$         & $0.04$         \\
                                         & $(0.15)$      & $(0.14)$      & $(0.14)$       & $(0.14)$       \\
Precolonial state presence (sqrt)        & $-0.26^{*}$   & $-0.30^{*}$   & $-0.11$        & $-0.09$        \\
                                         & $(0.12)$      & $(0.12)$      & $(0.12)$       & $(0.12)$       \\
Interaction term                         & $0.06^{**}$   & $0.06^{**}$   & $0.04^{\cdot}$ & $0.04^{\cdot}$ \\
                                         & $(0.02)$      & $(0.02)$      & $(0.02)$       & $(0.02)$       \\
\midrule
AIC                                      & $38058.14$    & $38025.54$    & $37976.24$     & $37979.70$     \\
BIC                                      & $38122.57$    & $38097.12$    & $38054.98$     & $38065.60$     \\
Log Likelihood                           & $-19020.07$   & $-19002.77$   & $-18977.12$    & $-18977.85$    \\
Deviance                                 & $3440.40$     & $3442.13$     & $3444.65$      & $3444.58$      \\
Num. obs.                                & $9492$        & $9492$        & $9492$         & $9492$         \\
\bottomrule
\multicolumn{5}{l}{\scriptsize{$^{***}p<0.001$; $^{**}p<0.01$; $^{*}p<0.05$; $^{\cdot}p<0.1$}}
\end{tabular}
}
\caption{Fatalities}
\label{interaction_interdeaths}
\end{center}
\end{sidewaystable}


\begin{sidewaystable}
\begin{center}
\scalebox{1}{
\begin{tabular}{l c c c c}
\toprule
 & Geography & North Africa & Population densisty & Distance
		  to border \\
\midrule
(Intercept)          & $2.28^{\cdot}$ & $2.33^{\cdot}$ & $0.24$        & $0.17$        \\
                     & $(1.20)$       & $(1.20)$       & $(1.21)$      & $(1.23)$      \\
sqrtSpAll            & $0.03$         & $-0.06$        & $-0.13$       & $-0.14$       \\
                     & $(0.15)$       & $(0.15)$       & $(0.15)$      & $(0.15)$      \\
logCapdist           & $0.23$         & $0.17$         & $0.39^{*}$    & $0.38^{*}$    \\
                     & $(0.18)$       & $(0.18)$       & $(0.18)$      & $(0.18)$      \\
mountains\_mean      & $0.75^{*}$     & $0.75^{*}$     & $0.57$        & $0.57$        \\
                     & $(0.35)$       & $(0.35)$       & $(0.35)$      & $(0.35)$      \\
water\_gc            & $-0.04^{***}$  & $-0.04^{***}$  & $-0.04^{***}$ & $-0.04^{***}$ \\
                     & $(0.01)$       & $(0.01)$       & $(0.01)$      & $(0.01)$      \\
barren\_gc           & $-0.01^{*}$    & $-0.01^{**}$   & $-0.00$       & $-0.00$       \\
                     & $(0.00)$       & $(0.00)$       & $(0.00)$      & $(0.00)$      \\
logCDist             & $-0.23^{***}$  & $-0.20^{***}$  & $-0.20^{***}$ & $-0.21^{***}$ \\
                     & $(0.04)$       & $(0.04)$       & $(0.04)$      & $(0.04)$      \\
sqrtSpAll:logCapdist & $0.01$         & $0.02$         & $0.03$        & $0.03$        \\
                     & $(0.02)$       & $(0.02)$       & $(0.03)$      & $(0.03)$      \\
region3              &                & $0.66^{*}$     & $0.32$        & $0.30$        \\
                     &                & $(0.28)$       & $(0.28)$      & $(0.28)$      \\
logPopd              &                &                & $0.73^{***}$  & $0.73^{***}$  \\
                     &                &                & $(0.15)$      & $(0.15)$      \\
logBDist             &                &                &               & $0.05$        \\
                     &                &                &               & $(0.09)$      \\
\midrule
AIC                  & $11446.85$     & $11441.68$     & $11428.40$    & $11430.13$    \\
BIC                  & $11511.27$     & $11513.26$     & $11507.14$    & $11516.03$    \\
Log Likelihood       & $-5714.43$     & $-5710.84$     & $-5703.20$    & $-5703.06$    \\
Deviance             & $1435.73$      & $1436.96$      & $1439.37$     & $1439.42$     \\
Num. obs.            & $9492$         & $9492$         & $9492$        & $9492$        \\
\bottomrule
\multicolumn{5}{l}{\scriptsize{$^{***}p<0.001$; $^{**}p<0.01$; $^{*}p<0.05$; $^{\cdot}p<0.1$}}
\end{tabular}
}
\caption{Conflict events *
		  Distance to capital}
\label{interaction_both}
\end{center}
\end{sidewaystable}

% latex table generated in R 4.1.2 by xtable 1.8-4 package
% Wed Jan 12 15:15:54 2022
\begin{table}[ht]
\centering
\begin{tabularx}{\textwidth}{l}
  \toprule
Atlas maps \\ 
  \midrule
\citet{Ajayi1985} \\ 
  \citet{Flint1976} \\ 
  \citet{Gailey1967} \\ 
  \citet{Kasule1998} \\ 
  \citet{mcevedy1996penguin} \\ 
  \citet{Oliver1985} \\ 
  \citet{Reid2012} \\ 
  The Times atlas of world history
			  (\citeyear{1978TTao}) \\ 
   \bottomrule
\end{tabularx}
\caption{List of maps from historical atlases
used in the Geo-ISD} 
\label{atlasmaps}
\end{table}


\begin{table}
\begin{center}
\scalebox{0.7}{
\begin{tabular}{l c c c}
\toprule
 & Model 1 & Model 2 & Model 3 \\
\midrule
Count model: (Intercept)          & $5.33^{***}$    & $3.43^{***}$  & $7.10^{***}$  \\
                                  & $(0.53)$        & $(0.55)$      & $(0.60)$      \\
Count model: sqrtSpAll            & $-0.33^{***}$   & $-0.39^{***}$ & $-0.36^{***}$ \\
                                  & $(0.07)$        & $(0.07)$      & $(0.07)$      \\
Count model: logCapdist           & $-0.28^{**}$    & $-0.38^{***}$ & $-0.46^{***}$ \\
                                  & $(0.09)$        & $(0.09)$      & $(0.09)$      \\
Count model: mountains\_mean      & $1.50^{***}$    & $0.13$        & $1.15^{***}$  \\
                                  & $(0.18)$        & $(0.17)$      & $(0.19)$      \\
Count model: region3              & $0.59^{***}$    &               & $0.18$        \\
                                  & $(0.13)$        &               & $(0.14)$      \\
Count model: water\_gc            & $0.01^{*}$      & $-0.01^{**}$  & $0.01^{*}$    \\
                                  & $(0.00)$        & $(0.00)$      & $(0.00)$      \\
Count model: logCDist             & $0.02$          & $-0.09^{***}$ & $0.07^{***}$  \\
                                  & $(0.02)$        & $(0.02)$      & $(0.01)$      \\
Count model: logPopd              & $0.83^{***}$    & $1.16^{***}$  & $0.56^{***}$  \\
                                  & $(0.07)$        & $(0.07)$      & $(0.07)$      \\
Count model: logBDist             & $-0.06$         & $-0.18^{***}$ & $-0.14^{**}$  \\
                                  & $(0.05)$        & $(0.04)$      & $(0.05)$      \\
Count model: sqrtSpAll:logCapdist & $0.06^{***}$    & $0.07^{***}$  & $0.06^{***}$  \\
                                  & $(0.01)$        & $(0.01)$      & $(0.01)$      \\
Count model: Log(theta)           & $-2.51^{***}$   & $-2.58^{***}$ & $-2.43^{***}$ \\
                                  & $(0.05)$        & $(0.03)$      & $(0.06)$      \\
Zero model: (Intercept)           & $0.39$          & $-11.90$      & $-0.54$       \\
                                  & $(0.96)$        & $(3235.61)$   & $(0.91)$      \\
Zero model: sqrtSpAll             & $-0.38^{\cdot}$ & $1.40$        & $-0.16$       \\
                                  & $(0.21)$        & $(136.83)$    & $(0.18)$      \\
Zero model: logCapdist            & $-0.07$         & $-73.25$      & $0.05$        \\
                                  & $(0.14)$        & $(701.24)$    & $(0.13)$      \\
Zero model: mountains\_mean       & $0.24$          & $27.63$       & $-0.01$       \\
                                  & $(0.23)$        & $(59.19)$     & $(0.23)$      \\
Zero model: region3               & $0.45^{**}$     &               & $0.35^{*}$    \\
                                  & $(0.15)$        &               & $(0.15)$      \\
Zero model: water\_gc             & $0.01^{*}$      & $-5.83$       & $0.01^{*}$    \\
                                  & $(0.01)$        & $(20.18)$     & $(0.01)$      \\
Zero model: logCDist              & $0.02$          & $-6.07$       & $0.02$        \\
                                  & $(0.02)$        & $(22.85)$     & $(0.02)$      \\
Zero model: logPopd               & $-2.81^{***}$   & $-20.02$      & $-2.50^{***}$ \\
                                  & $(0.28)$        & $(46.62)$     & $(0.28)$      \\
Zero model: logBDist              & $0.30^{***}$    & $1.12$        & $0.29^{***}$  \\
                                  & $(0.07)$        & $(10.80)$     & $(0.06)$      \\
Zero model: sqrtSpAll:logCapdist  & $0.04$          & $3.83$        & $0.01$        \\
                                  & $(0.03)$        & $(32.32)$     & $(0.03)$      \\
Count model: gbr                  &                 &               & $0.10$        \\
                                  &                 &               & $(0.15)$      \\
Count model: fra                  &                 &               & $-1.46^{***}$ \\
                                  &                 &               & $(0.14)$      \\
Zero model: gbr                   &                 &               & $0.62^{***}$  \\
                                  &                 &               & $(0.14)$      \\
Zero model: fra                   &                 &               & $0.20$        \\
                                  &                 &               & $(0.18)$      \\
\midrule
AIC                               & $42529.07$      & $19988.38$    & $42367.09$    \\
Log Likelihood                    & $-21243.53$     & $-9975.19$    & $-21158.55$   \\
Num. obs.                         & $9492$          & $9492$        & $9492$        \\
\bottomrule
\multicolumn{4}{l}{\scriptsize{$^{***}p<0.001$; $^{**}p<0.01$; $^{*}p<0.05$; $^{\cdot}p<0.1$}}
\end{tabular}
}
\caption{Statistical models}
\label{zinb}
\end{center}
\end{table}


\begin{figure}[htpb]
	\centering
	\includegraphics[width=\linewidth]{"../R/Output/ggBothPlot.pdf"}
	\caption{Predicted conflict events per state presence, grouped by
	distance to capital.}
	\label{both_int}
\end{figure}

\begin{figure}[htpb]
	\centering
	\includegraphics[width=\linewidth]{"../R/Output/bothzinbplot.pdf"}
	\caption{Predicted conflict events per state presence, grouped by
	distance to capital, in cells with at least one conflict event.}
	\label{bothzinb_int}
\end{figure}

\begin{figure}[htpb]
	\centering
	\includegraphics[width=\linewidth]{"../R/Output/spatialzinbbothplot.pdf"}
	\caption{Predicted conflict events per state presence, grouped by
	distance to capital, in cells with at least one conflict event,
controlling for spatial diffusion.}
	\label{spatbothzinb_int}
\end{figure}

\end{document}


