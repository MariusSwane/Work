
\documentclass[12pt]{article}
\usepackage{graphicx}
\usepackage{caption}
\usepackage{natbib}
\usepackage{authblk}
\usepackage[utf8]{inputenc}
\usepackage{setspace}
\usepackage{rotating}
\usepackage[british]{datetime2}
\usepackage{url}

\renewcommand{\harvardurl}{\textbf{URL:} \url}
\renewcommand\Affilfont{\itshape\small}

\title{Discussion Questions and Comments}
\author[1]{Marius Swane Wishman}
\affil[1]{Department of Sociology and Political Science, NTNU}

\date{\today}

\providecommand{\keywords}[1]
{
	\small	
	\textbf{\textit{Keywords---}} #1
}

\begin{document}

\maketitle

\begin{abstract}
	For each class session, students are required to prepare a brief list of
	discussion questions and comments (3-5 in number); these should be based
	on the readings. Your questions/comments should reflect a critical
	assessment of those readings. What are their strong and weak points?
	Their methodological, empirical contributions? How do they relate to or
	build upon other readings or discussions?
\end{abstract}

%\keywords{}

\pagebreak

%tableofcontents
%\pagebreak

\onehalfspacing

\section*{Session I}

\begin{itemize}
	\item I would like to start the course with a discussion about
		\textit{dynamics}. This is fuzzy term, that needs clarification
		given its place in the course title.

	\item \citet{Blattman2010} does a great job of summarizing the state of
		the field, albeit from a very economics centered perspective.
		However, the article is over 10 years old. What has changed
		since its publication? Are we able to answer any of the
		questions they raise?

	\item Something that I have not seen yet in my reading of
		principal-agent/rational choice models of the collective action
		problem in armed groups is formal modelling of the fact that
		they are armed. The possession of deadly weapon means that any
		actor can at any time enforce demands on other actors or even
		the principle, as long as he catches the other party off guard.
		This  works in government troops too, but unless the actor
		committing treason is able too cease control of a majority or so
		of the military he will need to hide from the remaining
		forces.\footnote{Government defectors/traitors at the highest
			levels also play to the international community.
			Motivation will therefor vary with expectations of
			international condemnation or support, depending
		primarily of the incumbent regimes' international standing.}
		Rebels on the other hand are all ready hiding from government
		forces, hiding from other rebels would be less of an addition.
		Defections and treason should be more common amongst rebel
		groups.
	
\end{itemize}

\section*{Session II}

\begin{itemize}

	\item How do the other course participants use/define `mechanisms'? Are
		there differences between quals and quants? Differences between
		scholarly backgrounds?

	\item Pragmatism in pursuit of mechanisms. How do we, from a
		methodological/philosophy of science perspective, avoid becoming
		quantum physicists?  

	\item What \citet{Johnson_2006} seems not to be fully aware of is that
		positivists are acutely aware of the empiricist problem of
		causation (the fact that causation itself cannot be observed).
		That is precisely why science (including social science) needs
		rigorous methods; to get as close to observing it as possible.
		A lot of \citet{Johnson_2006}'s hesitation comes from the
		ontological differences between KKV and himself (the old
		Plato/Aristotle dichotomy), something he does not address
		directly outside of criticising the emphasis placed on empirics.
		I suspect a lot of the criticism of positivism also stems from
		reading it as `minimum requirements' instead of a (ultimately
		unachievable) goal to strive towards.

\end{itemize}

\section*{Session III}

\begin{itemize}

	\item I find that \citet{Epstein_2002}'s assumption that the legitimacy
		of the regime is exogenous to the grievances of its population
		is a bit too unrealistic even for an admittedly simple model.

	\item What is the criteria of truth for agent based computation? What
		makes one better than the other? Based on the three articles in
		the curriculum its seems to be `eyeballed' resemblance to the
		real world, rather than systematic comparison.

	\item Do such models still have a place in social science, given that
		they examine causation of the (at least yet) unobservable? Or,
		as in the case of \citet{Bhavnani_2008}, a subject that does not
		lend itself to large-N approaches.

\end{itemize}

\section*{Session IV}

\begin{itemize}

	\item Can we be sure that the number of refugees in \citet{Salehyan2006}
		is not just a visible proxy for the amount of movement across a
		border from a country in conflict? For example, when a rebel
		group in eastern DRC steps up its activities, this is likely to
		simultaneously drive refugees into neighboring countries and
		provoke a government response that could push them (or parts of
		the group) across the border as well. As this scenario unfolds
		in eastern DRC most of both movements will likely be into DRC's
		neighbors to the East.

	\item  Why do \citet{Salehyan2006} use the natural logarithm of
		refugees? If not a linear relationship would it not an
		escalating function? Why is there no discussion of this even?
	
	\item Why do \citet{Salehyan2006} define neighbors as borders falling
		within 100 kilometers or less (or 950km or less)? Why not just
		use \textit{neighboring countries}?

\end{itemize}

\section*{Session VI}

\begin{itemize}

	\item What happens to child soldiers inside rebel organizations as they
		grow up to be adults?

	\item Western observers, through our rose tinted vision, tend to assume
		child soldiers are abducted. The assumption being that no child
		would take up arms in a rebel organization voluntarily. I
		believe that the relative ``excess" attention that child
		soldiers has received is due to the this assumption that causes
		child soldiers to occupy a more "mental real estate" due to
		their dual role as victims and perpetrators. A victim of
		strategic sexual violence on part of a rebel group, unspeakably
		horrible as it is, is still ``just" a victim.  ``Just" in the
		sense that he/she only occupies that \textit{one} mental
		category and not two. He/she is more easily mentally
		compartmentalized, and forgotten. Not forgetting is painful and
		leaves us with three options: bare this pain, do something about
		the problem, or forget about it. The issue of child soldiers
		thus receives more attention because it is more difficult to
		mentally compartmentalize, and by extension, to forget (although
		probably equally difficult to bare, as both issues are morally
		abhorrent).

		A test of this hypothesis could be identifying other issues that
		are likewise difficult to compartmentalized, and see if these
		also receive disproportionate attention.

	\item How would \citet{Manekin_2017}'s theory apply to the classic case
		of widespread shirking, draft-dodging and dissertations of the
		Vietnam war?

\end{itemize}

\pagebreak

\bibliographystyle{agsm}
\bibliography{../lib.bib}

\end{document}
