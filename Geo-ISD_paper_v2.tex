
\documentclass[12pt]{article}
\usepackage{graphicx}
\usepackage{caption}
\usepackage{natbib}
\usepackage{authblk}
\usepackage[utf8]{inputenc}
\usepackage{setspace}
\usepackage{rotating}
\usepackage[british]{datetime2}


\renewcommand\Affilfont{\itshape\small}

\title{Geo-ISD vs Paine}
\author[1]{Marius Swane Wishman}
\affil[1]{Department of Sociology and Political Science, NTNU}

\date{\today}

\providecommand{\keywords}[1]
{
	\small	
	\textbf{\textit{Keywords---}} #1
}

\begin{document}

\maketitle

\begin{abstract}
\end{abstract}

\keywords{}

\pagebreak

%tableofcontents
%\pagebreak

\onehalfspacing

\section{Introduction}

The literature on historical state entities (hereafter HSE's), pre-colonial states and so on, and conflict has drawn mixed conclusions.
The general tendency has been toward that HSE's are often locally peace inducing \citep{Wig2016, Wig2018} but nationally conflict inducing \citep{Paine2019}.
Employing new data allows this paper to test the existing theories in the literature, and answer the question of where, relative to HSE borders, does modern (post 1946) conflict occur.

\section{Literature/theory}

\subsection{Recap \citet{Wig2016} and counterpoints}
\citet{Wig2016} argues that ethnic groups with ties to pre-colonial statehood are more likely to have inherited institutions that allow the ethnic group to punish defections and hold their leaders accountable.
In this way, ethnic groups with ties to pre-colonial statehood are better able to make credible commitments, than 'non-state' ethnic groups.
Credible commitments help such groups both prevent conflict from occurring in the first place, but also make them better able to end conflicts when then they have broken out.
Empirically \citet{Wig2016} finds that groups with histories of statehood do indeed experience less dyadic conflict with their government.
\citet{Depetris-Chauvin2016} makes a similar argument and finds that regions with exposure to pre-colonial statehood are more peaceful, \emph{ceteris paribus}.

On the other hand, someone else found the opposite. Possibly \citet{Besley2014}.

\subsection{Recap \citet{Paine2019}}
\subsection{Multiple HSE's}
\section{Research design}

\section{Conclusion}

\pagebreak

\bibliographystyle{agsm}
\bibliography{lib.bib}

\section{Appendix}

\end{document}

