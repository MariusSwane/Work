
\documentclass{beamer}

\usepackage{graphicx}
\usepackage[light]{FiraSans}
\usepackage[british]{datetime2}
\usepackage{tabularx}	
\usetheme{default}
\setbeamertemplate{navigation symbols}{} % No navigation symbols
\definecolor{ntnu}{cmyk}{100,75,0,5}
\setbeamercolor{alerted text}{fg=ntnu}
\setbeamercolor{frame title}{fg=ntnu}
\setbeamercolor{title}{fg=ntnu}
\setbeamercolor{subtitle}{fg=ntnu}
\setbeamercovered{transparent}

%\setbeamertemplate{itemize item}{\color{white}$\bullet$} 
% Include above line to remove bullet indicators

\setbeamertemplate{footline}{
\begin{tabularx}{\textwidth} {
	 >{\raggedright\arraybackslash}X 
  	 >{\centering\arraybackslash}X 
  	 >{\centering\arraybackslash}X 
  	 >{\centering\arraybackslash}X 
  	 >{\centering\arraybackslash}X 
  	 >{\centering\arraybackslash}X}
	
	\raisebox{-0.3cm}
	{\includegraphics[width=2cm, keepaspectratio]{logo_ntnu_u-slagord.pdf}} &
	\insertshortauthor & 
	\insertshorttitle &
	\insertdate &
	\insertsection &
	$\big|$ \insertframenumber
\end{tabularx}
}

\makeatletter
\makeatother

%----------------------------------------------------------------------------------------
%	TITLE PAGE
%----------------------------------------------------------------------------------------

\title[POL2012]{POL2012 Theories and models in political economy}

\subtitle{Poverty and development}

\author[Wishman]{Marius Swane Wishman} 
\date{\today} 
\institute{ISS}

\begin{document}

\begin{frame}[plain]
\titlepage 
\centering
\includegraphics[width=5cm]{logo_ntnu_u-slagord.pdf} 
\end{frame}

\section{What not to do} 

\begin{frame}
\frametitle{What went wrong in Africa?}
	\begin{itemize}
		\item[-] What is the $y$?\pause % political order/ state
			% collapse
			% More political than we are used to
		\item[-] What would a causal model look like? 
		% state predates -> citizens arm -> political order declines
		% state protects -> citizens work -> political order is stable
			% (economy grows), persists in equilibrium
			% (public revenues - rewards from predation) * rate of discount ->
			% state choice
	\end{itemize}
\end{frame}

\section{Economic mismanagement}

\begin{frame}{}

	\begin{itemize}
		\item[-] Why did governments choose policies that impoverished
			their citizens? \pause
		\item[-] Fruits of independence \pause
			% Created small leading circles focused on providing
			% private rather than public benefits
	\end{itemize}

\end{frame}

\begin{frame}{Discussion}

	Zambian automobiles
	
	Did the "Zambia's" of Africa in the 60-ties follow Reinert's advice?

	\begin{itemize}
		\item[-] Why? \pause
		\item[-] To establish some political order and maximize private gains?
		\item[-] Genuine attempt at development through evolutionary
			economics?
		% EE responds by blaming the West - pulling up the ladder, IMF
			% forced liberalization etc.	
	\end{itemize}

\end{frame}

\begin{frame}{A delicate equilibrium}

	\begin{itemize}
		\item[-] Where did public revenue come from? \pause 
		% Taxing peasants through exports of agricultural goods
			% Easy! But vulnerable.
		\item[-] What would Keynes say? \pause
		\item[-] Control regimes
	\end{itemize}

\end{frame}

\section{What to do?}

\begin{frame}{How Asia did it} % Great example of evolutionary economics
\begin{itemize}
    \item Land reform \pause
    \item Manufacturing focus anchored in export discipline \pause
    \item Financial manipulation to these ends
\end{itemize}{}
\end{frame}

\begin{frame}{}
	What is the theory?
\end{frame}

\begin{frame}{Agriculture}

	\begin{itemize}
		\item[-] Land reform \pause
			% SO difficult!
		\item[-] The perfect market \pause
		\item[-] "Gardening" - economies of scale in reverse \pause
		\item[-] A sturdy foundation 
	\end{itemize}

\end{frame}

\begin{frame}{Manufacturing}

	\begin{itemize}
		\item[-] The usual suspects \pause
		\item[-] "Export discipline"
	\end{itemize}

\end{frame}

\begin{frame}{The short leash}

	\begin{itemize}
		\item[-] Control/manipulate financial markets to achieve the
			two previous steps \pause
		\item[-] What of monetary policy/exchange rates?
	\end{itemize}

\end{frame}

\section{Summary}

\begin{frame}{}

	Why did the Asian chose differently than the African?

	% None of them were democratic by any meningful measure
		% Why not engage in minimizing the electorate and maximizing
		% personal extraction?
	% Not all Asian countries went the way of the tigers...
	% Natural resources
	% No fruits of liberty problem
	% Longer time horizons
	% The case of Qaddafi

\end{frame}

\end{document}
