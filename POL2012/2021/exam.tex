% ANUfinalexam.tex (Version 2.0)
% ===============================================================================
% Australian National University Final Exam LaTeX template.
% 2004; 2009, Timothy Kam, ANU School of Economics
% Licence type: Free as defined in the GNU General Public Licence: http://www.gnu.org/licenses/gpl.html

\documentclass[a4paper,12pt,fleqn]{article}
\usepackage{amsmath}
\usepackage{fancyhdr}
\usepackage[utf8]{inputenc}


% Insert your course information here %%%%%%%%%%%%%%%%%%%%%%%%%%%%%%%%%%

\newcommand{\institution}{Department of sociology and political science}
\newcommand{\institusjon}{Institutt for sosiologi og statsvitenskap}
\newcommand{\titlehd}{POL 2012: Theories and Models in Political Economy}
\newcommand{\tittelhd}{POL 2012: Teorier og modeller i politisk økonomi}
\newcommand{\examtype}{Fall Semester Examination}
\newcommand{\examdate}{29.11.2021}
\newcommand{\examcode}{POL 2012}
\newcommand{\examtime}{09:00-13:00}
\newcommand{\materials}{None}
\newcommand{\lastwords}{End of Examination}
\newcommand{\sisteord}{Slutt på Eksamen}
\newcommand{\sisteordnn}{Slut på Eksamen}

%%%%%%%%%%%%%%%%%%%%%%%%%%%%%%%%%%%%%%%%%%%%%%%%%%%%

%\setcounter{MaxMatrixCols}{10}
\newtheorem{theorem}{Theorem}
\newtheorem{acknowledgement}[theorem]{Acknowledgement}
\newtheorem{algorithm}[theorem]{Algorithm}
\newtheorem{axiom}[theorem]{Axiom}
\newtheorem{case}[theorem]{Case}
\newtheorem{claim}[theorem]{Claim}
\newtheorem{conclusion}[theorem]{Conclusion}
\newtheorem{condition}[theorem]{Condition}
\newtheorem{conjecture}[theorem]{Conjecture}
\newtheorem{corollary}[theorem]{Corollary}
\newtheorem{criterion}[theorem]{Criterion}
\newtheorem{definition}[theorem]{Definition}
\newtheorem{example}[theorem]{Example}
\newtheorem{exercise}[theorem]{Exercise}
\newtheorem{lemma}[theorem]{Lemma}
\newtheorem{notation}[theorem]{Notation}
\newtheorem{problem}[theorem]{Problem}
\newtheorem{proposition}[theorem]{Proposition}
\newtheorem{remark}[theorem]{Remark}
\newtheorem{solution}[theorem]{Solution}
\newtheorem{summary}[theorem]{Summary}
\newenvironment{proof}[1][Proof]{\noindent\textbf{#1.} }{\ \rule{0.5em}{0.5em}}

% ANU Exams Office mandated margins and footer style
\setlength{\topmargin}{0cm}
\setlength{\textheight}{9.25in}
\setlength{\oddsidemargin}{0.0in}
\setlength{\evensidemargin}{0.0in}
\setlength{\textwidth}{16cm}
\pagestyle{fancy}
\lhead{} 
\chead{} 
\rhead{} 
\lfoot{} 
\cfoot{\footnotesize{Page \thepage \ of \pageref{finalpage} -- \titlehd \ (\examcode)}} 
\rfoot{} 


\renewcommand{\headrulewidth}{0pt} %Do not print a rule below the header
\renewcommand{\footrulewidth}{0pt}


\begin{document}

% Title page

\begin{center}
%\vspace{5cm}
\large\textbf{\institution}
\end{center}
\vspace{.5cm}

\begin{center}
\large\textbf{\titlehd}
\end{center}

\begin{center}
\textit{Date: \examdate}
\end{center}

\begin{center}
\textit{Exam Time: \examtime}
\end{center}

\vspace{.5cm}

\begin{center}
	\textbf{Permitted examination support material:} None
\end{center}

\begin{center}
	\textbf{Academic contact during examination:} Marius Swane Wishman
	\textbf{Phone:} 47056893
\end{center}

\vspace{.5cm}

\begin{center}
	\textbf{OTHER INFORMATION}
\end{center}

\paragraph{\textbf{Get an overview of the question set}} before you start
answering the questions.

\paragraph{\textbf{Read the questions carefully}} and make your own assumptions.
If a question is unclear/vague, make your own assumptions and specify them in
your answer. Only contact academic contact in case of errors or insufficiencies
in the question set. Address an invigilator if you wish to contact the academic
contact. Write down the question in advance.

\paragraph{\textbf{Weighting:}} The exam consists of two parts that are weighted
equally.

\paragraph{\textbf{Notifications:}} If there is a need to send a message to the
candidates during the exam (e.g. if there is an error in the question set), this
will be done by sending a notification in Inspera. A dialogue box will appear.
You can re-read the notification by clicking the bell icon in the top right-hand
corner of the screen.

\paragraph{\textbf{Withdrawing from the exam:}} If you become ill or wish to
submit a blank test/withdraw from the exam for another reason, go to the menu in
the top right-hand corner and click “Submit blank”. This cannot be undone, even
if the test is still open. 

\paragraph{\textbf{Access to your answers:}} After the exam, you can find your
answers in the archive in Inspera. Be aware that it may take a working day until
any hand-written material is available in the archive.

% End title page

%\newpage
\clearpage

\section*{English}
\subsection*{Part 1}
\begin{quote}
\textit{Answer \textbf{ALL} of the following questions (50\%).}
\end{quote}

\paragraph{\textbf{Question 1:}}

Explain Ricardo's theory of rent.

\paragraph{\textbf{Question 2:}}

Model and explain a shift in demand.

\paragraph{\textbf{Question 3:}}

Assume that Italy can produce either 120 tonnes of olives or 7000 liters of red wine,
and Greece can produce 80 tonnes of olives or 2000 liters of red wine. Use
Ricardo's theory of trade to explain potential trade between the two countries.

%\paragraph{\textbf{Question 4:}}
%
%LTV Marx and Smith

\paragraph{\textbf{Question 4:}}

Explain Marx's theory of the falling rate of profit.

\paragraph{\textbf{Question 5:}}

What are the most common units of analysis in neo-classical tradition?

\paragraph{\textbf{Question 6:}}

Explain why land reform is a key first step of economic development according to
Studwell (2013).

\paragraph{\textbf{Question 7:}}

According to Bates (2008), what is the problem with what he calls `control
regimes'?

\bigskip
\subsection*{Part 2}
\begin{quote}
	\textit{Answer \textbf{1} out of the following questions (50\%):}
\end{quote}

\paragraph{\textbf{Question 1:}}

From an evolutionary economics perspective, explain either the developmental
success, or developmental failure, of a country of choice.

\paragraph{\textbf{Question 2:}}

Choose a relevant theoretical perspective to discuss the problem of economic
inequality and potential remedies.

\paragraph{\textbf{Question 3:}}

Discuss the global financial crisis (2008-2009) in light of a sovereign debt
crisis, as happened in Greece and Iceland.

\begin{center}
\vspace{3cm}
--------- \textit{\lastwords} ---------
\end{center}

\clearpage

\begin{center}
%\vspace{5cm}
\large\textbf{\institusjon}
\end{center}
\vspace{.5cm}

\begin{center}
\large\textbf{\tittelhd}
\end{center}

\begin{center}
\textit{Dato \examdate}
\end{center}

\begin{center}
\textit{Eksamenstid: \examtime}
\end{center}

\vspace{.5cm}

\begin{center}
	\textbf{Tillatte hjelpemidler:} Ingen
\end{center}

\begin{center}
	\textbf{Faglig kontakt under eksamen:} Marius Swane Wishman
	\textbf{Tlf:} 47056893
\end{center}

\vspace{.5cm}

\begin{center}
	\textbf{ANNEN INFORMASJON:}
\end{center}

\paragraph{\textbf{Skaff deg overblikk over oppgavesettet}} før du begynner på
besvarelsen din.

\paragraph{\textbf{Les oppgavene nøye,}} gjør dine egne antagelser og presiser i
besvarelsen hvilke forutsetninger du har lagt til grunn i tolkning/avgrensing av
oppgaven. Faglig kontaktperson skal kun kontaktes dersom det er direkte feil
eller mangler i oppgavesettet. Henvend deg til en eksamensvakt hvis du ønsker å
kontakte faglærer. Noter gjerne spørsmålet ditt på forhånd.

\paragraph{\textbf{Vekting av oppgavene:}} Eksamen består av to deler som vektes
likt.

\paragraph{\textbf{Varslinger:}} Hvis det oppstår behov for å gi beskjeder til
kandidatene underveis i eksamen (f.eks. ved feil i oppgavesettet), vil dette bli
gjort via varslinger i Inspera. Et varsel vil dukke opp som en dialogboks på
skjermen. Du kan finne igjen varselet ved å klikke på bjella øverst til høyre.

\paragraph{\textbf{Trekk fra/avbrutt eksamen:}} Blir du syk under eksamen, eller
av andre grunner ønsker å levere blankt/avbryte eksamen, gå til
``hamburgermenyen" i øvre høyre hjørne og velg `Lever blankt'. Dette kan ikke
angres selv om prøven fremdeles er åpen.

\paragraph{\textbf{Tilgang til besvarelse:}}  Etter eksamen finner du
besvarelsen din i arkivet i Inspera. Merk at det kan ta én virkedag før
eventuelle håndtegninger vil være tilgjengelige i arkivet.

\clearpage

\section*{Bokmål}
\subsection*{Del 1}

\begin{quote}
	\textit{Svar på \textbf{ALLE} de følgende spørsmålene (50\%): }
\end{quote}

\paragraph{\textbf{Spørsmål 1:}}

Forklar Ricardo's teori om leie.

\paragraph{\textbf{Spørsmål 2:}}

Modeller og forklar en endring i etterspørsel.

\paragraph{\textbf{Spørsmål 3:}}

Annta at Italia kan produsere enten 120 tonn oliven eller 7000 liter rødvin, og
at Hellas kan produsere enten 80 tonn oliven eller 2000 liter rødvin. Bruk
Ricardo's teori om handel til å forklare potensiell handel mellom de to landene.

\paragraph{\textbf{Spørsmål 4:}}

Forklar Marx's teori om fallende profittrate.

\paragraph{\textbf{Spørsmål 5:}}

Hva er de vanligste analyseenhetene i den neo-klassiske tradisjonen?

\paragraph{\textbf{Spørsmål 6:}}

Forklar hvorfor jordreform er et sentralt første steg innen økonomisk utvikling
i følge Studwell (2013).

\paragraph{\textbf{Spørsmål 7:}}

I følge Bates (2008), hva er problemet med det han kaller `kontrollregimer' (`control
regimes')?

\bigskip
\subsection*{Del 2}
\begin{quote}
	\textit{Svar på \textbf{1} av følgende spørsmål (50\%):}
\end{quote}

\paragraph{\textbf{Spørsmål 1:}}

Forklar enten den vellykkede økonomiske utviklingen eller mislykkede økonomiske
utviklingen i et valgfritt land, fra et `evolutionary economics'-perspektiv.

\paragraph{\textbf{Spørsmål 2:}}

Velg et relevant teoretisk perspektiv å diskuter probelmet med økonomisk
ulikhet, samt mulige botemidler.

\paragraph{\textbf{Spørsmål 3:}}

Diskuter den globale finanskrisen (2008-2009) i lys av en statsgjeldkrise, slik
tilfellet var for Hellas og Island.

\begin{center}
\vspace{3cm}
--------- \textit{\sisteord} ---------
\end{center}

\clearpage

\begin{center}
%\vspace{5cm}
\large\textbf{\institusjon}
\end{center}
\vspace{.5cm}

\begin{center}
\large\textbf{\tittelhd}
\end{center}

\begin{center}
\textit{Dato \examdate}
\end{center}

\begin{center}
\textit{Eksamenstid: \examtime}
\end{center}

\vspace{.5cm}

\begin{center}
	\textbf{Tillatne hjelpemiddel:} Ingen
\end{center}

\begin{center}
	\textbf{Fagleg kontakt under eksamen:} Marius Swane Wishman
	\textbf{Tlf:} 47056893
\end{center}

\vspace{.5cm}

\begin{center}
	\textbf{ANNAN INFORMASJON:}
\end{center}

\paragraph{\textbf{Skaff deg eit overblikk over oppgåvesettet}} før du byrjar å
svare på oppgåvene.

\paragraph{\textbf{Les oppgåvene nøye,}} gjer deg opp dine eigne meiningar og
presiser i svara dine kva for føresetnadar du har lagt til grunn i
tolking/avgrensing av oppgåva. Fagleg kontaktperson skal berre kontaktast dersom
du meiner det er direkte feil eller manglar i oppgåvesettet. Vend deg til ei
eksamensvakt om du ynskjer å kontakte faglærar. Noter gjerne spørsmålet ditt på
førehand.

\paragraph{\textbf{Vekting av oppgåvene:}} Eksamen består av to deler som vektes
likt.

\paragraph{\textbf{Varslingar:}} Dersom det oppstår behov for å gje beskjedar
til kandidatane medan eksamen er i gang (f.eks. ved feil i oppgåvesettet), vil
dette bli gjort via varslingar i Inspera. Eit varsel vil dukke opp som en
dialogboks på skjermen i Inspera. Du kan finne att varselet ved å klikke på
bjølla i øvre høgre hjørne på skjermen. 

\paragraph{\textbf{Trekk frå/avbråten eksamen:}} Blir du sjuk under eksamen,
eller av andre grunnar ynskjer å levere blankt/avbryte eksamen, gå til
``hamburgermenyen" i øvre høgre hjørne og vel `Lever blankt'. Dette kan ikkje
angrast sjølv om prøven framleis er open.

\paragraph{\textbf{Tilgang til svara dine:}} Etter eksamen finn du svara dine i
arkivet i Inspera. Merk at det kan ta ein virkedag før eventuelle handteikningar
vert tilgjengelege i arkivet.

\clearpage
\section*{Nynorsk}
\subsection*{Del 1}

\begin{quote}
	\textit{Svar på \textbf{ALLE} dei følgjande spørsmåla (50\%): }
\end{quote}

\paragraph{\textbf{Spørsmål 1:}}

Forklar Ricardo's teori om leige.

\paragraph{\textbf{Spørsmål 2:}}

Modeller og forklar ei endring i etterspørsel.

\paragraph{\textbf{Spørsmål 3:}}

Annta at Italia kan produsere enten 120 tonn oliven eller 7000 liter raudvin, og
at Hellas kan produsere enten 80 tonn oliven eller 2000 liter raudvin. Bruk
Ricardo's teori om handel til å forklare potensiell handel mellom dei to landa.

\paragraph{\textbf{Spørsmål 4:}}

Forklar Marx's teori om fallande profittrate.

\paragraph{\textbf{Spørsmål 5:}}

Kva er dei vanligste analyseenhetane i den neo-klassiske tradisjonen?

\paragraph{\textbf{Spørsmål 6:}}

Forklar kvifor jordreform er eit sentralt første steg innan økonomisk utvikling
i følgje Studwell (2013).

\paragraph{\textbf{Spørsmål 7:}}

I følgje Bates (2008), kva er problemet med det han kaller `kontrollregimar' (`control
regimes')?

\bigskip
\subsection*{Del 2}
\begin{quote}
	\textit{Svar på \textbf{1} av følgjande spørsmåla (50\%):}
\end{quote}

\paragraph{\textbf{Spørsmål 1:}}

Forklar enten den vellykka økonomiske utviklinga eller mislykka økonomiske
utviklinga i eit valfritt land, frå eit `evolutionary economics'-perspektiv.

\paragraph{\textbf{Spørsmål 2:}}

Vel eit relevant teoretisk perspektiv å diskuter probelmet med økonomisk
ulikskap, samt mulige botemidler.

\paragraph{\textbf{Spørsmål 3:}}

Diskuter den globale finanskrisa (2008-2009) i lys av ein statsgjeldkrise, slik
tilfellet var for Hellas og Island.

\begin{center}
\vspace{3cm}
--------- \textit{\sisteordnn} ---------
\end{center}

\label{finalpage}

\end{document}
