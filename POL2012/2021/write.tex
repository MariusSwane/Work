
\documentclass{beamer}

\usepackage{graphicx}
\usepackage[light]{FiraSans}
\usepackage[british]{datetime2}
\usetheme{default}
\setbeamertemplate{navigation symbols}{} % No navigation symbols
\definecolor{ntnu}{cmyk}{100,75,0,5}
\setbeamercolor{alerted text}{fg=ntnu}
\setbeamercolor{frame title}{fg=ntnu}
\setbeamercolor{title}{fg=ntnu}
\setbeamercolor{subtitle}{fg=ntnu}
\setbeamercovered{transparent}

\setbeamertemplate{itemize item}{\color{white}$\bullet$} % Comment this line for default bullet points (triangles)

\setbeamertemplate{footline}
{
	\begin{center}
	\begin{tabular}{ccccc|c}
	\raisebox{-0.3cm}{\includegraphics[width=2cm, keepaspectratio]{logo_ntnu_u-slagord.pdf}} &
	\insertshortauthor & 
	\insertshorttitle &
	\insertdate &
	\insertsection
	\hspace{2cm} &
	\insertframenumber
	\end{tabular}
	\end{center}
}

\makeatletter
\makeatother

%----------------------------------------------------------------------------------------
%	TITLE PAGE
%----------------------------------------------------------------------------------------

\title[Term paper]{How to write a term paper}

\subtitle{}

\author[Wishman]{Marius Swane Wishman} 
\date{\today} 
\institute{ISS}

\begin{document}

\begin{frame}[plain]
\titlepage 
\centering
\includegraphics[width=5cm]{logo_ntnu_u-slagord.pdf} 
\end{frame}

\section{Causality and theory} 

\begin{frame}
\frametitle{Causality}
	\begin{itemize}
		\item Your paper should explain some \textbf{\textit{causality}} \pause
		\item Variable analysis: Explaining a change in one variable with the 
			change in other variables
        	\item Both quant and qual does this \pause
        	\item $$y=X_1+X_2+X_3$$ \pause
		\item Example: Current level of GDP/capita ($y$) across Africa
			can be explained (in part) by the mortality of European
			missionaries ($x_1$).  (Because theory).  However, both
			$x_1$ and $y$ could be explained by varying local
			climates ($x_2$, confounding-/control variable), or the
			which European power colonized that area ($x_3$,
			confounding-/control variable). (Because theory).

	\end{itemize}
\end{frame}

\begin{frame}
\frametitle{Theory}

	\begin{itemize}
		\item You should be answering `why' questions. So, read your
			text and look for places someone might stop and ask
			`but, why?' \pause
		\begin{itemize}
		
			\item Because: theory. \pause
		
		\end{itemize}
		\item Answers should (generally) be of logical explanations 
	\end{itemize}

\end{frame}

\section{Resources}

\begin{frame}{Resources}

	Sources
	\begin{itemize}
		\item Google scholar \pause
		\item Oria \pause
	\end{itemize}
	
	Organizing references
	\begin{itemize}
		\item Zotero \pause
		\item endNote \pause
		\item Something is better than nothing \pause
	\end{itemize}

	(For future) Writing
	\begin{itemize}
		\item  Overleaf \pause
	\end{itemize}

\end{frame}

\section{General pointers}

\begin{frame}{General pointers}

	\begin{itemize}
		\item The text should be structured (have a discernible intro,
			main part and conclusion) \pause
		\item Avoid hyperbole 
			\begin{itemize}
				\item \textbf{Example:} Now we are in a time of
					\textit{huge} multinational
					conglomerates, and, technological firms
					that have more money then whole
					countries. There are \textit{vast}
					disparities in terms of income, wealth,
					and opportunities between citizens in
					the most developed places.  \pause
			\end{itemize}
		\item Do not use `we' unless you are writing in collaboration
			with someone.
		\item Do not be afraid to use `I'. If it is your conclusion or
			your argument etc. then be honest about it!
	\end{itemize}

\end{frame}

\begin{frame}{General pointers}

\begin{itemize}

	\item Avoid normative statements 
	\item Avoid `as-is-spoken' phrasing
		\begin{itemize}

		\item Example: As the U.S. has a strong belief in the American
			dream, the notion of as long as you do your best and
			believe in yourself, you can achieve anything, they
			don’t have much of a welfare system to brag about. And
			as has been seen in cases of financial crises, that’s
			something than should be changed in the American
			society. \pause

		\end{itemize}

	\item Do not plagiarize\pause
	\begin{itemize}
		\item  If you cut and paste text from anywhere else, it needs to come
		with quotation marks and a paged reference in your text. Block
		quotes are still something you should try to avoid, as
		widespread use of quotes dampens your own voice in the text.
	\end{itemize}
\end{itemize}	

\end{frame}

\begin{frame}{General pointers}

	\begin{itemize}
		\item You should be trying to \textit{convince} the reader. 
			\begin{itemize}
				\item The point is not to simply show all sides
					to an argument, but to do so, and then
					explain to the reader what the
					``correct" conclusion is and why \pause
			\end{itemize}

		\item Avoid using non-academic sources
		\begin{itemize}
			\item Unless you absolutely have to, which
			is rarely the case. Empirical data is an exception, but
			you still need to be critical of your sources.
			Wikipedia, SNL (Norwegian), most addresses with a .com
			domain are NOT good sources! 
		\end{itemize}
	\end{itemize}

\end{frame}

\begin{frame}{General pointers}

\begin{itemize}
	\item You are not citing the author, you are citing specific work
\begin{itemize}

	\item I don't need to know his/hers fist name, nor profession,

	\item \textbf{Example of what not to do:} Economist Mark Blaug's (1976)
		view of human capital theory (HCT) is useful when trying to
		understand the scope of it. 
	
	\item \textbf{Example of what to do:} On a microeconomic level
		individuals invest in HC to get private economic returns, while
		on a macroeconomic level societies invest in HC for its members
		to help economic growth (Mincer 1984). 

\end{itemize}
\end{itemize}	

\end{frame}
\end{document}
