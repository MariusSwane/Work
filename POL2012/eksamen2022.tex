% ANUfinalexam.tex (Version 2.0)
% ===============================================================================
% Australian National University Final Exam LaTeX template.
% 2004; 2009, Timothy Kam, ANU School of Economics
% Licence type: Free as defined in the GNU General Public Licence: http://www.gnu.org/licenses/gpl.html

\documentclass[a4paper,12pt,fleqn]{article}
\usepackage{amsmath}
\usepackage{fancyhdr}
\usepackage{hyperref}
\usepackage[utf8]{inputenc}

\hypersetup{
    colorlinks=true,
    linkcolor=blue
}

% Insert your course information here %%%%%%%%%%%%%%%%%%%%%%%%%%%%%%%%%%

\newcommand{\institution}{Department of sociology and political science}
\newcommand{\institusjon}{Institutt for sosiologi og statsvitenskap}
\newcommand{\titlehd}{POL 2012: Theories and Models in Political Economy}
\newcommand{\tittelhd}{POL 2012: Teorier og modeller i politisk økonomi}
\newcommand{\examtype}{Fall Semester Examination}
\newcommand{\examdate}{10.05.2022}
\newcommand{\examcode}{POL 2012}
\newcommand{\examtime}{15:00-19:00}
\newcommand{\materials}{None}
\newcommand{\lastwords}{End of Examination}
\newcommand{\sisteord}{Slutt på Eksamen}
\newcommand{\sisteordnn}{Slut på Eksamen}

%%%%%%%%%%%%%%%%%%%%%%%%%%%%%%%%%%%%%%%%%%%%%%%%%%%%

%\setcounter{MaxMatrixCols}{10}
\newtheorem{theorem}{Theorem}
\newtheorem{acknowledgement}[theorem]{Acknowledgement}
\newtheorem{algorithm}[theorem]{Algorithm}
\newtheorem{axiom}[theorem]{Axiom}
\newtheorem{case}[theorem]{Case}
\newtheorem{claim}[theorem]{Claim}
\newtheorem{conclusion}[theorem]{Conclusion}
\newtheorem{condition}[theorem]{Condition}
\newtheorem{conjecture}[theorem]{Conjecture}
\newtheorem{corollary}[theorem]{Corollary}
\newtheorem{criterion}[theorem]{Criterion}
\newtheorem{definition}[theorem]{Definition}
\newtheorem{example}[theorem]{Example}
\newtheorem{exercise}[theorem]{Exercise}
\newtheorem{lemma}[theorem]{Lemma}
\newtheorem{notation}[theorem]{Notation}
\newtheorem{problem}[theorem]{Problem}
\newtheorem{proposition}[theorem]{Proposition}
\newtheorem{remark}[theorem]{Remark}
\newtheorem{solution}[theorem]{Solution}
\newtheorem{summary}[theorem]{Summary}
\newenvironment{proof}[1][Proof]{\noindent\textbf{#1.} }{\ \rule{0.5em}{0.5em}}

% ANU Exams Office mandated margins and footer style
\setlength{\topmargin}{0cm}
\setlength{\textheight}{9.25in}
\setlength{\oddsidemargin}{0.0in}
\setlength{\evensidemargin}{0.0in}
\setlength{\textwidth}{16cm}
\pagestyle{fancy}
\lhead{} 
\chead{} 
\rhead{} 
\lfoot{} 
\cfoot{\footnotesize{Page \thepage \ of \pageref{finalpage} -- \titlehd \ (\examcode)}} 
\rfoot{} 


\renewcommand{\headrulewidth}{0pt} %Do not print a rule below the header
\renewcommand{\footrulewidth}{0pt}


\begin{document}

\begin{center}
%\vspace{5cm}
\large\textbf{\institusjon}
\end{center}
\vspace{.5cm}

\begin{center}
\large\textbf{\tittelhd}
\end{center}

\begin{center}
\textit{Dato \examdate}
\end{center}

\begin{center}
\textit{Eksamenstid: \examtime}
\end{center}

\vspace{.5cm}

\begin{center}
	\textbf{Tillatte hjelpemidler:} A / Alle hjelpemidler tillatt
\end{center}

\begin{center}
	\textbf{Faglig kontakt under eksamen:} Marius Swane Wishman
	\textbf{Tlf:} 470 56 893
\end{center}

\begin{center}
	\textbf{Teknisk hjelp under eksamen:} NTNU Orakel
	\textbf{Tlf:} 735 91 600
\end{center}


Får du tekniske problemer underveis i eksamen, må du ta kontakt for teknisk
hjelp snarest mulig, og senest \underline{innen eksamenstida løper
ut/prøven stenger.} Kommer du ikke gjennom umiddelbart, hold linja til du får
svar.

\vspace{.5cm}

\begin{center}
	\textbf{ANNEN INFORMASJON:}
\end{center}

\paragraph{\textbf{Ikke ha Inspera åpen i flere faner, eller vær pålogget på
flere enheter, samtidig,}} da dette kan medføre feil med lagring/levering av
besvarelsen din.

\paragraph{\textbf{Skaff deg overblikk over oppgavesettet}} før du begynner på
besvarelsen din.

\paragraph{\textbf{Les oppgavene nøye,}} gjør dine egne antagelser og presiser i
besvarelsen hvilke forutsetninger du har lagt til grunn i tolkning/avgrensing av
oppgaven. Faglig kontaktperson kan kontaktes dersom du mener det er feil
eller mangler i oppgavesettet. 

\paragraph{\textbf{Juks/plagiat:}} Eksamen skal være et individuelt, selvstendig
arbeid. Det er tillatt å bruke hjelpemidler, men vær obs på at du må følge
eventuelle anvisningen om kildehenvisninger under. Under eksamen er det ikke
tillatt å kommunisere med andre personer om oppgaven eller å distribuere utkast
til svar. Slik kommunikasjon er å anse som juks. Alle besvarelser
blir kontrollert for plagiat.
\href{https://innsida.ntnu.no/wiki/-/wiki/Norsk/Juks+p\%C3\%A5\%20eksamen}{Du
kan lese mer om juks og plagiering på eksamen her.} 

\paragraph{\textbf{Kildehenvisninger:}} Før kilder i henhold til APA-stil.
\href{https://i.ntnu.no/wiki/-/wiki/Norsk/Bruke+referansestilen+APA}{Du kan lese
mer om APA-stil her.}

\paragraph{\textbf{Varslinger:}} Hvis det oppstår behov for å gi beskjeder til
kandidatene underveis i eksamen (f.eks. ved feil i oppgavesettet), vil dette bli
gjort via varslinger i Inspera. Et varsel vil dukke opp som en dialogboks på
skjermen i Inspera. Du kan finne igjen varselet ved å klikke på bjella øverst i
høyre hjørne på skjermen. Det vil i tillegg bli sendt SMS til alle kandidater
for å sikre at ingen går glipp av viktig informasjon. Ha mobiltelefonen din
tilgjengelig.

\begin{center}
	\textbf{BESVARE OG LEVERE}
\end{center}

\paragraph{\textbf{Besvare i Inspera:}} Hvis oppgavesettet inneholder oppgaver
som ikke er av typen filopplasting, skal de besvares direkte i Inspera. I
Inspera lagres svarene dine automatisk hvert 15. sekund.

NB! Klipp og lim fra andre programmer frarådes, da dette kan medføre at
formatering og elementer (bilder, tabeller etc.) vil kunne gå tapt.

\paragraph{\textbf{Automatisk innlevering:}} Besvarelsen din leveres automatisk
når eksamenstida er ute og prøven stenger, forutsatt at minst én oppgave er
besvart. Dette skjer selv om du ikke har klikket ``Lever og gå tilbake til
Dashboard" på siste side i oppgavesettet. Du kan gjenåpne og redigere
besvarelsen din så lenge prøven er åpen. Dersom ingen oppgaver er besvart ved
prøveslutt, blir ikke besvarelsen din levert. Dette vil anses som “ikke møtt”
til eksamen.

\paragraph{\textbf{Trekk fra/avbrutt eksamen:}} Blir du syk under eksamen, eller
av andre grunner ønsker å levere blankt/avbryte eksamen, gå til
``hamburgermenyen" i øvre høyre hjørne og velg `Lever blankt'. Dette kan
\underline{ikke} angres selv om prøven fremdeles er åpen.

\paragraph{\textbf{Tilgang til besvarelse:}}  Du finner din besvarelse i Arkiv
etter at sluttida for eksamen er passert.

\clearpage

\section*{Bokmål}

\bigskip

\begin{quote} \textit{Svar på \textbf{BEGGE} de to følgende oppgavene. Det
forventes at studentene er i stand til å identifisere hvilke teorier/teoretiske
perspektiver (merk: flertall) som er relevante for hver oppgave. Oppgavene
vektes likt.} \end{quote}

\bigskip

\paragraph{\textbf{Spørsmål 1:}}

Diskuter økonomisk ulikhet ut i fra \underline{relevant teori fra pensum}.

\paragraph{\textbf{Spørsmål 2:}}

Som følge av krigen i Ukraina er det flere aktører i Norge som har argumentert
for å bedre matsikkerhet gjennom økt proteksjonisme. \underline{Bruk relevant
teori} \underline{fra pensum} for å diskutere fordeler og ulemper ved en slik
(proteksjonistisk) politikk.

\begin{center}
\vspace{3cm}
--------- \textit{\sisteord} ---------
\end{center}

\label{finalpage}

\end{document}
