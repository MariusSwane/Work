\documentclass[10pt, show notes]{beamer}

\usetheme[progressbar=frametitle]{metropolis}
\usepackage{appendixnumberbeamer}

\usepackage{booktabs}
\usepackage[scale=2]{ccicons}

\usepackage{pgfplots}
%\usepgfplotslibrary{dateplot}

\usepackage{xspace}
\newcommand{\themename}{\textbf{\textsc{metropolis}}\xspace}

\title{POL2012: Theories and Models in Political Economy}
\subtitle{Introduction}
% \date{\today}
\date{}
\author{Marius Swane Wishman}
\institute{Department of Sociology and Political Science}
\titlegraphic{\hfill\includegraphics[height=1.5cm]{img/ntnu.pdf}}

\begin{document}

\maketitle

\begin{frame}{Today's Lecture}
  \setbeamertemplate{section in toc}[sections numbered]
  \tableofcontents[hideallsubsections]
 \end{frame}

\section{Evaluation Form}

\begin{frame}{Term Paper}
\begin{itemize}[<+- | alert@+>]
	\item Pass/No pass - \textit{No pass = no exam}
	\item 5000 words (MAX) on subject of choice
	\item Research question needs to be approved by me
	\item Academic standards \note{Quality, sources, referencing and plagarism}
	\item \textbf{{Deadline: October 28}}
\end{itemize}
\end{frame}

\begin{frame}{Exam}
\begin{itemize}[<+- | alert@+>]
	\item 4 hours written exam
	\item 100\% of the grade
	\item Based on syllabus and lectures
	\item \textbf{{Exam date: November 29}}
\end{itemize}
\end{frame}

\section{Lectures, Seminars and Course Outline}

\begin{frame}{Lectures}
\begin{itemize}[<+- | alert@+>]
	\item Wednesdays 10:15 - 12:00, D5
\end{itemize}
\end{frame}

\begin{frame}{Seminars}
\begin{itemize}[<+- | alert@+>]
	\item Every other week
	\item Tuesdays 14:15 - 16:00, D11
	\item Seminars start September 11
	\item Class discussions
	\item Term paper discussions
	\item Participation
\end{itemize}
\end{frame}

\begin{frame}{Course Material}
\begin{itemize}[<+- | alert@+>]
	\item Frank Stilwell (2012): Political Economy: The Contest of Economic Ideas
	\item Erik S. Reinert (2007): How Rich Countries Got Rich… and Why Poor Countries Stay Poor
	\item Some book chapters
	\item Academic papers
	\item Recommended reading
\end{itemize}
\end{frame}

\begin{frame}{Preliminary Course Outline}
\begin{itemize}[<+- | alert@+>]
	\item See handout, or pdf on blackboard
	\item Two parts:
	\item Theoretical
	\item Applied
\end{itemize}
\end{frame}

\section{What is Political Economy?}

\begin{frame}{What is Political Economy?}
\begin{itemize}[<+- | alert@+>]
	\item Weingast and Wittman (2008):
	\item "Most commonly refers to interdisciplinary studies drawing upon economics, sociology, and political science in explaining how political institutions, the political environment, and the economic system — capitalist, socialist, or mixed — influence each other"
	\item Political economy is about growth
	\item Why tho?
	\item Fine, but how?
	\item That's it!
\end{itemize}
\end{frame}

\begin{frame}{Relevance of Political Economy}
\begin{itemize}[<+- | alert@+>]
	\item \alert<6>{Economic orthodoxy}
	\item The role of markets
	\item The role of government
	\item Economy as self-equilibriating
	\item Growth the default state
\end{itemize}
\end{frame}

\begin{frame}{Relevance of Political Economy}
\begin{itemize}[<+- | alert@+>]
	\item Financial crisis
	\item Climate change
	\item Rising inequality
	\item Stagnating growth
\end{itemize}
\end{frame}

\begin{frame}{Currents of Economic Thought}
\begin{itemize}[<+- | alert@+>]
	\item The new orthodoxy
	\item Late 1970s/early 1980s \only<3>{-Thatcher}\only<4>{-Reagan}\only<5>{-Washington Consensus}\only<6>{-Chicago School}
\end{itemize}
\end{frame}

\begin{frame}{Currents of Economic Thought}
\begin{itemize}[<+- | alert@+>]
	\item Other approaches
	\item Classical
	\item Marxist
	\item Institutional
	\item Keynesian
	\item "Modern political economy"
	\item Evolutionary/Schumpeterian/"The Other Canon"
\end{itemize}
\end{frame}

\begin{frame}{What do Political Economists do?}
\begin{itemize}[<+- | alert@+>]
	\item Structural change
	\item \only<2>{New technologies}\only<3>{Mergers and acquisitions}\only<4>{Globalization}\only<5>{Reorganization of employment conditions}
\end{itemize}
\end{frame}
\begin{frame}{What do Political Economists do?}
\begin{itemize}[<+- | alert@+>]
	\item Structural change
	\item Challenges
	\item \only<3>{Trade and debt}\only<4>{Financial stability}\only<5>{Unemployment}\only<6>{Uneven development}\only<7>{Climate, energy and environment}
\end{itemize}
\end{frame}

\section{Who are you?}

\end{document}
