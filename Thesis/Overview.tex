%{{{ Header

\documentclass[12pt]{article}


\begin{document}

\section*{Overview of the PhD thesis}

\subsection*{Introductory chapter}

\subsection*{Introducing the Anatomy of Resistance Campaigns (ARC) Dataset}

Charles Butcher, Jessica Maves Braithwaite, Jonathan Pinckney, Eirin Haugseth,
Ingrid Vik Bakken and Marius Swane Wishman.\\

\textit{Published in Journal of Peace Research.}\\

\underline{Abstract:} We introduce the Anatomy of Resistance Campaigns (ARC)
dataset, which records information on 1,426 organizations that participated in
events of maximalist violent and nonviolent contention in Africa from 1990-2015.
The ARC data disaggregate episodes of contention into their organizational
components and inter-organizational networks, containing 18 variables covering
organization-level features such as type, age, leadership, goals, origins,
social bases, and inter-organizational alliances. These data facilitate new
measurements of key concepts in the study of contentious politics, such as the
social and ideological diversity of resistance episodes, in addition to measures
of network centralization and fragmentation. This paper outlines the core
concepts underpinning the ARC data, the data collection method, and descriptive
statistics that illustrate trends in organizational participation over time and
how organization types vary in their main features. The paper also provides
initial evidence that structural factors correlate with the participation of
some organization types, but not others. Finally, we show how organization types
cluster together or repel each other during periods of contention. The ARC
dataset can resolve existing debates in the field and opens new avenues of
inquiry in the study of contentious dissent. It should be useful to scholars of
violent and nonviolent contention, repression and dissent, along with
researchers aiming to understand the dynamics of revolution and democratization. 

\subsection*{Beyond Ethnicity: Historical States and Modern Conflict}

Marius Swane Wishman and Charles Butcher.\\

\textit{Published in European Journal of Peace Research.}\\

\underline{Abstract:} Historical states, be they sprawling empires or nominal
vassal states, can make lasting impressions on the territories they once
governed. We argue that more historical states located within the borders of
modern states increase the chance of civil conflict because they: (1) created
networks useful for insurgency, (2) were symbols of past sovereignty, (3)
generated modern ethnic groups that activated dynamics of ethnic inclusion and
exclusion, and (4) resisted western colonialism. Using new global data on
historical statehood, we find a robust positive association between more
historical states inside a modern state and the rate of civil conflict onset
between 1946-2019. This relationship is not driven by common explanations of
state-formation that also drive conflict such as the number of ethnic groups,
population density, colonialism, levels of historical warfare, or other
region-specific factors. We also find that historical states are more likely to
be conflict inducing when they are located far from the capital and in poorer
countries. Our study points to unexplored channels linking past statehood to
modern day conflict that are independent of ethno-nationalist conflict and open
possibilities for a new research agenda linking past statehood to modern-day
conflict outcomes.

\subsection*{Communal Violence and the Legacy of Pre-Colonial States}

Marius Swane Wishman and Ole Magnus Theisen.\\

\textit{Unpublished manuscript.}\\

\underline{Abstract:} Within the communal violence literature recent authors
have stressed the potential conflict inducing effects of precolonial states,
while others have emphasized the potential conflict reducing effects of local
institutions associated with prior statehood. We address this apparent puzzle by
arguing that an initial reduction of commitment issues and inter-group security
dilemma introduced by pre-colonial states set in motion a positive feedback loop
of increased trade, reduced information problems, increased relative gains from
continued cooperation, and a legacy of mixed ethnic settlement patterns. In
support of the proposed mechanism we find that more precolonial state presence
is associated with higher levels of ethnic fractionalization, and while
precolonial states could cause more state based violence we find a general
conflict reducing effect on communal violence. This effect is particularly
strong in East Africa.

\subsection*{After Forever: Pre-Colonial States and Civil Conflict} 

Marius Swane Wishman.\\

\textit{Unpublished manuscript.}\\

\underline{Abstract:} This paper examines the relationship between the presence
of pre-colonial states and post cold war civil conflict. I argue that
pre-colonial state presence can be conflict inducing or reducing depending on
the relationship between the pre-colonial and post-independence states. To test
this argument the paper introduces the Geo-ISD data set, which maps the borders
of 82 independent states in Africa in the 1800-1914 period. I use these data to
create a topographic measure of state presence. Proxying the relationship
between the pre-colonial and post-independence state using the distance from the
post-independence capital, the article finds that higher levels of pre-colonial
state presence are conflict reducing in areas surrounding modern capital cities,
which is consistent with greater continuity of traditions and institutions
associated with statehood that are inherently conflict reducing. In areas
further away from the post-independence capital, higher levels of pre-colonial
statehood are found to be conflict inducing, consistent with the view that state
legacies can represent powerful symbols of past independence useful for
mobilization, and leave behind regional elite networks with the potential to
violently resist centralisation efforts of national governments. 

\end{document}

