% latex table generated in R 4.1.2 by xtable 1.8-4 package
% Wed Jan 26 12:09:38 2022
\begin{xltabular}{\textwidth}{p{0.14\textwidth}p{0.15\textwidth}p{0.1\textwidth}p{0.12\textwidth}p{0.5\textwidth}}
  \toprule
Country & Actor & Onset year(s) & Mechanism & Narrative \\ 
  \midrule
Ethiopia & ALF & 1975-1991 & Networks & State tried to curtail traditional Sultanate \citep{Shehim1985}. \\ 
  Uganda & Buganda & 1966 & Networks & Power sharing agreement broke down leading to a breif civil war \citep{Tuck2005}. \\ 
  DR Congo & State of Katanga & 1961 & Networks & Formal institution (king) led bid for secession of the Katanga region. \citep[99-100]{Nzongola2002} \\ 
  DR Congo & BDK & 1989 & Symbols &  \\ 
  DR Congo & Mining State of South Kasai & 1960 & Networks, symbols & Traditional chief led secession movement of South-Kasai region, and resurrected the royal title of the Luba Empire. \citep[105]{Nzongola2002} \\ 
  Indonesia & GAM & 1976 & Networks & Networks with deep roots to the HSE of Aceh used for rebel recruitment \citep{Aspinall2009}. \\ 
  Mali & FLM & 2015 & Symbols & The name of the group refers to an HSE \citep{Brown1968}. \\ 
  Mali & MUJWA & 2011 & Symbols & The groups seeks to revive the jihad of a HSE \citep{Zenn2015}. \\ 
  Nigeria & Ansaru & 2009 & Symbols & The groups seeks to revive the jihad of a HSE \citep{Zenn2015}. \\ 
  Libya & CLA & 2012 & Networks, symbols & The groups name refers to a short lived kingdom in Eastern Libya and the group elected a descendent of the former king as their leader \citep{Ahram2019}. \\ 
  India & ULFA & 1979 & Symbols & The group frequently invokes the Ahom kingdom and the chairman claims to be a prince eligible for the long defunct royal title \citep{Mahanta_2013, Goswami2014} \\ 
  India & UNLF, KCP, PREPAK  & 1979 & Symbols & Manipuri insurgent groups used the name of the historical kingdom (Kangleipak) and fought against the ``forced merger'' of between India and the princely state of Manipur \citep{Pettersson2021} \\ 
  Pakistan & BLF, BLA & 1948, 1974, 2004, 2019 & Symbols, networks & Low scale insurgency following forced accession of the Khan of Kalat. Khan redeclared independence in 1958, and the new khan announced the creation of the Council of Independent Balochistan in 2009 \citep{Ahmad2017} \\ 
    China & ETIM & 2008 & Symbols &  East Turkistan Islamic Movement (ETIM) seeks to revive the historical state of East Turkistan \citep{Pettersson2021, Soloshcheva2017} \\ 	  
    China & Tibet & 1950, 1956, 1959 & Symbols, networks &  Tibetan insurgents aim to restore the independence of the historical state of Tibet \citep{Pettersson2021} \\ 
    India & Sikh insurgents & 1983 & Symbols &  Sikhs seek to establish the independent state of Khalistan, referring to a long history of Sikh statehood, famously led by Rajit Singh in the 19th century \citep{Pettersson2021} \\ 
    Somalia & SSDF, Puntland & 1982 & Networks & Following the first civil war in Somalia, Puntland declares itself an autonomous region within federal Somalia. The elite has close ties to the old Majeerteen sultanate elite \citep[111-112]{Wimmer2018} \\ 
    Nigeria, CHA & Boko Haram & 2009 & Symbols & Founder of Boko Haram aims to restore Islamic rule with reference to the pre-colonial states of Kanem-Borno and the Sokoto Caliphate \citep{Barkindo2016, Zenn2013}  \\ 
    Malaysia & Sultanate of Sulu & 2013 & Symbols, institutions, networks & Sultanate of Sulu claims historical territory in Sabah, leading to fighting with Malaysia \citep{Pettersson2021}  \\     
    Philippines & MILF & 2013 & Symbols & The Moro Islamic Liberation Front uses the examples of the Sultanate of Maguindanao and the Sultanate of Sulu to justify demands for independence \citep[81]{Tuminez2007}  \\  
    Sudan & Darfuri rebel groups & 2003 & State weakness & Darfuri rebel groups launch an insurgency against the central government in Khartoum. The Darfur region is comparatively weak and undeveloped in part because the historical Sultanate of Darfur was ruled indirectly during colonialism while Khartoum and it's surrounds were the site of colonial state and infrastructure investments \citep[299]{OFahey2008}  \\
   \bottomrule
\end{xltabular}
