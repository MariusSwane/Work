%{{{ Header

\documentclass[12pt]{article}
\usepackage{authblk}
\usepackage{booktabs}
\usepackage{caption}
\usepackage{fancyhdr}
\usepackage{graphicx}
\usepackage{hyperref}
\usepackage{import}
\usepackage[utf8]{inputenc}
\usepackage{longtable}
\usepackage{multicol}
\usepackage{natbib}
\usepackage{rotating}
\usepackage{setspace} 
\usepackage{tabularx}

\hypersetup{
colorlinks=true,
citecolor=black,
urlcolor=blue
}

\renewcommand{\harvardurl}{\textbf{URL:} \url}

\renewcommand\Affilfont{\itshape\small}

\title{Populærvitenskaplig sammendrag}
\author[1]{Marius Swane Wishman}
\affil[1]{Department of Sociology and Political Science, NTNU}

\date{\today}

\providecommand{\keywords}[1]
{
	\small	
	\textbf{\textit{Keywords---}} #1
}

\begin{document}

%\maketitle

% }}}

\section{Populærvitenskaplig Sammendrag}

\subsection{Janusansikt: Statshistorie og organisert vold}

Enten man liker det eller ikke, er all verdens landmasse i dag delt mellom
omtrent 200 stater. Noen steder er den nåværende staten siste ledd i en mer
eller mindre uavbrutt rekke lokale statsdannelser som strekker seg flere hundre
år bakover. Mange steder, særlig i Afrika, er dagens stat en blanding av ikke
bare én, men flere statshistorier, Europeiske institusjoner og områder med
svært lite eller ingen historisk erfaring med ``Staten" som en måte å organisere
samfunnet på. Hvordan har ulike statshistorier og ulike sammensetningner av
slike historier påvirket konfliktnivået siden slutten av den kalde krigen?

Denne avhandlingen finner at selv en moderat grad av statshistorie i og rundt
hovedstader kan forhindre at mindre voldsepisoder i hovedstaten eskalerer ut av
kontroll. Dette kan forklares med at mer erfarne stater har mer veletablerte og
velintegrerte institusjoner og sikkerhetsapparater. På en annen side, i områder
med omfattende staterfaring som ligger langt vekk fra dagens hovedstad, finner
avhandlingen at denne erfaringen ofte brukes for å mobilisere til kamp mot den
sentrale makten. Dette støttes av andre funn i avhandlingen som viser at jo
flere distinkte statshistorier som befinner seg innen et moderne lands grenser,
jo flere tilfeller av borgerkrig vil landet oppleve. Men, selv i områder fjernt
fra hovedstader kan statshistorie ha en voldsdempende effekt. De samme
institusjonenene som kan brukes for å mobilisere mot sentralmakten, kan også
gjøre grupper bedre til å forhandle seg imellom, og således forebygge og dempe
dødelige voldsspiraler mellom lokale etniske grupper. Avhandlingens tredje
artikkel underbygger disse funnene ved å vise at etniske grupper i større grad
tørr å bosette seg blant hverandre i områder med større grad av statshistorie.

Resultatene som presenteres i avhandlingen bygger på to omfattende og innovative
statistiske datasett. Det første kartlegger voldelige og ikke-voldelige
organisasjoner som søker grunnleggende endringer i hvordan regime eller stat er
bygd opp. Det andre bruker kart fra 1800 til 1914, supplert med kart tegnet av
moderne historikere for å generere et 3D-mål, som fanger opp både staters
utbredelse (slik vanlige verdenskart gjør) og staters historiske
tilstedeværelse eller dybde.

\section{Popular Science Digest}

Whether you like it or not, all the worlds' land mass is today divided between
roughly 200 states. In some places, the current state is the last paragraph of a
more or less unbroken chain of local states that extends hundreds of years
backwards. In many places, especially in Africa, today's state is a mixture of
not only one, but of several state histories, European institutions and areas
with very little or no historical experience with `` the state "as a way of
organizing society. How have different state legacies and different compositions
of such histories affected the level of conflict in the Post Cold War period?

This dissertation finds that even a moderate degree of state history in and
around capital cities can prevent minor episodes of violence in the capital from
escalating out of control. This can be explained by the fact that more
experienced states have more well-established and well-integrated institutions
and security apparatuses. On the other hand, in areas with extensive state
legacies that are far away from today's capital, the thesis finds that such
experience is often used to mobilize for battle against the central government.
This is supported by other findings in the dissertation showing that the more
distinct state histories that are situated within a modern country's borders,
The more outbreaks of civil war the country will experience. However, even in
remote areas, state legacies can have an inhibitive effect on violence. The same
institutions that can be used to mobilize against the central government can
also make groups better able to negotiate amongst themselves, and thus prevent
and stifle deadly spirals between local ethnic groups. The thesis's third
article underpin these findings by showing that ethnic groups to a greater
extent dare to settle among each other in areas with deeper legacies of
statehood.

The results presented in the dissertation are based on two extensive and
innovative statistical datasets. The first maps violent and non-violent
organizations seeking fundamental changes in how regimes or states are
organized. The second uses maps from 1800 to 1914, supplemented with maps drawn
by modern historians to generate a 3D measurement, which captures both the
extent of states (as ordinary world maps do) and historical state presence or
depth of historical statehood.

\clearpage

\bibliographystyle{apsr}
\bibliography{../lib.bib}

\end{document}

