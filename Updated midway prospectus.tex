\documentclass[12pt]{article}
\usepackage{graphicx}
\usepackage{caption}
\usepackage[sort&compress]{natbib}
\usepackage{authblk}
\usepackage[utf8]{inputenc}
\usepackage{setspace}
\usepackage{rotating}
\usepackage[british]{datetime2}

\renewcommand\Affilfont{\itshape\small}

\title{Midway Evaluation Prospectus (updated)}
\author[1]{Marius Swane Wishman}
\affil[1]{Department of Sociology and Political Science, NTNU}
\date{\today}

\providecommand{\keywords}[1]
{
	\small	
	\textbf{\textit{Keywords---}} #1
}

\begin{document}

\maketitle

% \begin{abstract}

% \end{abstract}

\keywords{Artificial states, contentious politics, civil conflict, historical 
	states, state entities, state formation, civil war, ethnic conflict, 
pre-colonial states, long history}

\pagebreak

\pagebreak

\onehalfspacing

\section{Introduction}
%The overarching research question and themes of the dissertation.

Based on the first article of the thesis, research interests, plans for 
articles 2-4 and the ARC and Geo-ISD projects, I pose the following overarching 
research question for my thesis: How do historical and pre-colonial states 
affect current contentious politics?

States have been some of the largest and most powerful organizations in the 
world ever since they fist appeared \citep{Tilly1990}.
Nevertheless, surprisingly little research has gone into what happens to states 
after they disappear from the international system of states, and what happens 
to states incorporating the such historical states.
In the first article of this thesis we show multiple examples of states 
surviving on a sub-state (no longer sovereign) level, even decades after 
disappearing from the international system of sovereign states.
Other states are dismantled by conquering states, yet live on in identities and 
histories of the inhabitants of the bygone state.
This thesis will add to a growing literature exploring long histories, how 
differences in the past experiences affects recent or current outcomes, by 
examining the long history of statehood and its effect on contentious politics 
(both violent and non-violent).

\section{Beyond Ethnicity: Dead States and Modern Conflict}

This paper started from the overarching research question: How do historical
state entities (states that are more or less 'dead') affect post World War 2
levels of conflict?

The emerging/existing literature on the subject of pre-colonial and historical
(no longer sovereign states has reached differing conclusions.
Some scholars find that ethnic groups with more centralized pre-colonial
institutions experience less conflict with the central government \citep{
Wig2016} and that regions with longer histories of statehood are more peaceful
\citep{Depetris-Chauvin2016}. However, others have found that the 
conflicts of prior states can leave legacies of ethnic tension \citep{
Besley2014}.
In a recent article \citet{Paine2019} found that ethnic groups who lack a
history of statehood/centralized ethnic institutions, and find themselves 
within a country that has a group with such history, are more conflict prone 
than ethnic groups living in countries where no groups has such histories.
\citet{Paine2019} argues that this is because ethnic groups with a history of 
statehood or centralized ethnic institutions were more likely to inherit the 
state apparatus after decolonization. 
These groups would then more effectively (\emph{ceteris paribus}) exclude other 
groups from political power, leaving the excluded groups few channels to 
political power other than violence.
He also finds that in those instances where the group prior state history did 
not inherit 'the keys to the kingdom' they would also be more likely to engage
in violence to achieve political power.

% Alesina/Easterly ?

%Drawbacks: 
%* Start from the base of ethnic groups.
%* Most rely on a limited African sample.
%* Most rely on incomplete data.

Our paper makes three main contributions to the literature.
First, we do not assume that prior statehood necessarily affects conflict through
ethnic groups. 
Not all pre-colonial states were ethnic states in any meaningful sense, while
other were multi-ethnic in nature.
Some were even the foundations of current ethnic identities (the paper gives 
multiple examples).
Second, we employ new data that improves upon previous sources on pre-colonial
statehood by identifying far more states than without compromising the pre-
requisites for statehood and by having global coverage. 
Most of the literature has relied on either the \citet{Murdock1967} map of 
ethnic groups in Africa, the state antiquity data (which is global but covers 
relatively few states) or other incomplete data.
Third, we propose/construct a new measure of 'artificial statehood' that is
more in line with theory than existing measures such as the straightness of 
boundaries \citep{Alesina2011} or the variance in pre-colonial ethnic 
centralization \citep{Englebert2002}.
We measure 'artificial statehood' -- the degree to which a state overlaps with
the pre-existing topology of statehood -- as the number of historical state 
entities within its current boundaries.
We propose 4 mechanisms through which more historical state entities (HSEs)
increase the chance of civil conflict: HSEs (1) created networks useful for 
insurgency, (2) were symbols of past sovereignty, (3) generated modern ethnic
groups that activated dynamics of ethnic inclusion and exclusion and (4)
resisted western colonialism and specific values it brought with it.

Our hypothesis is:

\bigskip
\textit{H\textsubscript{1}: More historical states in the territory of a state
increases the number of internal armed conflicts.}
\bigskip

We find a robust positive association between more HSEs inside a modern state 
and the number of civil conflict onsets between 1946-2019. 
This relationship is not driven by common explanations of state-formation that 
also drive conflict such as the number of ethnic groups, population density, 
colonialism, levels of historical warfare, or other region specific factors.
Using mediation analysis we find some moderate support for the colonialism 
mechanism, although a strong independent effect of more historical state 
entities on civil conflict onsets remains across all models. 

\bigskip
Status: Awaiting response from JPR.

\section{GEO-ISD Project}

\section{ARC data release paper}

Finally, I will also co-author the data release paper from the Anatomy of 
Resistance Campaigns (ARC) project.
The paper introduces the ARC data set on groups participating in violent an non-
violent maximalist dissent in Africa over the period 1990-2015.

% \section{Other Ideas}

% %How Wilson changed the interaction between ethnicity and statehood.

% Building on the second paper.
%Many African democracies struggle with fractious party systems, with large 
%numbers of regional and ethnic parties competing to capture spoils of the state 
%(jobs, aid-programs, financial transfers, development projects etc.).
%Through similar mechanisms as described in the second paper expect more cohesive 
%countries to be better able to form policy-based political parties that span 
%regions and ethnic groups.
%\bigskip

\section{Progress}

In terms of the articles that will form part of the thesis, the first one is 
currently under review at International Organization.
The next step for the remaining articles is to finish the Geo-ISD data 
collection project, which is funded and underway.
This should be done by the end of July.
The resulting data should provide basis for at least two articles, hopefully 
three.

Overall I am content with the rate of progress so far, despite only having 
submitted one article for review.
This is because I believe I now have a clearer idea of what the rest of the 
thesis will look like, and how I will go about writing it.
Additionally, most of the duties to the department are done.
After this semester there will only be approximately 30 hours remaining.
I have also finished the requisite methods and philosophy of science courses 
that are part of the PhD Program.
10 ETC points worth of substantive course(s) remain.
I have signed up for the PRIO course ``Civil Resistance: Causes and 
Consequences", which will be 5 ETC points if I complete that as planned.

%Lastly, I plan to go on parental leave for 15 weeks from November next year.

\pagebreak
\bibliographystyle{agsm}
\bibliography{ArtificialBorders.bib}
\end{document}
