\documentclass[12pt]{article}
\usepackage{graphicx}
\usepackage{caption}
\usepackage{natbib}
\usepackage{authblk}
\usepackage[utf8]{inputenc}
\usepackage{setspace}
\usepackage{rotating}
\usepackage[british]{datetime2}
\usepackage{amssymb}
%\usepackage{hyperref}
\usepackage{url}

\renewcommand{\harvardurl}{\textbf{URL:} \url}
\renewcommand\Affilfont{\itshape\small}

\title{Notes from the literature}
\author[1]{Marius Swane Wishman}
\affil[1]{Department of Sociology and Political Science, NTNU}

\date{\today}

\providecommand{\keywords}[1]
{
	\small	
	\textbf{\textit{Keywords---}} #1
}

\begin{document}

\maketitle

\begin{abstract}

	This is a document for keeping notes from what I read in one place.
	Section headers are works (books or articles etc.), and should contain
	the reference. Sections describe, chapter or heading from the source,
	and the page number. Paragraphs are individual quotes, except when
	written in bold font, which is commentary by me.

\end{abstract}

\pagebreak

\tableofcontents

\pagebreak

\onehalfspacing
\section{Nation Building - Why Some Countries Come Together While Others Fall 
	Apart \citep{Wimmer_2018}}

\subsection{Introduction, p. 3}

Locals might resist a national government that intruded more into their daily
lives that did its colonial predecessor.  Political elites competed over who
controlled the new center of power.  Economic poverty, artificially drawn
boundaries, the legacies of colonial divide-and-rule policies, and the weakness
of postcolonial states made national political integration difficult.

\subsection{Introduction, p. 8}

Such voluntary organizations facilitate building alliances across ethnic
communities and regions, I will argue.  They bundle individual interests, as it
were such that politicians or state agencies can respond to them more easily.
In patronage systems, by contrast, each alliance needs to be managed separately:
a patron needs to provide political protection or government favors to each of
his clients on an individual basis.

\bigskip

How far such voluntary organizations have developed matters especially in early
years after a country transitions to the nation-state - when an absolutist
monarchy is overthrown or when a former colony becomes independent.  If a dense
web of such organizations haves already emerged, the new power holders can tap
into these networks to extend relationships of authority and support across the
country. 

\subsection{Introduction, p. 17}

Rather, .I both high diversity and low capacity to provide public goods emerge
in societies without a historical legacy of centralized states, as argued
throughout the preceding chapters.

\subsection{Chapter 1, p. 31}

The reach of political alliance networks-rather than communication in a shared
language per se - turns out to be crucial.  Where these networks were confined
by political boundaries such as imperial provinces, nationalists divided the
space of a shared language, as in Latin America.  Where language barriers within
imperial domains hampered the establishment of such ties, linguistic communities
were imagined as nations.  This happened in Romanov Russia.  Where civil
societies flourished early, as in Switzerland, or where states were
exceptionally capable of providing public goods, political networks stretched
across ethnic divides and the nation was imagined as polyglot.  I other words,
the contours of political alliance networks determine which communities emerge
as nations.

\subsection{Chapter 1, p. 66}

Because voluntary associations facilitate horizontal linkages across a
territory, the more associational networks have developed during the period
leading to the creation of a nation-state the easier it will be for the new
governing elites to build alliances across ethnic divides bu relying on these
networks.

\textbf{Conversely, HSE's create \textit{"vertical networks."}}
\textbf{Potentially facilitating both mobilization of said networks, but also
patronage networks.} 

\subsection{Chapter 2, p. 97}
The administrative footprint, however, of both the Ottoman Empire and the
Zanzibari Sultanate was extremely light and consisted of a handful of
representatives and tax collectors per town only.  The city-states never managed
to due the nomads of the hinterland.  These could mobilize tens of thousands
of warriors against which the feebly fortified coastal cities stood little
chance (Lewis 1988:34-35).  Only the Majeerteen Sultanate (a clan of the Darood
family) situated at the Horn managed to gain effective control over the interior
and represented a territorial state comparable to some of the more powerful and
populous Tswana kingdoms.  It remained independent from both the Ottoman Empire
and the Oman Sultanate by signing a treaty with Britain in 1839.  In the 1920s ,
Italy ended its long-lasting sovereignty through military conquest.

\subsection{Chapter 3, p. 103}

This fragmented, embryonic, politicized administration had virtually no tac base
beyond the tariffs it collected from imports and exports-a legacy of the
resistance that the interior clans had mounted against any form of taxation, in
turn the consequence of long centuries of living without a state.  Somalia
therefore became one of the most aid-dependent countries in Africa

\subsection{Chapter 3, p. 105}

The expanded political arena also offered incentives to found new parties with
both northern and southern clan elements, as the short-lived Somali Democratic
Union and the more stable Somali National Congress show (Lewis 1988: 176.  The
latter included northern Dir and Isaaq clans bu also important Hawiye elements
from the former Italian parts of the country-all united in the attempt to
counter the dominance of Darood, particularly Majeerteen politicians during the
first decade of independence.  Indeed, it seems that the politically dominant
clans during that first decade were the Hawaiye from central Somalia, the Isaaq
from the north and especially the Majeerteen from the former eponymous sultanate
on the Horn (Laitin and Samatar 1987: 92), who supplied one out of two
presidents and two prime ministers out of three.

\subsection{Chapter 3, p. 105}

Further fragmentation followed.  In the 1969 elections 62 parties, most of
which represented narrower clan interests, competed with each other.  The more
broadly based SYL won again.  Indicating the opportunistic nature of politics in
independent Somalia, all but one member of the opposition parties defected after
the election to join the government party SYL, hoping to get a piece of the pie
and distribute it to their clan supporters.

\subsection{Chapter 4, p. 148}

Third, the memories of independent statehood provided politically ambitious
Polish noblemen with a model for the future.  They thus confined their networks
of alliances to other Polish nobles (who made up as much as 20% of the
population in order to one day achieve the dream of renewed independent
statehood, a dream first couched in terms of rights to dynastic succession
rather than modern nationalist discourse.  The peasant population, whether
speaking Polish, Belorussian, Lithuanian, or Ukrainian, as well as Jewish town
dwellers, remained largely excluded from these networks of agitation and
mobilization and thus indifferent to the proto-nationalist cause of the Polish
nobility.

\section{How State Presence Leads to Civil Conflict \citep{Ying_2020}}

\subsection{Abstract}

\textbf{Argues that conflict occurs when states attempt to penetrate areas or 
	groups it has not controlled previously.}

\textbf{He tests both at \textit{province} and \textit{"ethnic group"}
level, using the accuracy of census data in the first and global ground
transportation data in the second analysis.}

\subsection{State Weakness and Civil Conflict}

Two concepts are crucial in understanding state power: (1) state capacity, i.e.
the overall power of a nation-state to administrate its territory (Evans,
Rueschemeyer, and Skocpol 1985), and (2) state presence, i.e. the allocation of
state power at the -national level. The lower the state-capacity the more
likely is state presence to be insufficient and uneven across regions. Existing
literature terms this phenomenon as limited statehood (Risse 2011) or
incomplete sovereignty/inefficient governance (Lee 2018).

\subsection{From State Capacity to Civil Conflict}

\textbf{Similar to Koren and Sarbahi 2018 but focuses on low intensity state-based
violence instead of civil war.}

\section{A Short History of the Cartography of Africa \citep{Stone1995}}

\subsection{Chapter 5 The Nineteenth Century: Golden Age of the Cartography of
Imperialism, p. 47-48 }

\textbf{The 19th century is a transition period leading up to the revolution
of the 20th  century cartography of colonialism.}

\textbf{Cartography in Africa is still a mix of measurement, less accurate
observations, word of mouth,  previous maps and sources, educated guesses and
pure conjecture.  Nevertheless a distinct improvement on the maps of previous
periods.}

\textbf{Maps are generally published in the year following the return of an
expedition.  However, expeditions could last for years, so observations could be
several years after the fact by the time they met print.}

\subsection{Southern Africa, p. 49-50}

It was the use of instruments, including a sextant, an artificial horizon,
chronometer, compass and measuring chain for traversing but corrected by means
of daily observation of latitude and occasional observation of longitude (de
Smidt 1896), which resulted in John Barrow's influential 
\textit{"General chart of the Colony of the Cape of Good Hope"}, accompanying
his \textit{"Travels into the Interior of Southern Africa"}, published in 1801
(Penn 1994).

The missionary thrust northwards through the eastern Kalahari was continued by 
Robert Moffat, whose map of SOUTH AFRICA \textit{"compiled for the Revd. R. 
Moffat's work by James Wyld"}, published in Moffat's \textit{"Missionary Labours
and Scenes in Southern Africa"} (1842) improved considerably on Campbell's 
depiction in the amount of geographical and ethnographic information and its 
accuracy. \textbf{While our data do not include the maps in question, we do have
two maps by James Wyld among our sources.}

The information which was forthcoming in the aftermath of the Great Trek duly
made its appearance in contemporary maps, notably in the maps of John Arrowsmith
(Schrive 1965). \textbf{Our data includes four maps by John Arrowsmith.}

\subsection{East Africa, p. 53}

A \textit{"Sketch Map of Eastern Africa"} accompanied Speke's journal, engraved
by W. \& A.K. Johnston. It contained much less of the detail which he had
recorded and which was contained in the map accompanying Burton's account, with
one significant difference. It showed the 'Mountains of the Moon 6000 to 8000
feet by estimation' as a horseshoe forming the catchment at the north end of
Lake Tanganyika.  \textbf{Our data includes a map of Africa by the same
	Johnstons from 1961 (two years after the first publication from Bruton
	and Speke's expedition). The map is clearly based on the one mentioned
	by \citet{Stone1995}, and includes the infamous Mountains of the Moon,
	as described above, and with Speke's name attached. In our next map by
the Johnstons (from 1879) the erroneous mountains have been removed. In total
our data contains five maps by one or more of the Johnston's.}

\subsection{East Africa, p. 58}

Further detailed observations by sequent explorers followed, in turn to be
compiled into route maps which could be used as sources by the great nineteenth
century atlas publishers such as W. \& A.K. Johnston, John Bartholomew and the
Society for the Diffusion of Useful Knowledge, but the hydrographical framework
had now been provided for east and central Africa. \textbf{As mentioned
previously, our data contains five Johnston maps. We also have one map by John
Bartholomew and two maps by the Society for the Diffusion of Useful Knowledge.}

\subsection{West Africa, p. 59-60}

The map which accompanies Denham and Clapperton's \textit{Narrative of Travels
and Discoveries in Northern and Central Africa} (1826) extends from the Gulf of
Sirtees to the Bight of Benin and is a typical explorer's transect, with
comments about wells, oases, wadis and villages. The western shores of Lake Chad
are firmly mapped, but the northeast shores are shown tentatively and the lower
course of the Niger is speculative. The map made an immediate impact with those
searching fir source material to update their maps of Africa, as can be seen in
John Cary's \textit{New Map of Africa Exhibiting the Recent Discoveries
constructed from the most recent travels}, published in 1828. Denham, Clapperton
and Oudney's route stands out by comparison with the scarcity of information
elsewhere in the Sahara, as it does on Sidney Hall's map of Africa of 1829.
\textbf{Our data contains one map by Cary and one by Hall. However, not they are
	not the ones mentioned here.}

\subsection{The Momentum Sustained, p. 61-62}
The published findings of the explorers of West Africa fed rapidly into the
mainstream of map and atlas publishing. In larger scale maps which carried
forward much of the detail of the explorer's originals, the task of combining a
large number of sources into a single mosaic at the selected scale was clearly a
major work of cartographic compilation. This is very well seen in the atlas of
\textit{Maps of the Society for the Diffusion of Useful Knowledge} (1844), which
contains a small scale map of the African continent, supplemented by larger
scale maps of parts of the continent. West Africa is divided into two sheets (I
and II) at a scale of c.1:5,600,000 extending from Timbuktu to the Guinea Coast
and from Cape Verde to Lake Chad. The extremely detailed content of the two maps
includes the routes of more than twenty explorers, including Beaufort, Gray,
Mollien, Winterbottom, Caillié, Park O'Byrne, Liang, Dupuis, Bowdien, Dochard,
Houghton, Dickson, Clapperton, Denham, Oudney, Lander, Hornemann, Laird, Allen,
Oldfield, Coulthurst and Toole. \textbf{Not one of the two included in our data.}

\subsection{The Momentum Sustained, p. 62}

The maps were published in 1839, (earlier then the atlas) and the explorers'
dates range from Houghton in 1791 to Oldfield in 1836. \textbf{The atlas
referred to is the one in the above subsection, published in 1844. In other
words, in this illustrative case, maps were published three years after the
expedition and was published in an atlas after eight.}

\subsection{The Momentum Sustained, p. 63}
These `stay-at-home scholars' (Bridges 1987b) re-examined the classical sources,
and they sought to learn from anyone with recent experience of Africa. The maps
of armchair geographers such as William Desborough Cooley, James McQueen and
A.G. Findlay were published by the RGS and are properly part of the evolving map
of the African interior in the 1850s.

\subsection{The Momentum Sustained, p. 65}
By the End of the nineteenth century, mapping by military personnel had been
instituted elsewhere in Africa, not only by French and British military
authorities. However, in the context of the nineteenth century as a whole, it
was peripheral in its location and limited in its extent, by comparison with the
work of the explorers. 

\subsection{The Characteristics of the Cartography of Imperialism, p. 69}

The imperial relationship with Africa was essentially international in
character, in the sense that the European powers shared a common purpose in the
continuation of access to trade at the coast of Africa. ... On the other hand,
the colonial period which was to follow can be differentiated by parochial
exclusivity and nationalism within colonial boundaries in a continent
partitioned among seven European powers, in contrast to the common attitudes and
access among European powers which characterised the previous imperial
relationship. The Europeans who made maps of Africa prior to colonial partition
scribed to common attitudes of the imperial powers. The subject matter of
their maps cannot easily be differentiated by their particular European country
of origin. They were a part of an imperial relationship which was international
in character.

\subsection{The Characteristics of the Cartography of Imperialism, p. 70}

Professor Bridges (1982) is able to talk about unofficial planning amongst
Europeans for direct intervention in East Africa by 1876, interventions which
would amount to interference in the lives of Africans. Such actions were
thought of as part of a laudable civilizing process involving European
organizational and technological skills. It was in this context that the
compilation of maps of the highest possible scientific calibre, using
instrumentally determined data to measure and locate places with optimum
accuracy, contributed to legitimizing European penetration and even
Interference in Africa. This background situation influenced the form of the
growing number of explorer's maps in the late nineteenth century.

\textbf{The rest of the page is very relevant as well, but I cannot rewrite the
whole thing here.}

\subsection{The Nineteenth Century Atlas Map of Africa, p. 72-73}

The picture of Africa which atlas publishers were presenting at the beginning of
the nineteenth century was not consistent. For example, de la Rochette's
\textit{Africa} dated 1803, published by W. Faden (Tooley 1969) incorporates
recent explorations in West Africa. It shows an eastward flowing upper Niger
(but not the as yet undiscovered outlet to the Gulf of Guinea) and the sources
of the Blue Nile. South of the Equator, it continues to rely heavily on the long
standing Portuguese sources for the lower Congo and Zambezi basin, with
extensive intervening otherwise blank areas containing a curious mix of textual
descriptions, either of an ethnographic nature or about potential routs. On the
other hand, James Cundee's \textit{New Royal Atlas Engraved from the best
modern authorities}, of 1810, includes a map of Africa \textit{Engraved for the
Revd. Mr. Evans's New Geographical Grammar} in 1809, which is much more
concerned with ethnic names distributed evenly across the entire face of the
continent, albeit with the recent knowledge of the drainage of West Africa as a
part of the background information. 

\textbf{The map published by W. Faden is not in our data, but two later ones
are. Cundee is not in our data.}

\subsection{The Nineteenth Century Atlas Map of Africa, p. 73-74}

Nevertheless, the great weight of new information becoming available from
explorers' maps was clearly making an impact.This is particularly well seen in
the four maps compiled in 1839 under the auspices of the Society for the
Diffusion of Useful Knowledge, founded in 1826 to promote 'the moral improvement
of the great body of the population', in part by means of publishing and
distributing suitable books (Smith 1974). Their atlas entitled \textit{Maps of
the Society}... published by Chapman and Hall in 1844, contained four maps of
Africa mostly dated to 1839, one of the entire continent, one of South Africa
(dated 1834), plus cover of West Africa at a larger scale in two maps. The
contents of the two maps of West Africa have already been described in the
context of the exploration of West Africa, but an aspect worthy of mention is
their total reliance on recent explorers as their sources. This is also true of
the relatively large scale maps of South Africa but less so in the case of the
map of the continent on a single sheet.

\textbf{This map of the whole continent is in our data.}

\subsection{Additional comments by me}

\textbf{\citet{Stone1995} builds the chapter covering the 19th century by
	covering a selection of `a few influential or characteristic maps. ...
	The choice of maps has been influenced by the valuable summary history
	of African exploration by Professor R.C. Bridges (1991).' I interpret
	this to mean that the emphasis is on quality and the maps drawn by the
	explorers themselves.  The overlap between our data and the maps covered
	by \citet{Stone1995} should therefor give an indication of the quality
	of our data.  At least in terms of including quality sources.}

\textbf{Of the 47 maps referred to in \citet{Stone1995} only six are included in
	the GeoISD. However, given \citet{Stone1995}'s emphasis on maps
	published by explorers and our reliance on compiled atlases this is no
	surprise. A different way to reckon would be to include maps by the same
	author or drawn by, compiled by, engraved by or published by the same
	author across \citet{Stone1995} and the GeoISD. This count should be
	able to capture some of the maps as they have ``trickled down" to atlas
	compilation. It is additionally a further ``fuzzy" measure of quality as
the quality of maps varied far more between mapmakers than between maps by the
same author. By this more inclusive count close to half the maps are ``partially
included" in the GeoISD (22/47).}
	
\section{The Art of Not Being Governed \citep{Scott2009}}

\subsection{State Space, p.42}

The realpolitik behind this elective affinity is evident in the fact that ``for
European governors and Southeast Asian rules alike, large settled populations
supported by abundant amounts of food were seen as the key to authority and
power." Land grants in ninth- and twentieth-century Java, for which we have
inscriptional evidence, were made in the understanding that the recipient would
clear the forest and convert shifting, swidden plots into permanent irrigated
rice fields (sawah). The Logic, as Jan Wisseman Christie notes, is that ``sawah
... had the effect of anchoring populations and increasing their visibility, and
making the size of the crop relatively stable and easy to calculate."

\textbf{An extension of this logic would be that where once states existed,
conditions for re-incorporation into a state are likely better, \textit{ceteris
paribus.}} Taken together with \citet{Ying_2020} there should be less resistance
to incorporation and thus less civil conflict in pre-colonial state areas.

\subsection{State Space, p. 53}

One might emphasize with Edmund Leach the fact that ``the riceland stayed in one
place" and thus represented a potential ecological and demographic strong
point, which a clever and lucky political entrepreneur might exploit to create a
new, or revived, state space.

\subsection{Ethnogenesis, p. 251}

Much of what passes as the ethnic particularity and distinctiveness of the
Shan, Burmese, and Thai cultures is closely tied to the basic devices for
statebuilding. Put in another way, ``stateness" is built into the foundations
of ethnicity. Reciprocally, in the Shan, Burmese, and Thai view, much of the
ethnicity of those populations in the hills, those not-yet-gathered-in,
consists precisely of their statelessness. 

\subsection{Ethnogenesis, p. 259}

The political entrepreneurs--official or not--who endeavor to mark out an
identity based on supposed cultural differences are not so much discovering a
social boundary as selecting one of innumerable cultural differences on which to
base group distinctions.  Whichever of these differences is emphasized (dialect,
dress, diet, mode of sistence presumed descent) leads to the stipulation
of a different cultural and ethnographic boundary distinguishing an ``us" from
a ``them." This is why the invention of the tribe is best understood as a
political project. The chosen boundary is a strategic choice because it is a
political device for group formation. 

\textbf{Once overcome, conquered or incorporated into a state, pre-colonial
states should be easier to govern effectively and peacefully according to Scott's
logic. However, in times of central state weakness (economic collapse, successive
coups, chaotic colonial hand-over) or otherwise opportunity-costs favoring
rebellion (access to lootable goods like oil or diamonds) prior state centers
can reemerge or become viable alternatives to the central government. Perhaps
like Somaliland or South Sudan. Not necessarily struggles for independence. More
likely increased autonomy or privileged access to resources or power. I expect
this to happen in PCS-areas because (1) the legitimizing effect of prior
statehood, (2) these areas have local elites on top of hierarchical social
structures \citep{Scott2009} and (3) often have access to relatively large
concentrated population. This last point however, also makes these areas
vulnerable to state crack-down. This could lead these areas to remain relatively
peaceful until the central government is especially weak.}

\textbf{A contrast could be to other, non-PCS ethnic groups. According to
\citet{Scott2009} these groups should be more numerous, more likely to be
minorities, more likely to be excluded from power, more geographically
dispersed and more likely to live in terrain chosen to avoid the state. For all
these reasons they should be more likely to violently resist state integration
\citep{Ying_2020, Scott2009}.}

\textbf{The resulting different patterns of conflict should be something like the
following for PCS areas/groups relative to Non-PCS groups/areas:}

\begin{itemize}
	\item Fewer conflict onsets
		\subitem Conflict resolution institutions/mechanisms
		\subitem Better integrated in the state already
		\subitem Easy target for repression (before onset, and
		deterrence)
		\subitem ``Pacified by earlier exposure to statehood and state
		monopoly of violence \citep{Pinker2012}
		\subitem Intermediate in terms of economic development at the
		national level? (Less than capital, more than peripheries?)
	\item Shorter but deadlier conflicts?
		\subitem Easier to target due to more concentrated urban
		population, not located in geography chosen to avoid state
		capture. I.e. conflict more closely resembling ``conventional
		warfare", as opposed to guerilla warfare, than for non-PCS
		groups/areas.
		\subitem State targeting urban centers is likely to cause more
		deaths.
	\item The most conflict events occur somewhere between the
		PCS area and the state capital. (As opposed to inside, or close
		to the core of the PCS area)?
		\subitem By the time fighting has reached either the core of
		the PCS area either side should recognize defeat?
\end{itemize}

\section{A History of Borno \citep{HiribarrenVincent2017AHoB}}

\subsection{Introduction, p. 14}

Its main argument is that Borno had a relatively structured territorial
framework in the nineteenth century. Disputing Jeffrey Herbst's assumption that
political domination was not territorial in pre-colonial Africa, this first
chapter emphasises the territorial singularity of Borno by analysing its borders
and the spatial structure of the kingdom. It argues that Borno was a bounded
territory with a codified relationship with its vassals.

\subsection{The Territory of Borno in the 19\textsuperscript{th} Century, p.30}

\textbf{The explorer Nachtigal quoted:}

Where there is no sharply defined natural frontier, such as the Chad and Shari,
these boundaries are indeterminate as towards the desert, or arbitrary and
fluctuating, as in the regions of Pagan and semi-Pagan tribes who have not been
brought completely under control. Where the Muhammadan inhabitants of two such
comparatively well-ordered states as Bornu and the Hausa country are adjacent,
the boundary can be fixed fairly exactly, though encroachments on either side
and boundary disputes are not lacking. Where, however, between the two
lie more or less independent regions as along a great pan of the western and
southern frontier of Bornu, the contours of the empire fluctuate according to
the measure of military success against tribes which are kept in subjection only
by force. This is the situation especially with the regions of the Bedde,
Ngizzem, Kerrikerri, Babir and Musgo, while the position in the Margin country,
where the proximity of Adamawa to the south has a decisive influence, is
somewhat more stable. Because of their more solid state organisation, Mandara
and Logon are also in a more regular relationship of dependence upon Bornu.

\textbf{Based on what I have read, Nachtigal's description is plausible.}

\subsection{The Territory of Borno in the 19\textsuperscript{th} Century, p.35}

Notions of `space' and `limits' are therefore present in Borno and Hausaland as
much as in Europe in the nineteenth century. Thus, even if the explorers had a
Eurocentric vision of the African continent, the Bornoan boundaries could have
been very similar to some European ones. Indeed, the whole kingdom of Borno was
a territory imagined not only by its inhabitants but also by the European
travellers.

\subsection{The Territory of Borno in the 19\textsuperscript{th} Century, p.37}

Cohen also quoted the case of the ruler of Zinder who was nominally under the
domination of the rulers of Borno but who rebelled in the 1830s and 1840s. Every
time, the Shehu of Borno had to re-establish his authority with an army. He
even left a consul in Xinder to represent the Bornoan power. In the 1870s, the
ruler of Xinder annexed another vassal of Borno, Muniyo. Maïkorema Zakari argued
that this invasion was possible because of the weakness of Borno at the end of
the nineteenth century. Thus, the relationship between Borno and its vassals
depended also on the political situation in Kukawa. The distance between Borno
`proper* and its vassals certainly played a role but was not essential as the
political situation in Kukawa seemed to be altogether more important.

\subsection{The Territory of Borno in the 19\textsuperscript{th} Century, p.41}

The Europeans defined the boundaries as lines which could be closed if needed.
They applied their own vision of the boundaries to the kingdom of Borno but
also reminded the readers of the frontier policy used in this part of Africa.
Barth by this comment revealed his knowledge of the local geopolitics but also
patronised the `vizier' of Borno who, according to Barth, could not properly
manage his northern boundary. This last showed how tenuous the power of Borno
must have been in this area.

It is clear that the traveller's narratives were largely influenced by theories
of statehood which were widespread in Europe in the nineteenth century. However,
their narratives also gave precious information about the borders of nineteenth
century Borno. They showed that Borno had a spatial structure which could be
sometimes precisely bounded. Nineteenth century Borno was thud more than a
network. Despite their fledging authority at the end of the century, the Shehus
exerted their authority over a territory. It does not mean that their authority
was supreme over the whole of Borno throughout the century; it rather indicates
that territorial rule was one of the attributes of statehood in
nineteenth-century Borno.

\subsection{All Paths Lead to Borno, p. 47}

\textbf{A Bornoan pilgrim quoted in a French intelligence report:}
The name of Borno was extended to the whole of the Empire of the Dunama which
consisted for the rest, in separated states, governed by vassal dynasties; the
vassality link becoming looser when these states were situated far from the
center of the Empire.

\subsection{All Paths Lead to Borno, p. 58}

The will to preserve the unity of the nineteenth-century polities was linked to
the fact that they would be easier to colonise and, later, administer. Thus, if
the British and French wanted to preserve the unity of Borno, it was because it
served their interests. This was the reason for which they conceptualised its
territory in their negotiations.

\subsection{The Quest for a Territorial Framework, p.72}

At the end of the nineteenth-century, the European conquest of Borno clearly
displayed the ambivalent European attitudes to African polities. On the one
hand, these considerations could be discarded to carve out colonies in Africa.

\subsection{The Quest for a Territorial Framework, p.85}

Therefore, the newly created boundary of Borno was a paradox as the Westphalian
boundary created in 1893 by the Germans and British divided the territory which
possessed delimited boundaries. Thus, the nineteenth-century boundaries, even if
blatantly ignored by the Europeans, still existed in Germany and British Borno.
From 1902 to 1914, the European-made boundary became the limit of the Shehus'
authority, thus recreating Borno within each colonial territory. It can be
argued that the Europeans were eager to transfer the concept of
nineteenth-century Borno to the newly created colonial territory. Thus, even
if, as seen in the previous chapter, the nineteenth-century limits of Borno were
ignored by the British and Germans in 1893, the concept of the territory of
Borno was re-used as it conveyed legitimacy to each European colonial territory.

\subsection{The Quest for a Territorial Framework, p.88}

The need for a legitimate ruler forced the British to bring Shehu Garbai into
their sphere of influence as their Northern Nigerian territories through their
hereditary leaders was at the centre of the British ambitions. Moreover, if the
French assertions were well founded, a Kanemi ruler could have claimed
suzerainty over the whole of Borno, British and German, despite the presence of
the Anglo-German boundary. In 1903 and at the start of 1904, both Shehus were
still trying to levy taxes in some border-villages located on the other side of
`their' borders. 

\subsection{The Quest for a Territorial Framework, p.91-92}

At the beginning of the twentieth century, Borno was not an ethnic group but a
kingdom. 

...

Thus, for two reasons, the maps representing Borno differed from other early colonial
maps of Africa. First, they represented a territory and not an ethnic group.
Secondly, the kingdom of Borno was an identified and prestigious landmark for
European cartographers.

\subsection{The Resurrection of Borno: (1902-1960), p.100}

As seen previously for Machina, historical reasons were more relevant to the
first colonial officers than ethnic ones. The main interest of British colonial
rule was not ethnic consistency but colonial efficiency. Preserving the
pre-conquest administrative framework was the key to this efficiency.

\subsection{The Resurrection of Borno: (1902-1960), p.103}

As the authors of the maps were still European and their source information was
still Bornoan, they reproduced the same image of Borno ruling over its distant
peripheries. This would not be the first time the Europeans gave one of their
rulers more land than he had before the European conquest. For example, when the
British created the protectorate of Uganda, they transferred the southern part
of the kingdom of Bunyoro to Buganda.

\subsection{The Resurrection of Borno: (1902-1960), p.104}

This fact could be compared with the French situation where, in theory, the
French are supposed to have created new political entities based on a Cartesian
and equalitarian conception of space. Actually, the administrative subdivisions
in the French African colonies, the \textit{cantons}, were based on a loose
conception of the nineteenth-century territorial structures. Robert
Delavignette, a French colonial administrator, stressed the inherited
territoriality of these divisions: `The canton is in most cases a former feudal
province turned into an administrative district'. In a chapter of his book
\textit{Freedom and Authority in French West Africa}, he also emphasised the
role of the African chiefs in the territorial framework of the French
sub-Saharan colonies. French rule was partly based on the assumption that strong
pre-colonial strictures were a sound basis for their own power. For example, in
what is now Burkina Faso, the French used the Mossi kingdoms as a basis for
their colonial administrative division. Even after different attempts at reform
in the first decade of the twentieth century, the structure of these kingdoms
was preserved in the colonial administration.

\subsection{Re-writing the History of Borno, p.121-122}

This personal sense of belonging enabled them to administer Borno more as a
ruler than as a colonial officer. Two factors were responsible for such an
attitude. Firstly, the geographical distance between the administrative capitals
of Northern Nigeria, Zugeru and later Kaduna, and the capital of colonial Borno.
As a result Borno was particularly isolated from the rest of the colony of
Northern Nigeria -- arguably, the residents in charge of Sokoto never developed
such a strong autonomous attitude. Secondly, it could be argued that the British
colonial officials were directly influenced by the strong political framework of
Borno.

\subsection{Re-writing the History of Borno, p.129}

Furthermore, a British version of Bornoan history was also transmitted in
schools. The successors of the nineteenth-century rulers of Borno also supported
the teaching of Bornoan history. The monarch of Borno from 1922 to 1937, Shehu
Sanda Kura, was particularly involved in this production of history books. The
joint work of the colonial officers and the local elite was thus taught to the
minority of young Bornoans who attended school.

\subsection{The Reunion of Dikwa and Borno: (1916-1959), p.138-139}

The historian Michael Callahan studied the impact of the creation of mandates on
European colonial policies. He revealed how the ideals of the American President
Woodrow Wilson pervaded the foreign policy of Britain and France towards their
mandates.

...

This mandate spirit, Callahan argues, created an obligation for the Europeans to
care for the populations living in the former German colonies. By stressing that
ethnic identities should not be split by colonial boundaries the British and
French governments recognised a legal right for Borno to be reunited. This right
was expressed in ethnic, cultural and religious terms by the colonial
administration and the press.

\subsection{Postcolonial Borno, p.165}

For example, Nigeria is supposed to be one of the best examples of an African
failed state. The British amalgamation of Nigeria in 1914 is thought to be one
of the culprits explaining this failure.

\subsection{Postcolonial Borno, p.170}

In a similar vein, in 2011, a Sudanese provincial border created by the British
became the international boundary separating North Sudan and South Sudan. The
current dispute over the Sudanese province of Abyei shows to what extent the
question of the regional borders is fundamental.

\subsection{Postcolonial Borno, p.173-174}

The official position was that to facilitate the development of its western
half, Borno was split into two parts. The creation of Yobe State was nor
officially triggered by ethnic strife, but the choice of the capital of the new
state seemed to reveal underlying local conflicts. In 1991 the main city of the
new state, Potiskum, was mainly inhabited by Ngizim, Bole and Karekare
populations. The preference for Damaturu, mainly inhabited by Kanuries, as the
state capital of the new state can be explained by the pressure of members of
the political elite of Borno State to preserve a Kanuri grip on Yobe. The Shehu
of Borno, Mustapha Amin, was particularly displeased with the creation of Yobe.
As he inherited the Kanemi throne, he was the main spiritual and political
authority wishing to safeguard the territorial integrity of his legacy. The
choice of Damaturu can be seen as a compromise from the Babangida administration
willing to offer some form of compensation to the defenders of Borno State. The
Emir of Damaturu, like his counterpart in Maiduguri, is also an heir to the
nineteenth-century rulers of the kingdom of Borno and styles himself Shehu;
titles from the Shehus' court in Maiduguri were even duplicated in Damaturu. 

The similarity between Borno and Sokoto in 1991 is striking. When confronted with
the threat of having Sokoto State divided, the Sultan of Sokoto opposed the
Babangida regime. The Sultan was not the only opponent of this new state but his
intervention contributed to the victory of the unitary movement.

\subsection{Postcolonial Borno, p.179}

Politicians eager to obtain support from their voters did not hesitate to call
on Kanuri feelings. The most striking example is the Governor of Yobe state
who, in 2001, desired the rise of a Kanuri nation. The Lagos newspaper
\textit{This Day} reported:

...

Governor Bukar Abba Ibrahim said the unity he envisaged would accord due
recognition to the Kanuri heritage of a superior civilisation, bequeathed by the
Kanem-Borno Empire which should fire the spirit of the descendants and promote
the socio-economic advancement of the Kanuri nation of today, and of generations
unborn.

\subsection{Postcolonial Borno, p.180}

This phenomenon, which was already widely studied in 1950s Nigeria, reinforced
some of the pre-existing ethnic identities. The Borno Youth movement mentioned
before was an example of these ethnic politics. Political parties could only
hope to obtain power of they based their discourse on ethnicity. 

Bounded territoriality was never totally substituted for ethnicity. Indeed, some
Kanuri-speaking inhabitants of Niger and Cameroon tend also to recognise
themselves in these transnational cultural identity. The term `identity' is used
here on purpose for its vague meaning as the Kanuri-speakers present in the
neighbouring countries recognise cultural links with Borno. For example, in
2009, Kanuri delegates from Cameroon, Chad and Niger attended the coronation of
the new Shehu of Borno and recognised Shehu Abubakar Ibn Garbai as their
historical overlord. The transborder links evoked above always refer to a vague
and distant past when the kings of Kanem-Borno ruled over the whole Lake Chad
basin.

...

The vague memory of a glorious distant past is thus entertained by Cameroonian,
Chadian and Nigérien citizens regardless of the nineteenth-century history of
Borno. As they were once part of the kingdom, the ancient political links can be
re-activated in the name of a common history and culture. The fact that these
regions were no longer part of the Kanemi kingdom during the nineteenth century
is not relevant. Borno is no longer a region or a state but has become a
cultural area with all the vagueness that entails. A certain form of ahistorical
territoriality is thus promoted across the twenty-first century borders.

Twentieth-century examples of cooperation across international borders among
Kanuris would tend to demonstrate that Borno still has an influence on its
former empire. The Kanuris and Kanembus living in the area once dominated by the
empire of Kanem-Borno have a clear sense of belonging to  an assumed Kanuri
nation. Ade Adefuye, Nigerian historian, found that a certain `Kanuri factor' is
responsible for explaining the Nigerian interventions during the Chadian civil
war. Indeed, Nigeria supported a Kanuri Chadian, Aboubakar Abdel Rahman, in 1977
when the latter founded the `Third Liberation Army'. According to Virginia
Thompson and Richard Adloff, Maiduguri's airport was even used by Gaddafi's
planes to bomb N'Djamena in 1980. For Thompson and Adloff, the Nigerian
authorities were unaware of this military help.

\subsection{Postcolonial Borno, p.186}

The Kanuris seem to have provided Nigeria with more statesmen than their share
of the Nigerian population would indicate. As mentioned above, in the context of
the `Northernisation' of politics in the 1950s, numerous Bornoans were part of
the civil service. How can we explain the disproportion between the relatively
high number of Kanuris holding positions in Nigeria and their minor demographic
importance? The first explanation offered by Paden was that the presence of
Kanuri civil servants endured the continued loyalty of Borno to the `North' of
Ahmadu Bello. The presence of Bornoans in the Nigerian administration and army
was not only a phenomenon in the 1950s. Other Kanuris from Borno such as Kashim
Ibrahim, Ibrahim Imam and Baba Gana Kingibe, and Shuwa Arabs such as Musa
Daggash, distinguished themselves in the recent history of Nigeria. It is clear
that a statesmanship tradition exists in Borno because of the prolonged
existence of a state in the Chad basin. The infamous dictator Sani Abacha was a
Kanuri from Kano. However, in the absence of detailed studies of these
politicians, this argument only remains a hypothesis.

\section{Freedom and Authority in French West Africa
\citep{delavignette2018freedom}}

\subsection{The Native Chiefs, p.74-75}



\pagebreak

\bibliographystyle{agsm}
\bibliography{lib.bib}

\end{document}
