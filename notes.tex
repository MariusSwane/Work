\documentclass[12pt]{article}
\usepackage{graphicx}
\usepackage{caption}
\usepackage{natbib}
\usepackage{authblk}
\usepackage[utf8]{inputenc}
\usepackage{setspace}
\usepackage{rotating}
\usepackage[british]{datetime2}
\usepackage{amssymb}
%\usepackage{hyperref}
\usepackage{url}

\renewcommand{\harvardurl}{\textbf{URL:} \url}
\renewcommand\Affilfont{\itshape\small}

\title{Notes from the literature}
\author[1]{Marius Swane Wishman}
\affil[1]{Department of Sociology and Political Science, NTNU}

\date{\today}

\providecommand{\keywords}[1]
{
	\small	
	\textbf{\textit{Keywords---}} #1
}

\begin{document}

\maketitle

\begin{abstract}

	This is a document for keeping notes from what I read in one place.
	Section headers are works (books or articles etc.), and should contain
	the reference. Subsections describe, chapter or heading from the source,
	and the page number. Paragraphs are individual quotes, except when
	written in bold font, which is commentary by me.

\end{abstract}

\pagebreak

\tableofcontents

\pagebreak

\onehalfspacing
\section{Nation Building - Why Some Countries Come Together While Others Fall 
	Apart \citep{Wimmer_2018}}

\subsection{Introduction, p. 3}

Locals might resist a national government that intruded more into their daily
lives that did its colonial predecessor.  Political elites competed over who
controlled the new center of power.  Economic poverty, artificially drawn
boundaries, the legacies of colonial divide-and-rule policies, and the weakness
of postcolonial states made national political integration difficult.

\subsection{Introduction, p. 8}

Such voluntary organizations facilitate building alliances across ethnic
communities and regions, I will argue.  They bundle individual interests, as it
were such that politicians or state agencies can respond to them more easily.
In patronage systems, by contrast, each alliance needs to be managed separately:
a patron needs to provide political protection or government favors to each of
his clients on an individual basis.

\bigskip

How far such voluntary organizations have developed matters especially in early
years after a country transitions to the nation-state - when an absolutist
monarchy is overthrown or when a former colony becomes independent.  If a dense
web of such organizations haves already emerged, the new power holders can tap
into these networks to extend relationships of authority and support across the
country. 

\subsection{Introduction, p. 17}

Rather, .I both high diversity and low capacity to provide public goods emerge
in societies without a historical legacy of centralized states, as argued
throughout the preceding chapters.

\subsection{Chapter 1, p. 31}

The reach of political alliance networks-rather than communication in a shared
language per se - turns out to be crucial.  Where these networks were confined
by political boundaries such as imperial provinces, nationalists divided the
space of a shared language, as in Latin America.  Where language barriers within
imperial domains hampered the establishment of such ties, linguistic communities
were imagined as nations.  This happened in Romanov Russia.  Where civil
societies flourished early, as in Switzerland, or where states were
exceptionally capable of providing public goods, political networks stretched
across ethnic divides and the nation was imagined as polyglot.  I other words,
the contours of political alliance networks determine which communities emerge
as nations.

\subsection{Chapter 1, p. 66}

Because voluntary associations facilitate horizontal linkages across a
territory, the more associational networks have developed during the period
leading to the creation of a nation-state the easier it will be for the new
governing elites to build alliances across ethnic divides bu relying on these
networks.

\textbf{Conversely, HSE's create \textit{"vertical networks."}}
\textbf{Potentially facilitating both mobilization of said networks, but also
patronage networks.} 

\subsection{Chapter 2, p. 97}
The administrative footprint, however, of both the Ottoman Empire and the
Zanzibari Sultanate was extremely light and consisted of a handful of
representatives and tax collectors per town only.  The city-states never managed
to subdue the nomads of the hinterland.  These could mobilize tens of thousands
of warriors against which the feebly fortified coastal cities stood little
chance (Lewis 1988:34-35).  Only the Majeerteen Sultanate (a clan of the Darood
family) situated at the Horn managed to gain effective control over the interior
and represented a territorial state comparable to some of the more powerful and
populous Tswana kingdoms.  It remained independent from both the Ottoman Empire
and the Oman Sultanate by signing a treaty with Britain in 1839.  In the 1920s ,
Italy ended its long-lasting sovereignty through military conquest.

\subsection{Chapter 3, p. 103}

This fragmented, embryonic, politicized administration had virtually no tac base
beyond the tariffs it collected from imports and exports-a legacy of the
resistance that the interior clans had mounted against any form of taxation, in
turn the consequence of long centuries of living without a state.  Somalia
therefore became one of the most aid-dependent countries in Africa

\subsection{Chapter 3, p. 105}

The expanded political arena also offered incentives to found new parties with
both northern and southern clan elements, as the short-lived Somali Democratic
Union and the more stable Somali National Congress show (Lewis 1988: 176.  The
latter included northern Dir and Isaaq clans bu also important Hawiye elements
from the former Italian parts of the country-all united in the attempt to
counter the dominance of Darood, particularly Majeerteen politicians during the
first decade of independence.  Indeed, it seems that the politically dominant
clans during that first decade were the Hawaiye from central Somalia, the Isaaq
from the north and especially the Majeerteen from the former eponymous sultanate
on the Horn (Laitin and Samatar 1987: 92), who supplied one out of two
presidents and two prime ministers out of three.

\subsection{Chapter 3, p. 105}

Further fragmentation followed.  In the 1969 elections 62 parties, most of
which represented narrower clan interests, competed with each other.  The more
broadly based SYL won again.  Indicating the opportunistic nature of politics in
independent Somalia, all but one member of the opposition parties defected after
the election to join the government party SYL, hoping to get a piece of the pie
and distribute it to their clan supporters.

\subsection{Chapter 4, p. 148}

Third, the memories of independent statehood provided politically ambitious
Polish noblemen with a model for the future.  They thus confined their networks
of alliances to other Polish nobles (who made up as much as 20% of the
population in order to one day achieve the dream of renewed independent
statehood, a dream first couched in terms of rights to dynastic succession
rather than modern nationalist discourse.  The peasant population, whether
speaking Polish, Belorussian, Lithuanian, or Ukrainian, as well as Jewish town
dwellers, remained largely excluded from these networks of agitation and
mobilization and thus indifferent to the proto-nationalist cause of the Polish
nobility.

\section{How State Presence Leads to Civil Conflict \citep{Ying_2020}}

\subsection{Abstract}

\textbf{Argues that conflict occurs when states attempt to penetrate areas or 
	groups it has not controlled previously.}

\textbf{He tests both at \textit{province} and \textit{"ethnic group"}
level, using the accuracy of census data in the first and global ground
transportation data in the second analysis.}

\subsection{State Weakness and Civil Conflict}

Two concepts are crucial in understanding state power: (1) state capacity, i.e.
the overall power of a nation-state to administrate its territory (Evans,
Rueschemeyer, and Skocpol 1985), and (2) state presence, i.e. the allocation of
state power at the sub-national level. The lower the state-capacity the more
likely is state presence to be insufficient and uneven across regions. Existing
literature terms this phenomenon as limited statehood (Risse 2011) or
incomplete sovereignty/inefficient governance (Lee 2018).

\subsection{From State Capacity to Civil Conflict}

\textbf{Similar to Koren and Sarbahi 2018 but focuses on low intensity state-based
violence instead of civil war.}

\section{A Short History of the Cartography of Africa \citep{Stone1995}}

\subsection{Chapter 5 The Nineteenth Century: Golden Age of the Cartography of
Imperialism, p. 47-48 }

\textbf{The 19th century is a transition period leading up to the revolution
of the 20th  century cartography of colonialism.}

\textbf{Cartography in Africa is still a mix of measurement, less accurate
observations, word of mouth,  previous maps and sources, educated guesses and
pure conjecture.  Nevertheless a distinct improvement on the maps of previous
periods.}

\textbf{Maps are generally published in the year following the return of an
expedition.  However, expeditions could last for years, so observations could be
several years after the fact by the time they met print.}

\subsection{Southern Africa, p. 49-50}

It was the use of instruments, including a sextant, an artificial horizon,
chronometer, compass and measuring chain for traversing but corrected by means
of daily observation of latitude and occasional observation of longitude (de
Smidt 1896), which resulted in John Barrow's influential 
\textit{"General chart of the Colony of the Cape of Good Hope"}, accompanying
his \textit{"Travels into the Interior of Southern Africa"}, published in 1801
(Penn 1994).

The missionary thrust northwards through the eastern Kalahari was continued by 
Robert Moffat, whose map of SOUTH AFRICA \textit{"compiled for the Revd. R. 
Moffat's work by James Wyld"}, published in Moffat's \textit{"Missionary Labours
and Scenes in Southern Africa"} (1842) improved considerably on Campbell's 
depiction in the amount of geographical and ethnographic information and its 
accuracy. \textbf{While our data do not include the maps in question, we do have
two maps by James Wyld among our sources.}

The information which was forthcoming in the aftermath of the Great Trek duly
made its appearance in contemporary maps, notably in the maps of John Arrowsmith
(Schrive 1965). \textbf{Our data includes four maps by John Arrowsmith.}

\subsection{East Africa, p. 53}

A \textit{"Sketch Map of Eastern Africa"} accompanied Speke's journal, engraved
by W. \& A.K. Johnston. It contained much less of the detail which he had
recorded and which was contained in the map accompanying Burton's account, with
one significant difference. It showed the 'Mountains of the Moon 6000 to 8000
feet by estimation' as a horseshoe forming the catchment at the north end of
Lake Tanganyika.  \textbf{Our data includes a map of Africa by the same
	Johnstons from 1961 (two years after the first publication from Bruton
	and Speke's expedition). The map is clearly based on the one mentioned
	by \citet{Stone1995}, and includes the infamous Mountains of the Moon,
	as described above, and with Speke's name attached. In our next map by
the Johnstons (from 1879) the erroneous mountains have been removed. In total
our data contains five maps by one or more of the Johnston's.}

\subsection{East Africa, p. 58}

Further detailed observations by subsequent explorers followed, in turn to be
compiled into route maps which could be used as sources by the great nineteenth
century atlas publishers such as W. \& A.K. Johnston, John Bartholomew and the
Society for the Diffusion of Useful Knowledge, but the hydrographical framework
had now been provided for east and central Africa. \textbf{As mentioned
previously, our data contains five Johnston maps. We also have one map by John
Bartholomew and two maps by the Society for the Diffusion of Useful Knowledge.}

\subsection{West Africa, p. 59-60}

The map which accompanies Denham and Clapperton's \textit{Narrative of Travels
and Discoveries in Northern and Central Africa} (1826) extends from the Gulf of
Sirtees to the Bight of Benin and is a typical explorer's transect, with
comments about wells, oases, wadis and villages. The western shores of Lake Chad
are firmly mapped, but the northeast shores are shown tentatively and the lower
course of the Niger is speculative. The map made an immediate impact with those
searching fir source material to update their maps of Africa, as can be seen in
John Cary's \textit{New Map of Africa Exhibiting the Recent Discoveries
constructed from the most recent travels}, published in 1828. Denham, Clapperton
and Oudney's route stands out by comparison with the scarcity of information
elsewhere in the Sahara, as it does on Sidney Hall's map of Africa of 1829.
\textbf{Our data contains one map by Cary and one by Hall. However, not they are
	not the ones mentioned here.}

\subsection{The Momentum Sustained, p. 61-62}
The published findings of the explorers of West Africa fed rapidly into the
mainstream of map and atlas publishing. In larger scale maps which carried
forward much of the detail of the explorer's originals, the task of combining a
large number of sources into a single mosaic at the selected scale was clearly a
major work of cartographic compilation. This is very well seen in the atlas of
\textit{Maps of the Society for the Diffusion of Useful Knowledge} (1844), which
contains a small scale map of the African continent, supplemented by larger
scale maps of parts of the continent. West Africa is divided into two sheets (I
and II) at a scale of c.1:5,600,000 extending from Timbuktu to the Guinea Coast
and from Cape Verde to Lake Chad. The extremely detailed content of the two maps
includes the routes of more than twenty explorers, including Beaufort, Gray,
Mollien, Winterbottom, Caillié, Park O'Byrne, Liang, Dupuis, Bowdien, Dochard,
Houghton, Dickson, Clapperton, Denham, Oudney, Lander, Hornemann, Laird, Allen,
Oldfield, Coulthurst and Toole. \textbf{Not one of the two included in our data.}

\subsection{The Momentum Sustained, p. 62}

The maps were published in 1839, (earlier then the atlas) and the explorers'
dates range from Houghton in 1791 to Oldfield in 1836. \textbf{The atlas
referred to is the one in the above subsection, published in 1844. In other
words, in this illustrative case, maps were published three years after the
expedition and was published in an atlas after eight.}

\subsection{The Momentum Sustained, p. 63}
These `stay-at-home scholars' (Bridges 1987b) re-examined the classical sources,
and they sought to learn from anyone with recent experience of Africa. The maps
of armchair geographers such as William Desborough Cooley, James McQueen and
A.G. Findlay were published by the RGS and are properly part of the evolving map
of the African interior in the 1850s.

\subsection{The Momentum Sustained, p. 65}
By the End of the nineteenth century, mapping by military personnel had been
instituted elsewhere in Africa, not only by French and British military
authorities. However, in the context of the nineteenth century as a whole, it
was peripheral in its location and limited in its extent, by comparison with the
work of the explorers. 

\subsection{The Characteristics of the Cartography of Imperialism, p. 69}

The imperial relationship with Africa was essentially international in
character, in the sense that the European powers shared a common purpose in the
continuation of access to trade at the coast of Africa. ... On the other hand,
the colonial period which was to follow can be differentiated by parochial
exclusivity and nationalism within colonial boundaries in a continent
partitioned among seven European powers, in contrast to the common attitudes and
access among European powers which characterised the previous imperial
relationship. The Europeans who made maps of Africa prior to colonial partition
subscribed to common attitudes of the imperial powers. The subject matter of
their maps cannot easily be differentiated by their particular European country
of origin. They were a part of an imperial relationship which was international
in character.

\subsection{The Characteristics of the Cartography of Imperialism, p. 70}

Professor Bridges (1982) is able to talk about unofficial planning amongst
Europeans for direct intervention in East Africa by 1876, interventions which
would amount to interference in the lives of Africans. Such actions were
thought of as part of a laudable civilizing process involving European
organizational and technological skills. It was in this context that the
compilation of maps of the highest possible scientific calibre, using
instrumentally determined data to measure and locate places with optimum
accuracy, contributed to legitimizing European penetration and even
Interference in Africa. This background situation influenced the form of the
growing number of explorer's maps in the late nineteenth century.

\textbf{The rest of the page is very relevant as well, but I cannot rewrite the
whole thing here.}

\subsection{The Nineteenth Century Atlas Map of Africa, p. 72-73}

The picture of Africa which atlas publishers were presenting at the beginning of
the nineteenth century was not consistent. For example, de la Rochette's
\textit{Africa} dated 1803, published by W. Faden (Tooley 1969) incorporates
recent explorations in West Africa. It shows an eastward flowing upper Niger
(but not the as yet undiscovered outlet to the Gulf of Guinea) and the sources
of the Blue Nile. South of the Equator, it continues to rely heavily on the long
standing Portuguese sources for the lower Congo and Zambezi basin, with
extensive intervening otherwise blank areas containing a curious mix of textual
descriptions, either of an ethnographic nature or about potential routs. On the
other hand, James Cundee's \textit{New Royal Atlas Engraved from the best
modern authorities}, of 1810, includes a map of Africa \textit{Engraved for the
Revd. Mr. Evans's New Geographical Grammar} in 1809, which is much more
concerned with ethnic names distributed evenly across the entire face of the
continent, albeit with the recent knowledge of the drainage of West Africa as a
part of the background information. 

\textbf{The map published by W. Faden is not in our data, but two later ones
are. Cundee is not in our data.}

\subsection{The Nineteenth Century Atlas Map of Africa, p. 73-74}

Nevertheless, the great weight of new information becoming available from
explorers' maps was clearly making an impact.This is particularly well seen in
the four maps compiled in 1839 under the auspices of the Society for the
Diffusion of Useful Knowledge, founded in 1826 to promote 'the moral improvement
of the great body of the population', in part by means of publishing and
distributing suitable books (Smith 1974). Their atlas entitled \textit{Maps of
the Society}... published by Chapman and Hall in 1844, contained four maps of
Africa mostly dated to 1839, one of the entire continent, one of South Africa
(dated 1834), plus cover of West Africa at a larger scale in two maps. The
contents of the two maps of West Africa have already been described in the
context of the exploration of West Africa, but an aspect worthy of mention is
their total reliance on recent explorers as their sources. This is also true of
the relatively large scale maps of South Africa but less so in the case of the
map of the continent on a single sheet.

\textbf{This map of the whole continent is in our data.}

\subsection{Additional comments by me}

\textbf{\citet{Stone1995} builds the chapter covering the 19th century by
	covering a selection of `a few influential or characteristic maps. ...
	The choice of maps has been influenced by the valuable summary history
	of African exploration by Professor R.C. Bridges (1991).' I interpret
	this to mean that the emphasis is on quality and the maps drawn by the
	explorers themselves.  The overlap between our data and the maps covered
	by \citet{Stone1995} should therefor give an indication of the quality
	of our data.  At least in terms of including quality sources.}

\textbf{Of the 47 maps referred to in \citet{Stone1995} only six are included in
	the GeoISD. However, given \citet{Stone1995}'s emphasis on maps
	published by explorers and our reliance on compiled atlases this is no
	surprise. A different way to reckon would be to include maps by the same
	author or drawn by, compiled by, engraved by or published by the same
	author across \citet{Stone1995} and the GeoISD. This count should be
	able to capture some of the maps as they have ``trickled down" to atlas
	compilation. It is additionally a further ``fuzzy" measure of quality as
the quality of maps varied far more between mapmakers than between maps by the
same author. By this more inclusive count close to half the maps are ``partially
included" in the GeoISD (22/47).}
	
\section{The art of not being governed \citep{Scott2009}}

\pagebreak

\bibliographystyle{agsm}
\bibliography{lib.bib}

\end{document}
